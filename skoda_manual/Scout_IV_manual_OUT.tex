%\stackrel{?}{=}
%\begin{alignat}{n}

\documentclass[a4paper]{article}

\usepackage[document]{ragged2e}

% Set margins.
\usepackage[top=1.cm, bottom=1.5cm, left=1.5cm, right=1.5cm]{geometry}

% \usepackage[polish]{babel}
\usepackage[utf8]{inputenc}
\usepackage[T1]{fontenc}  % umożliwia wstawianie znaków z kodowania 8-bitowego (domyślnie jest 7-bitowe, przez co część znaków musi być wstawiana za pomocą kombinacji np. \l{} - "ł", \"{o} - "ö"; zob. https://tex.stackexchange.com/a/677/44391)
\usepackage{polski}

\usepackage{array}
\usepackage[shortlabels]{enumitem}
\usepackage[hidelinks]{hyperref} 

\usepackage{amsmath}
\usepackage{textcomp} 
\usepackage{amsfonts}
\usepackage{gensymb}

\usepackage{amssymb}  % \checkmark, \blacktriangleright
\usepackage{tipa}  % \textrptr

\newcommand{\notion}[1]{\textbf{{#1}}}
\newcommand{\set}[1]{\mathbf{{#1}}}


\usepackage{amsthm}
\theoremstyle{remark}
\newtheorem{theorem}{Twierdzenie}[section]
\newtheorem{lemma}[theorem]{Lemat}
\newtheorem{example}{Przykład}

% FIXME: wyróżnić graficznie (np. tło)
\newenvironment{extended}{}{}

\usepackage{graphicx}
\usepackage{subcaption}
\usepackage[inkscapeformat=png]{svg}  % For compilation pdflatex needs the command line switch --shell-escape.
\newlength{\picwidth}

% Facilitate the conditional compilation
\usepackage{etoolbox}  % umożliwia kompilację warunkową [\newtoggle, \toggletrue, \togglefalse, \iftoggle, ...]
\usepackage{xstring}  % FIXME: co robi?

\newtoggle{GFXDEBUG}  % zdefiniuj flagę `GFXDEBUG`
\togglefalse{GFXDEBUG}
%\toggletrue{GFXDEBUG}  % nadaj fladze `GFXDEBUG` wartość `true`

% Wyświetlenie grafiki z opcją `draft` spowoduje wyświetlenie ramki zamiast faktycznego obrazka - co przyspiesza "robocze" generowanie dokumentu:
%   \includegraphics[draft]{...}
% Po zdefiniowaniu polecenia, którego wartość wynosi "draft" w przypadku generowania "roboczego" i "" w przypadku generowania "normalnego" i umieszczeniu go w każdym poleceniu `\includegraphics`, możemy w efektywny sposób zmieniać sposób generowania dla całego dokumentu:
%   \includegraphics[\draftgraphics]{...}
\iftoggle{GFXDEBUG}{
	% Debug ON
	\newcommand{\draftgraphics}{true}
}{
	% Debug OFF
	\newcommand{\draftgraphics}{false}
}

\newcommand{\envalias}[2]{\newenvironment{#1}{\begin{#2}}{\end{#2}}}
\envalias{itemizeTriangle}{itemize}
\envalias{itemizeArrow}{itemize}
\envalias{itemizeTick}{itemize}

\newcommand{\genericItem}[1]{\item[#1]}
%\newcommand{\itemizeTriangle}{\itemize}
%\newcommand{\itemizeArrow}{\itemize}
%\newcommand{\itemizeTick}{\itemize}
\newcommand{\itemTriangle}{\genericItem{$\blacktriangleright$}}
\newcommand{\itemArrow}{\genericItem{\textrptr}}
\newcommand{\itemTick}{\genericItem{$\checkmark$}}

%\title{Simple Sample}
%\author{My Name}
%\date{\today}

\begin{document}
%    \maketitle

\section{Przeglądy pojazdów}

\subsection{Przedni obszar pojazdu}

% FIXME: Rys.

A Pod przednią szybą:
\begin{itemizeTriangle}
	\itemTriangle Kamera do systemów wspomagających
	\itemTriangle Czujnik światła do automatycznego obwodu świateł jazdy \guillemotright strona 48
	\itemTriangle Czujnik deszczu do automatycznego uruchamiania wycieraczek \guillemotright strona 54
	\itemTriangle Czujniki wilgotności i promieniowania słonecznego
\end{itemizeTriangle}
B Szyba przednia – ogrzewanie \guillemotright strona 60
C Wycieraczka szyby przedniej – obsługa \guillemotright strona 54.
D Dźwignia zwalniająca klapę komory silnika (pod klapą) \guillemotright strona 167
E Układ zmywania reflektorów \guillemotright strona 54
F Osłona grilla chłodnicy (w zależności od wyposażenia pojazdu):
\begin{itemizeTriangle}
	\itemTriangle Przedni czujnik radarowy do systemów wspomagających
	\itemTriangle Kamera do systemów wspomagających
\end{itemizeTriangle}
G Pokrywa do mocowania wkręcanego haka holowniczego \guillemotright strona 138
H Czujniki ultradźwiękowe do systemów wspomagających
I Światła przeciwmgielne – obsługa \guillemotright strona 49
J Koła:
\begin{itemizeTriangle}
	\itemTriangle Opony i felgi \guillemotright strona 178
	\itemTriangle Wymiana koła i podnoszenie samochodu \guillemotright strona 179
	\itemTriangle Zestaw do naprawy opon \guillemotright strona 181
	\itemTriangle Ciśnienie w oponach \guillemotright strona 182
	\itemTriangle Wskaźnik ciśnienia w oponach \guillemotright strona 183
\end{itemizeTriangle}
K Reflektor (w zależności od wyposażenia pojazdu):
\begin{itemizeTriangle}
	\itemTriangle Obsługa \guillemotright strona 49
	\itemTriangle Asystent świateł drogowych Light Assist \guillemotright strona 51
	\itemTriangle Asystent reflektorów Dynamic Light Assist \guillemotright strona 51
\end{itemizeTriangle}
L Klapka ładowania akumulatora \guillemotright strona 162
M Klamki zewnętrzne:
\begin{itemizeTriangle}
	\itemTriangle Otwieranie drzwi \guillemotright strona 27
	\itemTriangle Blokowanie bezkluczykowe (KESSY) \guillemotright strona 25
\end{itemizeTriangle}
N Lusterko zewnętrzne (w zależności od wyposażenia pojazdu):
\begin{itemizeTriangle}
	\itemTriangle Obsługa \guillemotright strona 39
	\itemTriangle Kamera do systemów wspomagających
\end{itemizeTriangle}
O Okna drzwi bocznych – obsługa \guillemotright strona 28
Funkcjonalność czujników i kamer
\begin{itemizeArrow}
	\itemArrow Utrzymuj czujniki i kamery w czystości w systemach wspomagających \guillemotright strona 17.
\end{itemizeArrow}


\subsection{Tylny obszar pojazdu}

% FIXME: Rys.


A Tylna szyba – ogrzewanie \guillemotright strona 60
B Wycieraczka i spryskiwacz szyby tylnej – obsługa \guillemotright strona 54
C Uchwyt pokrywy bagażnika (w zależności od wyposażenia):
\begin{itemizeTriangle}
	\itemTriangle Klapa z obsługą ręczną \guillemotright strona 31
	\itemTriangle Klapa z obsługą elektryczną \guillemotright strona 31
	\itemTriangle Kamera do systemów wspomagających
\end{itemizeTriangle}
D Czujniki radarowe dla systemów wspomagających (w zderzaku)
E Czujniki ultradźwiękowe do systemów wspomagających
F Bezdotykowa obsługa elektrycznej pokrywy bagażnika \guillemotright strona 33
G Pokrywa do mocowania wkręcanego haka holowniczego \guillemotright strona 138
H Pokrywa wlewu paliwa:
\begin{itemizeTriangle}
	\itemTriangle Otwieranie \guillemotright strona 170
	\itemTriangle Naklejka z wartościami ciśnień w oponach \guillemotright strona 182
	\itemTriangle Naklejki z zalecanym paliwem
	\itemTriangle skrobaczka do lodu
\end{itemizeTriangle}
I Relingi dachowe \guillemotright strona 197
J Dach przesuwno-uchylny \guillemotright strona 29
Funkcjonalność czujników i kamer
\begin{itemizeArrow}
	\itemArrow Utrzymuj czujniki i kamery w czystości w systemach wspomagających \guillemotright strona 17
\end{itemizeArrow}

\subsection{Miejsce kierowcy}

% FIXME: Rys.

A Wskaźnik asystenta zmiany pasa ruchu Side Assist \guillemotright strona 150
B Dźwignia otwierania drzwi \guillemotright strona 27
C Włącznik świateł \guillemotright strona 49
D Kratki nawiewu powietrza
E Dźwignia obsługowa (w zależności od wyposażenia pojazdu):
\begin{itemizeTriangle}
	\itemTriangle Kierunkowskazy i światła drogowe \guillemotright strona 49
	\itemTriangle Tempomat \guillemotright strona 145
	\itemTriangle Ogranicznik prędkości \guillemotright strona 144
	\itemTriangle Funkcja automatycznego włączania świateł drogowych \guillemotright strona 52
	\itemTriangle Asystent reflektorów \guillemotright strona 52
\end{itemizeTriangle}
F Przyciski / pokrętła na kierownicy wielofunkcyjnej \guillemotright strona 38
G Cyfrowy zestaw wskaźników \guillemotright strona 62
H Dźwignia obsługowa:
\begin{itemizeTriangle}
	\itemTriangle Wycieraczki i spryskiwacze \guillemotright strona 54
\end{itemizeTriangle}
I Przycisk rozruchu silnika \guillemotright strona 132
J Kierownica z sygnałem dźwiękowym / z czołową poduszką powietrzną kierowcy \guillemotright strona 46
K Dźwignia do ustawiania kierownicy \guillemotright strona 38
L Obsługa układu automatycznego utrzymania odległości \guillemotright strona 147
M Odblokowanie pokrywy komory silnika \guillemotright strona 167
N Przycisk elektrycznej klapy bagażnika \guillemotright strona 31
O Obsługa lusterek zewnętrznych \guillemotright strona 39
P Obsługa szyb \guillemotright strona 28

\subsection{Konsola środkowa i fotel pasażera}

% FIXME: Rys.

A Infotainment (w zależności od wyposażenia pojazdu):
\begin{itemizeTriangle}
	\itemTriangle Columbus \guillemotright strona 95
	\itemTriangle Bolero \guillemotright strona 68
\end{itemizeTriangle}
B Przyciski (w zależności od wyposażenia pojazdu):
\begin{itemizeTriangle}
		\itemTriangle XX Szybki dostęp do ustawień niektórych systemów pojazdu (w zależności od wyposażenia pojazdu):
		\itemTriangle XX Wyłączanie kontroli trakcji ASR \guillemotright strona 141
		\itemTriangle XX Włączanie programu stabilizacji ESC Sport \guillemotright strona 141
		\itemTriangle XX Układ monitorowania wnętrza \guillemotright strona 27
		\itemTriangle XX Wskaźnik ciśnienia w oponach \guillemotright strona 183
		\itemTriangle Dostęp do innych ustawień pojazdu
		\itemTriangle Dostęp do wyboru i regulacji systemów wspomagania kierowcy
	\itemTriangle XX Wybór trybu jazdy \guillemotright strona 135
	\itemTriangle XX Menu z asystentami parkowania \guillemotright strona 154
	\itemTriangle XX Przycisk świateł awaryjnych \guillemotright strona 49
	\itemTriangle XX Centralne ryglowanie \guillemotright strona 24
	\itemTriangle XX Wyświetlanie ekranu operacyjnego klimatyzatora na ekranie informacyjno-rozrywkowym \guillemotright strona 57
	\itemTriangle XX Szybka wentylacja / odszranianie przedniej szyby
	\itemTriangle XX Ogrzewanie szyby tylnej \guillemotright strona 60
\end{itemizeTriangle}
C Kratki nawiewu powietrza
D Dźwignia otwierania drzwi \guillemotright strona 27
E Wskaźnik asystenta zmiany pasa ruchu Side Assist \guillemotright strona 150
F Obsługa okna w drzwiach pasażera \guillemotright strona 28
G Dźwignia sterująca automatycznej skrzyni biegów \guillemotright strona 133
H Przyciski:
\begin{itemizeTriangle}
	\itemTriangle XX Auto Hold \guillemotright strona 140
	\itemTriangle XX Hamulec postojowy \guillemotright strona 139
\end{itemizeTriangle}

\subsection{Komora silnika}

% FIXME: Rys.

A Zbiornik wyrównawczy chłodziwa silnika spalinowego \guillemotright strona 168
B Bagnet do pomiaru poziomu oleju silnikowego \guillemotright strona 167
C Wlew oleju silnikowego \guillemotright strona 167
D Zbiornik wyrównawczy płynu chłodzącego w układzie wysokiego napięcia \guillemotright strona 168
E Zbiornik płynu hamulcowego \guillemotright strona 139
F Skrzynka bezpieczników \guillemotright strona 177
G Zbiornik płynu do spryskiwaczy \guillemotright strona 55

\section{Lampki kontrolne}

\subsection{Sposób działania}

Dodatkowe lampki kontrolne

Razem z niektórymi lampkami kontrolnymi na wyświetlaczu świeci się również dodatkowa lampka kontrolna:
XX - Niebezpieczeństwo
XX - Ostrzeżenie

\subsection{Przegląd lampek kontrolnych}

Po włączeniu zapłonu na krótko zapala się kilka lampek kontrolnych informujących o sprawności układów pojazdu. Jeśli sprawdzane układy są w prawidłowym stanie, odpowiednie lampki kontrolne gasną kilka sekund po włączeniu zapłonu lub po uruchomieniu silnika.

Dalsze informacje \guillemotright strona 12, Sposób działania.


Symbol Znaczenie
XX	wraz z inną kontrolką sygnalizuje ostrzeżenie \guillemotright strona 12.
XX	Niezapięty pas bezpieczeństwa z przodu i z tyłu\guillemotright strona 41.
XX	Proaktywny system ochrony pasażerów interweniuje \guillemotright Sposób działania
XX	akumulator 12 V nie jest naładowany \guillemotright strona 173. \\ Niski poziom naładowania akumulatora 48 V \guillemotright Rozwiązywanie problemów \\ Wraz z XX - Usterka silnika\guillemotright strona 169,\guillemotright strona 173.
XX	Zbyt niskie ciśnienie oleju silnikowego\guillemotright strona 167.
XX	Poziom oleju silnikowego za niski\guillemotright strona 168.
XX	Zbyt niski poziom płynu chłodzącego\guillemotright strona 169. \\ Temperatura płynu chłodzącego za wysoka\guillemotright strona 169.
XX	Zbyt niski poziom płynu hamulcowego\guillemotright strona 139. \\ Elektromechaniczny wzmacniacz siły hamowania zakłócony\guillemotright strona 142. \\ Wraz z XX - układ hamulcowy i ABS są zakłócone\guillemotright strona 142.
XX	Miga -- Parkowanie na stoku o zbyt dużym nachyleniu\guillemotright strona 140. \\ Błyszczy -- Hamulec postojowy włączony\guillemotright strona 139.
XX	Świeci się - usterka wspomagania układu kierowniczego\guillemotright strona 38. \\ Miga - usterka blokady kolumny kierownicy\guillemotright strona 39.
XX	Automatyczna skrzynia biegów jest zakłócona\guillemotright strona 134. \\ Przegrzana automatyczna skrzynia biegów\guillemotright strona 134.
XX	Przejmij kontrolę natychmiast\guillemotright strona 149.
XX	AdBlue® poziom za niski \guillemotright Rozwiązywanie problemów
XX	AdBlue® system zakłócony \guillemotright Rozwiązywanie problemów
XX	Ostrzeżenie o niebezpieczeństwie kolizji\guillemotright strona 143.
XX	ACC nie zwalnia wystarczająco\guillemotright strona 146.
XX	Błąd w układzie sterowania silnika \guillemotright Rozwiązywanie problemów

XX	Wraz z inną kontrolką sygnalizuje ostrzeżenie \guillemotright strona 12.
XX	Zapas paliwa dotarł do obszaru rezerwowego.\guillemotright strona 171.
XX	Usterka / niski stan naładowania akumulatora 12 V pojazdu \guillemotright strona 173,\guillemotright strona 173,\guillemotright strona
173.
XX	Zbyt niski poziom wody w zbiorniku spryskiwacza szyby\guillemotright strona 55.
XX	Uszkodzona żarówka\guillemotright strona 51.
XX	Włączone tylne światło przeciwmgielne\guillemotright strona 50.
XX	Zbyt wysoki poziom oleju silnikowego lub Uszkodzony czujnik poziomu oleju silnika\guillemotright strona 168.
XX	Zatkany filtr cząstek stałych\guillemotright strona 169.
XX	Usterka hamulca postojowego\guillemotright strona 140.
XX	ABS zakłócony\guillemotright strona 142.
XX	Zużyte klocki hamulcowe\guillemotright strona 139.
XX	Przegrzana automatyczna skrzynia biegów\guillemotright strona 134. \\ Automatyczna skrzynia biegów jest zakłócona\guillemotright strona 134.
XX	Woda w filtrze paliwa \guillemotright Rozwiązywanie problemów
XX	AdBlue® system zakłócony \guillemotright Rozwiązywanie problemów
XX	AdBlue® poziom za niski \guillemotright Rozwiązywanie problemów
XX	Świeci się – usterka wspomagania układu kierowniczego \guillemotright Rozwiązywanie problemów \\ Miga – blokada kolumny kierownicy nie odblokowana \guillemotright Rozwiązywanie problemów \\ Miga – usterka blokady kolumny kierownicy \guillemotright Rozwiązywanie problemów
XX	Świeci się - usterka wspomagania układu kierowniczego\guillemotright strona 39. \\ Miga - blokada kolumny kierownicy nie odblokowana\guillemotright strona 39. \\ Miga - Zakłócona blokada kierownicy\guillemotright strona 39.
XX	KESSY - problem z uruchomieniem\guillemotright strona 132. \\ KESSY - nie znaleziono klucza\guillemotright strona 26.
XX	Zakłócony napęd na wszystkie koła \guillemotright Rozwiązywanie problemów
XX	Uszkodzone sterowanie silnika benzynowego\guillemotright strona 169.
XX	System kontroli emisji został zakłócony\guillemotright strona 170.
XX	Świeci się lub nie świeci się po włączeniu zapłonu – zakłócenie układu podgrzewania wstępnego silnika \guillemotright Rozwiązywanie problemów
XX	Włączona czołowa poduszka powietrzna pasażera\guillemotright strona 48. \\ Miga razem z XX - Wyłącznik kluczykowy do wyłączenia poduszki powietrznej ma usterkę\guillemotright strona 48.
XX	Poduszka powietrzna pasażera wyłączona\guillemotright strona 48

XX	Zakłócenie układu poduszki powietrznej\guillemotright strona 47. \\ Zakłócenie proaktywnego systemu ochrony pasażerów\guillemotright strona 152. \\ Świeci 4 s a następnie miga - poduszka powietrzna lub napinacz pasa wyłączony urządzeniem diagnostycznym\guillemotright strona 47.
XX	ASR wyłączony\guillemotright strona 142,\guillemotright strona 142.
XX	Świeci - zakłócony ESC lub ASR\guillemotright strona 142. \\ Miga - interweniuje ESC lub ASR\guillemotright strona 141.
XX	Front Assist dezaktywowany\guillemotright strona 143.
XX	Front Assist jest niedostępny .\guillemotright strona 143.
XX	ACC niedostępny\guillemotright strona 148.
XX	Lane Assist ingeruje\guillemotright strona 148.
XX	Włączone ogrzewanie lub wentylacja postojowa\guillemotright strona 59.

XX	Kierunkowskaz lewy\guillemotright strona 49,\guillemotright strona 51.
XX	Kierunkowskaz prawy\guillemotright strona 49,\guillemotright strona 51.
XX	Zapięty pas bezpieczeństwa z tyłu\guillemotright strona 41.
XX	Kierunkowskaz przyczepy\guillemotright strona 51.
XX	Światła przeciwmgielne włączone\guillemotright strona 49.
XX	Automatyczna skrzynia biegów znajduje się w trybie P.\guillemotright strona 133.
XX	Pojazd jest zabezpieczony przez Auto Hold\guillemotright strona 140.
XX	Lane Assist jest aktywowany i gotowy do ingerencji\guillemotright strona 148.
XX	ACC reguluje prędkość jazdy\guillemotright strona 146.
XX	Tempomat reguluje prędkość jazdy\guillemotright strona 145.

XX	Aktywny asystent podróży\guillemotright strona 149.
XX	Asystent podróży włączony – tempomat jest aktywny\guillemotright strona 149.
XX	Asystent podróży włączony – asystent pasa ruchu jest aktywny\guillemotright strona
149.

XX	Niska temperatura zewnętrzna\guillemotright strona 20.
XX	Włączono światło drogowe lub migacz reflektorów\guillemotright strona 49.

XX	Niezapięty pas bezpieczeństwa na siedzeniu tylnym. \guillemotright Sposób działania

XX	Założony pas bezpieczeństwa na siedzeniu tylnym \guillemotright Sposób działania

XX	Świeci się -- Niezajęte tylne siedzenie\guillemotright strona 41.
XX	Asystent świateł drogowych włączony\guillemotright strona 52. \\ Asystent reflektorów\guillemotright strona 52.
XX	Żadne światła nie są włączone\guillemotright strona 49.
XX	Przejąć kontrolę\guillemotright strona 149.

XX	Zdarzenia serwisowe\guillemotright strona 198.

XX	Silnik został automatycznie wyłączony przez START-STOP \guillemotright Sposób działania
XX	Silnik nie został wyłączony automatycznie przez system START-STOP \guillemotright Sposób działania

XX	Ograniczenie prędkości zostało zakłócone\guillemotright strona 145.

XX	Świeci się -- Włączono ogranicznik prędkości\guillemotright strona 144. \\ Wraz z XX -- Ogranicznik prędkości steruje prędkością jazdy
XX	ACC aktywowany\guillemotright strona 146.

XX	ACC kontroluje prędkość jazdy w zależności od zbliżającego się ruchu okrężnego \guillemotright strona 146.
XX	ACC reguluje prędkość jazdy w zależności od zbliżającego się ruchu FIXME okrężnego \guillemotright strona 146.
XX	ACC kontroluje prędkość jazdy w zależności od trasy\guillemotright strona 146.
XX	ACC reguluje prędkość jazdy zgodnie z dopuszczalną prędkością\guillemotright strona 146.
XX	Usterka tempomatu\guillemotright strona 146.
XX	Włączono tempomat\guillemotright strona 145. \\ Wraz z XX -- Tempomat reguluje prędkość jazdy \guillemotright Sposób działania
XX	Aktywowany jest asystent zjazdu \guillemotright Sposób działania \\ Interweniuje asystent zjazdowy \guillemotright Sposób działania
XX	Uruchomiono Front Assist\guillemotright strona 143.
XX	Bezpieczna odległość przekroczona\guillemotright strona 143.
XX	Zalecana pauza\guillemotright strona 152.
XX	Tryb jazdy Normalny\guillemotright strona 135.
XX	Tryb jazdy Eco\guillemotright strona 135.
XX	Tryb jazdy Comfort\guillemotright strona 135.
XX	Tryb jazdy Indywidualny\guillemotright strona 135.
XX	Tryb jazdy Offroad\guillemotright strona FIXME 135.
XX	Tryb jazdy Sport\guillemotright strona 135.

\section{Dobrze i bezpiecznie}

\subsection{Nowy pojazd lub nowe części}

\textbf{Nowy pojazd - docieranie silnika}

Styl jazdy podczas pierwszych 1500 km decyduje o jakości procesu docierania silnika.

\begin{itemizeTriangle}
	\itemTriangle Podczas pierwszego 1000 km nie przekraczać 3/4 maksymalnej dozwolonej prędkości obrotowej silnika i zrezygnować z jazdy z przyczepą.
	\itemTriangle Podczas dalszych 500 km prędkość obrotowa silnika może być powoli zwiększana.
\end{itemizeTriangle}

\textbf{Nowe klocki hamulcowe}

Nowe klocki hamulcowe nie dają najlepszego możliwego efektu hamowania podczas pierwszych 200~km, muszą się najpierw ułożyć.

\textbf{Nowe opony}

Nowe opony nie mają najlepszej możliwej przyczepności podczas pierwszych 500~km.

\subsection{Zachowaj funkcjonalność czujników i kamer}

Utrzymuj w czystości czujniki i kamery.

\subsection{Komora silnika}

Przed otwarciem pokrywy komory silnika

Niebezpieczeństwo poparzenia! Nie otwieraj pokrywy komory silnika, jeśli z komory silnika wydobywa się para lub płyn chłodzący.
\begin{itemizeTriangle}
	\itemTriangle Wyłączyć silnik i odczekać, aż się schłodzi.
	\itemTriangle Otwórz drzwi kierowcy.
\end{itemizeTriangle}

Podczas pracy w komorze silnika
\begin{itemizeTriangle}
	\itemTriangle Postępuj zgodnie z instrukcjami bezpieczeństwa wymienionymi w rozdziale dotyczącym układu wysokiego napięcia \guillemotright strona 160, Czego należy przestrzegać.
	\itemTriangle Nigdy nie sięgać do wentylatora chłodnicy. Wentylator chłodnicy może się samoczynnie włączyć również wtedy, gdy zapłon jest wyłączony.
	\itemTriangle Nie dotykaj kabli elektrycznych. Unikać zwarć w instalacji elektrycznej – szczególnie przy akumulatorze 12 V.
	\itemTriangle Gdy istnieje konieczność pracy w komorze silnika przy uruchomionym silniku, zwracać uwagę na obracające się części silnika i instalacje elektryczne.
\end{itemizeTriangle}


\subsection{Usiądź bezpiecznie}

Regulowany zagłówek ustawić w taki sposób, aby jego górna krawędź znajdowała się możliwie na tej samej wysokości, co górna część głowy.

% FIXME: Rys.
Fotel kierowcy powinien być tak ustawiony w płaszczyźnie przesuwu w tył / przód, aby po całkowitym wciśnięciu pedałów nogi pozostawały lekko zgięte w kolanach.
\begin{itemizeArrow}
	\itemArrow Kierownicę ustawić tak, aby odstęp pomiędzy nią a mostkiem wynosił co najmniej 25 cm – A .
\end{itemizeArrow}

% FIXME: Czy mamy poduszkę kolan?
W samochodach wyposażonych w poduszkę powietrzną kolan kierowcy przesunąć fotel w przód lub w tył tak, aby odstęp pomiędzy nogami a deską rozdzielczą na wysokości poduszki powietrznej wynosił przynajmniej 6 cm – B .

Oparcie fotela ustawić tak, aby kierownicę można było chwycić w najwyższym jej punkcie lekko zgiętymi rękoma.

\begin{itemizeArrow}
	\itemArrow Fotel pasażera przesunąć jak najdalej w tył. Przedni pasażer musi zachować minimalną odległość 25 cm od tablicy rozdzielczej.
\end{itemizeArrow}

\subsection{Bezpieczne parkowanie samochodu}

Czynności podczas parkowania należy wykonywać wyłącznie w podanej niżej kolejności.

\begin{itemizeArrow}
	\itemArrow Zatrzymać samochód i przytrzymać wciśnięty pedał hamulca.
	\itemArrow Zabezpieczyć pojazd hamulcem postojowym.
	\itemArrow Za pomocą dźwigni automatycznej skrzyni biegów wybierz tryb \gearP.
	\itemArrow Wyłączyć silnik.
	\itemArrow Zwolnić pedał hamulca.
\end{itemizeArrow}

\subsection{Numer alarmowy}


Lampka kontrolna
%FIXME: Rys.

Ten stan systemu wyświetlany jest po włączeniu zapłonu przez zapalenie się lampki kontrolnej A .
\begin{itemizeTriangle}
	\itemTriangle Zielony – system działa
	\itemTriangle Zielony – miga – nawiązywane jest połączenie z centrum alarmowym
	\itemTriangle Czerwony – świeci – występuje awaria systemu, natychmiast zwróć się o pomoc do specjalistycznej firmy
	\itemTriangle Nie świeci – system jest niesprawny z powodu długotrwałej niedostępności sieci komórkowej, jeśli sytuacja ta się utrzyma, zwróć się o pomoc do specjalistycznej firmy.
\end{itemizeTriangle}

Obsługa -- Połączenie ręczne

Przycisk wezwania awaryjnego znajduje się pod pokrywą oznaczoną ikoną XX.
% FIXME: Rys.

\begin{itemizeArrow}
	\itemArrow Aby otworzyć pokrywę, nacisnąć ją.
	\itemArrow Aby zamknąć pokrywę, nacisnąć ją, aż się słyszalnie zatrzaśnie.
\end{itemizeArrow}

% FIXME: Rys.
\begin{itemizeArrow}
	\itemArrow Nacisnąć i przytrzymać wciśnięty przycisk XX pod pokrywą.
\end{itemizeArrow}
> Potwierdź połączenie z centrum alarmowym na ekranie urządzenia multimedialnego.
\begin{itemizeArrow}
	\itemArrow Aby anulować połączenie z centrum alarmowym przed rozpoczęciem rozmowy, naciśnij ponownie przycisk XX lub potwierdź zakończenie konfiguracji połączenia na ekranie Infotainment.
\end{itemizeArrow}

Z ręcznego nawiązania połączenia można korzystać np. w sytuacji zgłaszania wypadku, w którym nie bierze się bezpośredniego udziału.


\section{Klucze, zamki i alarm}

\subsection{Kluczyk}

Skuteczny zasięg kluczyka wynosi około 30 m.
Zakres klucza może się zmniejszyć, np. z powodu zakłóceń sygnału przez inne nadajniki.

Wyjmij brodę klucza

% FIXME: Rys.

\begin{itemizeArrow}
	\itemArrow Naciśnij języczek zabezpieczający A.
\end{itemizeArrow}
Przywieszka B ostrza klucza wysuwa się.
\begin{itemizeArrow}
	\itemArrow Naciśnij ponownie języczek zabezpieczający A.
	\itemArrow Wyjmij ostrze klucza.
\end{itemizeArrow}

\subsection{Centralne ryglowanie}

\subsubsection{Sposób działania}

Sterownik systemu centralnego ryglowania

System odblokowuje i blokuje wszystkie drzwi, klapę gniazda ładowania i pokrywę bagażnika w tym samym czasie.
Wskaźnik odryglowania: podwójne miganie kierunkowskazów.
Wskaźnik zaryglowania: pojedyncze miganie kierunkowskazów.

Lampka kontrolna w drzwiach kierowcy będzie migać przez około 2 sekundy w krótkich odstępach czasu po zamknięciu pojazdu, a następnie zacznie regularnie migać w dłuższych odstępach czasu.

Jeśli żadne drzwi ani pokrywa bagażnika nie zostaną otwarte w ciągu 45 sekund od odblokowania, pojazd automatycznie się zablokuje.


\subsubsection{Funkcja SAFE}
%FIXME: Czy mamy tę funkcję?

W zależności od wyposażenia centralny zamek może zawierać w zestawie funkcję SAFE.
Funkcja SAFE zapobiega otwieraniu drzwi od wewnątrz po zamknięciu pojazdu.
Funkcja SAFE włączy się podczas ryglowania pojazdu z zewnątrz automatycznie.
Na wyświetlaczu zestawu wskaźników po wyłączeniu zapłonu pojawia się komunikat dotyczący funkcji SAFE.

ZAGROŻENIE
Niebezpieczeństwo!
\begin{itemizeTriangle}
	\itemTriangle W zaryglowanym pojeździe z włączoną funkcją SAFE nie mogą przebywać żadne osoby.
\end{itemizeTriangle}

\subsubsection{Wyłączanie funkcji SAFE}

\begin{itemizeTriangle}
	\itemTriangle Przez dwukrotne zaryglowanie w ciągu 2 sekund,
\end{itemizeTriangle}
Lub:
\begin{itemizeTriangle}
	\itemTriangle Wraz z dezaktywacją monitoringu wnętrza \guillemotright strona 27, Ustawienia.
\end{itemizeTriangle}
Lampka kontrolna w drzwiach kierowcy miga przez około 2 sekundy w krótkich odstępach czasu po zablokowaniu pojazdu, a następnie gaśnie i zaczyna migać regularnie w dłuższych odstępach czasu po około 30 sekundach.
Po wyłączeniu funkcji SAFE, drzwi można otworzyć od wewnątrz, pociągając raz za dźwignię otwierającą.
Funkcja SAFE jest ponownie włączana po odblokowaniu i zablokowaniu pojazdu.


\subsubsection{Obsługa}

Środki do obsługi centralnego zamka
W zależności od wyposażenia samochodu:
\begin{itemizeTriangle}
	\itemTriangle Kluczyk \guillemotright strona 23
	\itemTriangle Blokowanie bezkluczykowe (KESSY) \guillemotright strona 25
\end{itemizeTriangle}
% FIXME: Czy mamy KESSY?
\begin{itemizeTriangle}
	\itemTriangle Przycisk centralnego ryglowania
\end{itemizeTriangle}
Zablokuj / odblokuj za pomocą przycisku centralnej blokady
\begin{itemizeArrow}
	\itemArrow Nacisnąć przycisk XX w środkowej części deski rozdzielczej.
\end{itemizeArrow}
Symbol XX w przycisku zapala się podczas ryglowania.

Przycisk rygluje/odryglowuje wszystkie drzwi i pokrywę bagażnika.
Odblokowanie pojazdu ma również miejsce, gdy drzwi są otwarte od wewnątrz.

% FIXME: Systemy bezkluczykowe
W pojazdach z bezkluczykowym systemem odblokowującym dotknięcie klamki odblokowuje przednie drzwi, w pobliżu których znajduje się kluczyk oraz klapę wlewu paliwa.

Po ponownym odblokowaniu pozostałe drzwi i pokrywa bagażnika są odblokowane.

Drzwi po jednej stronie pojazdu
Za pomocą przycisku na kluczyku drzwi po stronie kierowcy i klapa wlewu paliwa są odblokowane.
W pojazdach z bezkluczykowym systemem odblokowującym dotknięcie klamki zwalnia drzwi po stronie pojazdu, w pobliżu którego znajduje się kluczyk, oraz klapkę wlewu paliwa.
Po ponownym odblokowaniu pozostałe drzwi i po-
krywa bagażnika są odblokowane.

\subsubsection{Ustawianie funkcji odblokowania i blokowania}

\begin{itemizeArrow}
	\itemArrow Wybierz w urządzeniu Infotainment w menu XX > na zewnątrz.
	\itemArrow Przesuwając palec bokiem do ekranu, wybierz element menu Centralne ryglowanie .
	\itemArrow Wybierz pozycję menu Centralne ryglowanie .
\end{itemizeArrow}

Wszystkie drzwi

Funkcja umożliwia odryglowanie wszystkich drzwi, pokrywy bagażnika i pokrywy wlewu paliwa.

Pojedyncze drzwi

Przycisk XX na kluczyku odbokowuje drzwi kierowcy i klapę wlewu paliwa.

\subsubsection{Usterka zamka centralnego}

\begin{itemizeTriangle}
	\itemTriangle Lampka ostrzegawcza w drzwiach kierowcy miga najpierw przez 2 sekundy w krótkich odstępach czasu.
	\itemTriangle Następnie świeci ciągle.
	\itemTriangle Po 30 sekundach miga powoli.
	\begin{itemizeArrow}
		\itemArrow Skorzystać z pomocy specjalistycznej stacji obsługi.
	\end{itemizeArrow}
\end{itemizeTriangle}


\subsubsection{Odblokuj mechanicznie i zablokuj drzwi}

Odblokuj i zablokuj drzwi za pomocą cylindra zamka

Zdejmowanie pokrywy
% FIXME: Rys.

\begin{itemizeArrow}
	\itemArrow Pociągnij przedni lewy uchwyt drzwi i przytrzymaj go.
	\itemArrow Włóż klucz we wgłębienie pod spodem pokrywy.
	\itemArrow Przesunąć osłonę w kierunku strzałki.
	\itemArrow Zwolnić klamkę.
\end{itemizeArrow}

Odblokuj i zablokuj
% FIXME: Rys.

\begin{itemizeArrow}
	\itemArrow Usunięte ostrze klucza z przywieszką skierowaną w prawo, włożyć w cylinder zamka i odblokować lub zablokować.
\end{itemizeArrow}

Zamontować pokrywę.
\begin{itemizeArrow}
	\itemArrow Pociągnąć za klamkę i przytrzymać.
	\itemArrow Ponownie założyć osłonę.
	\itemArrow Zwolnić klamkę.
\end{itemizeArrow}

Zablokuj drzwi bez zamka
% FIXME: Rys.

\begin{itemizeArrow}
	\itemArrow Otwórz drzwi.
	\itemArrow W pojazdach z przesłoną na otwór zdejmij tapicerkę.
	\itemArrow Włóż klucz lub śrubokręt płaski do gniazda.
	\itemArrow Wyjmij przez obrót kluczyk lub wkrętak w kierunku z pojazdu (pozycja sprężynowana).
\end{itemizeArrow}
Po zamknięciu drzwi zostaną zaryglowane.

\subsection{Blokowanie bezkluczykowe (KESSY)}
% FIXME: Czy mamy KESSY?

\subsection{System alarmowy}

Wyłączanie wyzwolonego alarmu
\begin{itemizeArrow}
	\itemArrow Odryglować samochód.
\end{itemizeArrow}
Lub:
\begin{itemizeArrow}
	\itemArrow Włączyć zapłon.
\end{itemizeArrow}



\subsubsection{Ustawienia}
Następujące funkcje systemu alarmowego można dezaktywować jednocześnie:
\begin{itemizeTriangle}
	\itemTriangle Układ kontroli wnętrza
	\itemTriangle Ochrona przed odholowaniem.
\end{itemizeTriangle}
Przez dezaktywację wyłączona jest również funkcja SAFE \guillemotright strona 24, Sposób działania.

Dezaktywacja

\begin{itemizeTriangle}
	\itemTriangle Przez dwukrotne zaryglowanie w ciągu 2 sekund,
\end{itemizeTriangle}
Lub:
W urządzeniu Infotainment w menu: XX > na zewnątrz.
\begin{itemizeArrow}
	\itemArrow Przesuwając palec bokiem do ekranu, wybierz element menu Centralne ryglowanie .
	\itemArrow Wybierz pozycję menu Centralne ryglowanie i wyłącz monitorowanie wnętrza, przesuwając suwak w lewo.
\end{itemizeArrow}
Wyłączone funkcje systemu alarmowego są reaktywowane po odblokowaniu i zablokowaniu pojazdu.

Dezaktywację należy przeprowadzić, jeżeli pojazd jest np. holowany lub transportowany.

Po wyłączeniu silnika na ekranie Infotainment wyświetla się menu, w którym można wyłączyć monitorowanie wnętrza.

% FIXME: Zabezpieczenie przed dziećmi w tylnych drzwiach  # uzupełnić

\section{Drzwi, okna i pokrywa bagażnika}

\subsection{Zabezpieczenie przed dziećmi w tylnych drzwiach}

Zabezpieczenie przed dziećmi nie pozwala otworzyć tylnych drzwi od wewnątrz.

%FIXME: Rys.

\begin{itemizeArrow}
	\itemArrow Obróć zabezpieczenie kluczykiem samochodowym lub płaskim śrubokrętem.
\end{itemizeArrow}
Zabezpieczenie A wyłączone
Zabezpieczenie B włączone

\subsection{Okno - z obsługą elektryczną}

Wyłącznik przeciążeniowy

Aby zmniejszyć ryzyko urazów na skutek zaciśnięcia podczas zamykania okien, pojazd ma ograniczenie siły.

W razie napotkania przeszkody zamykanie okna jest przerywane i następuje opuszczenie szyby o kilka centymetrów.

Jeśli przeszkoda ponownie uniemożliwia zamknięcie okna w ciągu kolejnych 10 sekund, zamykanie szyby zostanie ponownie przerwane, a szyba opuszczona o kilka centymetrów.

Jeżeli próba zamknięcia zostanie ponowiona w ciągu 10 sekund od drugiego opuszczenia szyby, mimo że przeszkoda nie została usunięta, proces zamykania szyby zostanie przerwany. W tym czasie nie można automatycznie zamknąć okien, ciągnąc przycisk do końca. Wyłącznik przeciążeniowy wciąż jest aktywny.

Wyłącznik przeciążeniowy wyłączy się dopiero wtedy, gdy ponowna próba zamknięcia okna nastąpi w ciągu kolejnych 10 sekund – podnośnik działa wtedy z pełną siłą! Po upływie 10 sekund wyłącznik przeciążeniowy będzie znów aktywny.

\subsubsection{Przegląd przycisków obsługowych w drzwiach kierowcy}

%FIXME: Rys. (tylko lewy)

A Okno przednie lewe
B Okno przednie prawe
C Okno tylne lewe
D Okno tylne prawe
E dezaktywacja / aktywacja przycisków na drzwiach tylnych

\subsubsection{Obsługa}

Otwieranie

\begin{itemizeArrow}
	\itemArrow Aby otworzyć, nacisnąć lekko odpowiedni przycisk i przytrzymać go tak długo, aż szyba osiągnie żądaną wysokość.
\end{itemizeArrow}
Lub:
\begin{itemizeArrow}
	\itemArrow Nacisnąć przycisk do oporu, szyba automatycznie otworzy się całkowicie. Ponowne naciśnięcie przycisku wyłącza proces otwierania.
\end{itemizeArrow}

Zamykanie

\begin{itemizeArrow}
	\itemArrow Aby zamknąć, pociągnąć lekko górną krawędź odpowiedniego przycisku i przytrzymać go tak długo, aż szyba osiągnie żądaną wysokość.
\end{itemizeArrow}
Lub:
\begin{itemizeArrow}
	\itemArrow Krótko nacisnąć przycisk do oporu, szyba automatycznie zamknie się całkowicie. Ponowne naciśnięcie przycisku zatrzymuje proces zamykania.
\end{itemizeArrow}

Po wyłączeniu zapłonu można otwierać i zamykać szyby jeszcze przez około 10 minut, o ile nie zostaną otwarte żadne drzwi.

Zamykanie wszystkich okien w tym samym czasie

Nacisnąć i przytrzymać wciśnięty przycisk w kluczyku.
LUB
\begin{itemizeArrow}
	\itemArrow Wyłącz zapłon, otwórz drzwi kierowcy i przytrzymaj przycisk okna kierowcy pociągnięty do oporu.
\end{itemizeArrow}

\subsubsection{Ustawienia}

Ustawienia obsługi okna wprowadza się w urządzeniu Infotainment w menu XX > na zewnątrz.
\begin{itemizeArrow}
	\itemArrow Przesuwając palec bokiem do ekranu wybierz ekran z punktem menu do obsługi okna.
	\itemArrow Wybierz punkt menu dla obsługi okna.
\end{itemizeArrow}

\subsubsection{Rozwiązywanie problemów}

Podnośnik szyby nie działa po wielokrotnym otwieraniu i zamykaniu

Mechanizm regulatora szyby może zostać przegrzany.

\begin{itemizeArrow}
	\itemArrow Pozwól mechanizmowi podnoszenia okna ostygnąć.
\end{itemizeArrow}


\subsubsection{Aktywacja po odłączeniu akumulatora 12 V}

Aktywacja automatycznego działania okien
\begin{itemizeArrow}
	\itemArrow Włączyć zapłon.
	\itemArrow Pociągnąć w górę odpowiedni przycisk i zamknąć okno.
	\itemArrow Zwolnić przycisk.
	\itemArrow Ponownie pociągnąć do góry odpowiedni przycisk i przytrzymać przez 1 sekundę.
\end{itemizeArrow}

% FIXME: Rolety przeciwsłoneczne - str. 30

\subsection{Pokrywa bagażnika - z obsługą ręczną}

% FIXME: Rys. - otwieranie

% FIXME: Rys. - zamykanie

\subsubsection{Ustawianie opóźnionej blokady klapy}

Jeżeli klapa bagażnika została odryglowana przy użyciu przycisku XX na kluczyku, to po jej zamknięciu zostanie ponownie automatycznie zaryglowana.

Czas, po którym pokrywa bagażnika zostanie po zamknięciu automatycznie zaryglowana, może zostać ustawiony w specjalistycznej stacji obsługi.

%FIXME: Pokrywa bagażnika - z elektrycznym sterowaniem - str. 31

\subsection{Odryglowanie pokrywy bagażnika}

Jeśli pokrywa bagażnika nie otwiera się, można ją odblokować ręcznie w następujący sposób.

% FIXME: Rys.

\begin{itemizeArrow}
	\itemArrow Włóż śrubokręt do otworu w osłonie.
	\itemArrow Otwórz klapkę przesuwając w kierunku strzałki.
\end{itemizeArrow}


\section{Fotele, kierownica i lusterko}

\subsection{Przedni fotel - z ręczną regulacją}

%FIXME: Rys.; który wariant?

\subsection{Fotele tylne}

\subsubsection{Złożyć do przodu oparcia siedzenia}

Przed złożeniem
\begin{itemizeArrow}
	\itemArrow Wsuń lub zdejmij tylne zagłówki tak daleko, jak to możliwe.
	\itemArrow Dostosuj pozycję przednich foteli, aby nie zostały uszkodzone przez składane oparcia.
	\itemArrow Pociągnij zewnętrzny pas bezpieczeństwa do bocznej listwy.
\end{itemizeArrow}

Składanie od strony kabiny
% FIXME: Rys.
Naciśnij uchwyt zwalniający i złóż oparcie siedzenia.

Składanie od strony bagażnika
% FIXME: Rys.
\begin{itemizeArrow}
	\itemArrow Pociągnij za dźwignię.
\end{itemizeArrow}
Oparcie jest odblokowane i częściowo złożone do przodu.

Rozkładanie
% FIXME: Rys.
Pociągnij zewnętrzny pas bezpieczeństwa do bocznej listwy.
\begin{itemizeArrow}
	\itemArrow Rozłożyć oparcie fotela.
\end{itemizeArrow}
Zaczep musi słyszalnie się zatrzasnąć.
\begin{itemizeArrow}
	\itemArrow Sprawdź zatrzask oparcia siedzenia. Kołek A nie może być widoczny.
\end{itemizeArrow}

\subsection{Zagłówki}

\subsubsection{Regulacja zagłówków}

Zagłówki z przodu
% FIXME: Rys.
\begin{itemizeArrow}
	\itemArrow Aby ustawić wysokość, należy przytrzymać przycisk bezpieczeństwa i przesunąć zagłówek w wymaganym kierunku.
\end{itemizeArrow}

Tylne zagłówki
% FIXME: Rys.
Przesunąć zagłówek w wybrane położenie.

Podczas przesuwania w dół przycisk bezpieczeństwa musi być wciśnięty.

Składane boczne panele tylnego zagłówka
% FIXME: Czy posiadamy?
Złożone panele boczne ograniczają niekontrolowany ruch głowy na boki, np. podczas snu.
% FIXME: Rys.
\begin{itemizeArrow}
	\itemArrow Ustaw zagłówek do pierwszego wyciągniętego położenia.
	\itemArrow Panele boczne zagłówka rozłożyć do przodu.
\end{itemizeArrow}

\subsubsection{Wyjmowanie i wkładanie zagłówków tylnych}

Wyjmowanie
% FIXME: Rys.
\begin{itemizeArrow}
	\itemArrow Częściowo złożyć odpowiednie oparcie siedzenia.
	\itemArrow Przesunąć podpórkę do oporu do góry.
	\itemArrow Przycisk zabezpieczający A i B naciśnij jednocześnie i usuń podpórkę.
\end{itemizeArrow}

\subsection{Przedni podłokietnik}

% FIXME: Rys.

Regulacja wysokości
\begin{itemizeArrow}
	\itemArrow Podnieś oparcie w jednej z pozycji blokujących.
\end{itemizeArrow}

Opuszczanie

\begin{itemizeArrow}
	\itemArrow Podnieś oparcie poza najwyższą pozycję blokującą i złóż je ponownie.
\end{itemizeArrow}

Ustawianie wzdłuż
\begin{itemizeArrow}
	\itemArrow Przesuń oparcie do żądanej pozycji.
\end{itemizeArrow}

\subsection{Tunel w oparciu tylnej kanapy}

Otwieranie z przedziału pasażerskiego
% FIXME: Rys.
\begin{itemizeArrow}
	\itemArrow Naciśnij pokrywę w górnej części i pociągnij za uchwyt.
	\itemArrow Otwórz pokrywę.
\end{itemizeArrow}

Otwieranie z bagażnika
% FIXME: Rys.
\begin{itemizeArrow}
	\itemArrow Naciśnij przycisk zabezpieczający.
	\itemArrow Otwórz pokrywę.
\end{itemizeArrow}

Zamykanie
\begin{itemizeArrow}
	\itemArrow Odchylić pokrywę od przedziału pasażerskiego, aż usłyszysz kliknięcie.
\end{itemizeArrow}
Po zamknięciu czerwone oznaczenie z tyłu pokrywy nad przyciskiem bezpieczeństwa nie może być widoczne.

\subsection{Kierownica}

\subsubsection{Przyciski / pokrętła na kierownicy wielofunkcyjnej}

% FIXME: Rys.

XX włączanie / wyłączanie obsługi głosowej
XX Ogrzewanie kierownicy \guillemotright strona 62

A Obrót – ustawianie głośności
Naciśnięcie – włączanie / wyłączanie dźwięku

XX Przejście do kolejnego odtwarzanego utworu / stacji radiowej
XX Przejście do poprzedniego odtwarzanego utworu / stacji radiowej
XX Wyświetlanie menu układów wspomagających
XX Włączanie / wyłączanie systemu pomocy Travel Assist \guillemotright strona 149
B Obsługa cyfrowego zestawu wskaźników \guillemotright strona 63
XX Obsługa cyfrowego zestawu wskaźników \guillemotright strona 63
XX Obsługa cyfrowego zestawu wskaźników \guillemotright strona 63


\subsubsection{Regulacja położenia kierownicy}

% FIXME: Rys.
Przesuń dźwignię bezpieczeństwa w dół.

% FIXME: Rys.
Ustaw kierownicę w wybranej pozycji.

% FIXME: Rys.
Naciśnij dźwignię zabezpieczającą do oporu.

\subsubsection{Rozwiązywanie problemów}

Usterka układu kierowniczego

XX Świeci się - całkowita awaria układu wspomagania kierownicy, brak wspomagania kierowania

Wyłącz zapłon, uruchom silnik i przejedź kilka metrów.

Jeśli lampka kontrolna XX nie gaśnie, nie kontynuuj jazdy.
Skorzystaj z pomocy specjalistycznej stacji obsługi.

XX świeci się - częściowa awaria wspomagania kierownicy, możliwe zmniejszenie wspomagania kierownicy
\begin{itemizeArrow}
	\itemArrow Wyłącz zapłon, uruchom silnik i przejedź kilka metrów.
	\itemArrow Jeśli lampka kontrolna XX świeci się nie gaśnie, dalsza jazda jest możliwa tylko przy zachowaniu odpowiedniej ostrożności.
\end{itemizeArrow}
Skorzystać z pomocy specjalistycznej stacji obsługi.

Blokada kolumny kierownicy ma usterkę

XX miga
Komunikat dotyczący usterki w zamku kolumny kierownicy
\begin{itemizeArrow}
	\itemArrow Odstaw pojazd.
	\itemArrow Skorzystaj z pomocy specjalistycznej stacji obsługi.
\end{itemizeArrow}

Po wyłączeniu zapłonu nie jest już możliwe włączenie zapłonu, zablokowanie układu kierowniczego i włączenie obciążeń elektrycznych.

XX miga

Komunikat dotyczący usterki blokady kolumny kierownicy

\begin{itemizeArrow}
	\itemArrow Dalsza jazda jest możliwa tylko przy zachowaniu odpowiedniej ostrożności.
\end{itemizeArrow}
Skorzystaj z pomocy specjalistycznej stacji obsługi.

Blokada kolumny kierownicy nie odblokowana

XX miga

Komunikat dotyczący koniecznego ruchu koła kierownicy
\begin{itemizeArrow}
	\itemArrow Poruszaj kierownicą lekko w lewo i w prawo.
	\itemArrow Jeśli układ kierowniczy nie jest odblokowany, zatrzymaj pojazd i skorzystaj z pomocy specjalistycznej stacji obsługi.
\end{itemizeArrow}

\subsection{Lusterko boczne}

Obsługa

W zależności od wyposażenia lusterka można składać ręcznie lub elektrycznie.

Pozycje pokrętła
% FIXME: Rys.

L Ustawianie powierzchni lusterka z lewej strony
O Wyłącz obsługę
R Ustawianie powierzchni lusterka z prawej strony
XX Elektryczne składanie lusterek (aby rozłożyć, należy przestawić pokrętło w inne położenie)
XX Podgrzewanie lusterek przy włączonym silniku

Ustawianie powierzchni lusterka
\begin{itemizeArrow}
	\itemArrow Wybrać pozycję XX lub XX.
	\itemArrow Przestawić pokrętło w kierunku strzałek.
\end{itemizeArrow}
Automatyczne składanie elektrycznie składanych lusterek

Jeśli ta funkcja jest aktywna, lusterka są składane, gdy pojazd jest zablokowany, a składane do tyłu, gdy jest odblokowany.

Lusterko składane ręcznie
% FIXME: Czy mamy tę opcję?
\begin{itemizeArrow}
	\itemArrow Złóż lusterko do bocznego okna, naciskając ręką.
\end{itemizeArrow}

Synchroniczne ustawianie powierzchni lusterek

Jeśli ta funkcja jest aktywna, powierzchnia lusterka pasażera jest ustawiana również wtedy, gdy ustawiana jest powierzchnia lusterka kierowcy.

Ustawienia
Włączanie lub wyłączanie automatycznego składania lusterek zewnętrznych oraz synchroniczna regulacja powierzchni lusterka odbywa się w następującym menu:
\begin{itemizeArrow}
	\itemArrow Wybierz w urządzeniu Infotainment w menu XX > na zewnątrz .
	\itemArrow Przesuwając palec bokiem do ekranu, wybierz element menu Lusterko .
	\itemArrow Wybierz pozycję menu Lusterko i wykonaj aktywację lub dezaktywację.
\end{itemizeArrow}

Rozwiązywanie problemów

Usterka działania lusterka elektrycznego

\begin{itemizeArrow}
	\itemArrow Ustaw powierzchnię lustra za pomocą lekkiego nacisku palca.
\end{itemizeArrow}

\section{Oświetlenie, wycieraczki i spryskiwacze szyb}

\subsection{Oświetlenie zewnętrzne}

Sposób działania

Światła działają tylko przy włączonym zapłonie, o ile nie podano inaczej.
światła do jazdy dziennej
Światła do jazdy dziennej zapewniają oświetlenie przedniej części samochodu.
W pojazdach dostosowanych do niektórych rynków zapewnia to również oświetlenie tylnej części pojazdu.

Warunki działania
\begin{itemizeTick}
	\itemTick Przełącznik świateł jest w trybie XX.
\end{itemizeTick}

Automatycznie włączaj / wyłączaj światła

Światło jest automatycznie włączane lub wyłączane zgodnie z warunkami oświetlenia i działaniem pojazdu (postój / jazda).

Niektóre z poniższych funkcji oświetlenia włączają się / wyłączają automatycznie:
\begin{itemizeTriangle}
	\itemTriangle Światła mijania i światła postojowe
	\itemTriangle światła do jazdy dziennej
\end{itemizeTriangle}

Warunki działania
\begin{itemizeTick}
	\itemTick Przełącznik świateł jest w trybie XX.
\end{itemizeTick}

Automatyczne włączenie świateł mijania i postojowych jest sygnalizowane przez podświetlenie symbolu XX we włączniku światła.

Włączanie automatyczne światła mijania w przypadku deszczu

Warunki działania
\begin{itemizeTick}
	\itemTick Przełącznik świateł jest w trybie XX.
	\itemTick Funkcja jest aktywna.
	\itemTick Wycieraczki przednie działają przez ponad 30 s.
\end{itemizeTick}

Automatyczne włączenie świateł drogowych jest sygnalizowane przez podświetlenie symbolu XX we włączniku światła.

Reflektor przedni Full LED

Kiedy włącznik świateł jest w pozycji XX reflektory zapewniają najlepszą możliwą wiązkę światła przed pojazdem.
Funkcja Dynamiczne światło doświetlające zakręty zapewnia najlepsze możliwe oświetlenie obszaru zakrętu.

Funkcja CORNER

Funkcja CORNER przeznaczona jest do skręcania lub manewrowania (np. podczas parkowania).

Funkcja oświetla bliskie otoczenie przodu pojazdu w kierunku jazdy.

Warunki działania
\begin{itemizeTick}
	\itemTick Kierunkowskazy są włączone lub przednie koła są mocno skręcone.
	\itemTick Światła mijania są włączone.
	\itemTick W otoczeniu pojazdu widoczność jest pogorszona.
	\itemTick Reflektory przeciwmgłowe nie są włączone.
	\itemTick Prędkość jazdy jest mniejsza niż 40 km/h.
\end{itemizeTick}

Gdy nie są włączone żadne światła

Podświetlenie symbolu XX w przełączniku świateł, jeśli to konieczne również w zestawie wskaźników, wskazuje, że konieczne jest włączenie świateł.
% FIXME: ikona

Obsługa

Tryby świateł

% FIXME: Rys.

Przełącznikiem XX można wybrać tryb świateł.

\begin{itemizeArrow}
	\itemArrow Przełącznik XX naciśnij kilkakrotnie i wybierz żądany tryb.
\end{itemizeArrow}

Informacja o wybranym trybie jest krótko wyświetlana na wyświetlaczu zestawu wskaźników.

Menu z dostępnymi trybami oświetlenia różni się w zależności od danych warunków oświetlenia i działania pojazdu (postój / jazda).

XX - Tryb automatyczny
XX - Światła mijania
XX - Światła postojowe
XX - Światło wyłączone

Po włączeniu zapłonu automatycznie ustawiany jest tryb XX.

Światła przeciwmgielne

Przełącznik reflektorów przeciwmgłowych i tylnego światła przeciwmgłowego
A Światła przeciwmgielne - w zestawie wskaźników świeci się lampka kontrolna XX
B Tylne światło przeciwmgielne - na tablicy rozdzielczej zaświeci się lampka kontrolna XX

Jednostronne światło postojowe XX
%FIXME: Rys.

Jednostronne światło postojowe pozwala na oświetlenie jednej strony zaparkowanego pojazdu za pomocą odpowiedniego światła postojowego.

A Włącz światło postojowe po prawej stronie
B Włącz światło postojowe po lewej stronie
\begin{itemizeArrow}
	\itemArrow Wyłączyć zapłon.
	\itemArrow Przesuń dźwignię do odpowiedniej pozycji.
	\itemArrow Zaryglować pojazd.
\end{itemizeArrow}
Gdy światła postojowe są włączone, po wyłączeniu zapłonu i otworzeniu drzwi kierowcy rozlega się akustyczny sygnał ostrzegawczy. Po kilku sekundach lub po zamknięciu drzwi kierowcy dźwiękowy sygnał ostrzegawczy wyłącza się.

Dwustronne światła postojowe XX

Dwustronne światła postojowe pozwalają na oświetlenie zaparkowanego pojazdu za pomocą świateł postojowych.
\begin{itemizeArrow}
	\itemArrow Włączyć zapłon.
	\itemArrow Wybierz tryb świateł XX.
	\itemArrow Wyłączyć zapłon.
	\itemArrow Zaryglować pojazd.
\end{itemizeArrow}

Gdy światła postojowe są włączone, po wyłączeniu zapłonu i otworzeniu drzwi kierowcy rozlega się akustyczny sygnał ostrzegawczy. Po kilku sekundach
lub po zamknięciu drzwi kierowcy dźwiękowy sygnał ostrzegawczy wyłącza się.
Światła mogą zostać automatycznie wyłączone w przypadku zbyt niskiego poziomu naładowania akumulatora 12 V.

% FIXME: Ustawienia (str. 50-51) - Edge


Rozwiązywanie problemów

Brak kierunkowskazów

XX miga szybciej - usterka prawego kierunkowskazu
\begin{itemizeArrow}
	\itemArrow Sprawdź kierunkowskaz po prawej stronie.
\end{itemizeArrow}

XX miga szybciej - usterka lewego kierunkowskazu
\begin{itemizeArrow}
	\itemArrow Sprawdź lewy kierunkowskaz.
\end{itemizeArrow}

% FIXME: Oświetlenie zewnętrzne COMING HOME, LEAVING HOME (str. 51)

% FIXME: Asystent świateł drogowych Light Assist  (str. 51)

\subsection{Asystent świateł drogowych Light Assist}

Light Assist włącza / wyłącza światła drogowe automatycznie.

Warunki działania
\begin{itemizeTick}
	\itemTick System jest aktywny.
	\itemTick Przełącznik świateł jest w trybie XX.
	\itemTick Prędkość jazdy przekracza 30 km/h.
	\itemTick Światła mijania są włączone.
\end{itemizeTick}

Włączanie

% FIXME: Rys.

Naciśnij dźwignię w kierunku strzałki.
W zestawie wskaźników zapala się kontrolka XX.

Wyłączanie
\begin{itemizeArrow}
	\itemArrow Ręcznie włączaj i wyłączaj światła drogowe.
\end{itemizeArrow}


Aktywacja / dezaktywacja

> Wybierz w urządzeniu Infotainment w menu XX > na zewnątrz .
> Przesuwając palec bokiem do ekranu, wybierz element menu Światła mijania .
\begin{itemizeArrow}
	\itemArrow Wybierz pozycję menu Światła mijania i włącz/wyłącz funkcję Light Assist .
\end{itemizeArrow}

\subsection{Asystent reflektorów Dynamic Light Assist}

Asystent reflektorów włącza / wyłącza światła drogowe automatycznie.

Po włączeniu świateł drogowych reguluje wiązkę, aby kierowcy pojazdów nadjeżdżających i poprzedzających nie byli oślepieni.

Warunki działania
\begin{itemizeTick}
	\itemTick System jest aktywny.
	\itemTick Przełącznik świateł jest w trybie XX.
	\itemTick Prędkość jazdy jest wyższa niż 30 km/h (w niektórych krajach wyższych niż 60 km/h).
	\itemTick Światła mijania są włączone.
\end{itemizeTick}

Obsługa

Włączanie

% FIXME: Rys.

\begin{itemizeArrow}
	\itemArrow Naciśnij dźwignię w kierunku strzałki.
\end{itemizeArrow}

W zestawie wskaźników zapala się XX.

Wyłączanie

\begin{itemizeArrow}
	\itemArrow Ręcznie włączaj i wyłączaj światła drogowe.
\end{itemizeArrow}

Ustawienia -- Aktywacja / dezaktywacja

\begin{itemizeArrow}
	\itemArrow Wybierz w urządzeniu Infotainment w menu XX
	\itemArrow Przesuwając palec bokiem do ekranu, wybierz element menu Światła mijania .
	\itemArrow Wybierz pozycję menu Światła mijania i włącz/wyłącz funkcję Dynamic Light Assist .
\end{itemizeArrow}


\subsection{Oświetlenie wnętrza}

Obsługa oświetlenia

XX Włącz / wyłącz przednie i tylne światła
XX Dezaktywacja automatycznej aktywacji
(XX świeci na żółto po dezaktywacji)

Poszczególne światła można włączać / wyłączać, dotykając odpowiedniego światła.

Poziom jasności oświetlenia można regulować, trzymając palec na świetle. Ustawiony poziom jasności nie jest zapisywany po wyłączeniu oświetlenia.

Włącz oświetlenie lusterka do makijażu w osłonach przeciwsłonecznych na przedniej szybie
\begin{itemizeArrow}
	\itemArrow Nasuń pokrywę lusterka do makijażu.
\end{itemizeArrow}


Ustawienia

Podświetlenie włączników i przyrządów

Regulacja jasności przełącznika i oświetlenia instrumentów odbywa się w Infotainment na jeden z poniższych sposobów:
\begin{itemizeArrow}
	\itemArrow Naciśnij pasek u góry ekranu i przesuń palcem w dół.
\end{itemizeArrow}
Otworzy się okno z suwakiem do ustawiania poziomu jasności.
\begin{itemizeArrow}
	\itemArrow Dostosuj poziom jasności za pomocą suwaka.
\end{itemizeArrow}

Lub:

Menu XX > wewnątrz .

Przesuwając palec bokiem do ekranu wybierz element menu Oświetlenie wnętrza.

\begin{itemizeArrow}
	\itemArrow Wybierz element menu Oświetlenia wnętrza i dostosuj poziom jasności za pomocą suwaka.
\end{itemizeArrow}

\subsection{Wycieraczki i spryskiwacze}

Wytrzyj szybę i umyj

% FIXME: Rys.

XX Szybkie wycieranie
XX Powolne wycieranie
XX W zależności od wyposażenia samochodu:
% FIXME: czujnik deszczu?
\begin{itemizeTriangle}
	\itemTriangle Praca interwałowa
	\itemTriangle Automatyczne wycieranie kontrolowane przez czujnik deszczu
\end{itemizeTriangle}

XX Wyłączanie
XX Wycieranie końcówek (pozycja sprężynowa)
XX Ustawianie prędkości wycierania dla pozycji XX
XX Mycie i wycieranie (pozycja sprężynowa)

W zależności od wyposażenia samochodu dysze spryskiwaczy szyby przedniej mogą być automatycznie podgrzewane.

Oczyszczanie reflektorów

Włączone światła są czyszczone po pierwszym i po dziesiątym myciu przedniej szyby.
Ustawienie odstępu opryskiwania może przeprowadzić wyspecjalizowana firma.

Wycieranie i mycie tylnej szyby

% FIXME: Rys.

XX Pozycja sprężynująca:
\begin{itemizeTriangle}
	\itemTriangle Mycie i wycieranie szyby
	\itemTriangle Czyszczenie kamery cofania
\end{itemizeTriangle}
XX Mycie
XX Wyłączanie

Automatyczne wycieranie szyby tylnej

Jeżeli wycieranie szyby przedniej odbywa się w sposób ciągły, następuje automatyczna regularna praca przerywana wycieraczki szyby tylnej.

Po włączeniu wycieraczek szyba tylna jest automatycznie czyszczona po włączeniu biegu wstecznego.

Ustawienia

Włącz / wyłącz automatyczne czyszczenie

Funkcje automatycznego czyszczenia wycieraczki tylnej szyby i automatycznego czyszczenia w deszczu można aktywować i dezaktywować w systemie Infotainment.
\begin{itemizeArrow}
	\itemArrow Menu XX > na zewnątrz .
	\itemArrow Przesuwając palec bokiem do ekranu, wybierz element menu Lusterka i wycieraczki .
\end{itemizeArrow}

\begin{itemizeArrow}
	\itemArrow Wybierz pozycję menu Lusterka i wycieraczki .
	\itemArrow Włącz lub wyłącz funkcje.
\end{itemizeArrow}

Uzupełnić płyn do spryskiwacza

Zbiornik płynu do spryskiwaczy znajduje się w komorze silnika \guillemotright strona 12.
Pojemność zbiornika wynosi 3 litry, a w samochodach z układem czyszczenia reflektorów 4,7 litrów.

% FIXME: Rys.

\begin{itemizeArrow}
	\itemArrow Otwórz pokrywę komory silnika \guillemotright strona 167.
	\itemArrow Ostrożnie otwórz górną część pokrywy.
	\itemArrow Uzupełnij płyn do spryskiwaczy szyb.
	\itemArrow Zamknij pokrywę pojemnika.
\end{itemizeArrow}

Rozwiązywanie problemów

Zbyt niski poziom wody w zbiorniku spryskiwacza szyby

XX zapala się razem z XX


Rozkładanie ramion wycieraczek przedniej szyby i wymiana piór wycieraczek szyby przedniej

Aby odchylić wycieraczki od szyby należy najpierw ustawić ramiona wycieraczek w pozycji odchylonej.


Ustaw ramiona wycieraczek w pozycji odchylonej - za pomocą dźwigni obsługowej
\begin{itemizeArrow}
	\itemArrow Włączyć i wyłączyć zapłon.
	\itemArrow W ciągu 10 sekund wcisnąć dźwignię obsługi i przytrzymać ją wciśniętą w dół przez ok. 2 sekundy.
\end{itemizeArrow}
Ustaw ramiona wycieraczek w pozycji odchylonej
- za pomocą urządzenia Infotainment
na zewnątrz .
\begin{itemizeArrow}
	\itemArrow Menu XX > na zewnątrz 
	\itemArrow Przesuwając palec bokiem do ekranu, wybierz element menu Lusterka i wycieraczki .
	\itemArrow Wybierz pozycję menu Lusterka i wycieraczki a następnie wybierz pozycję serwisową ramion wycieraczek.
\end{itemizeArrow}

Po wyłączeniu zapłonu na ekranie Infotainment wyświetla się menu, w którym można ustawić pozycję złożóną ramion wycieraczki.

Wymiana piór wycieraczek

Wymieniaj wycieraczki szyby przedniej raz lub dwa razy w roku.
% FIXME: Rys.
\begin{itemizeArrow}
	\itemArrow Odłóż ramiona wycieraczki od szyby.
	\itemArrow Naciśnij zabezpieczenie i wyjmij pióro wycieraczki.
	\itemArrow Włóż nowe pióro wycieraczki, aż usłyszysz kliknięcie.
	\itemArrow Ramię wycieraczki położyć na szybie.
	\itemArrow Włączyć zapłon i wcisnąć dźwignię obsługową w kierunku strzałki.
\end{itemizeArrow}

\section{Ogrzewanie i klimatyzacja}

\subsection{Klimatyzacja automatyczna Climatronic}

\subsubsection{Czego należy przestrzegać}

\begin{itemizeTriangle}
	\itemTriangle Zalecamy utrzymywanie temperatury wewnętrznej na poziomie maksymalnie o 5°C niższym od temperatury zewnętrznej.
	\itemTriangle Zalecamy wyłączenie układu chłodzenia około 10 minut przed końcem podróży, aby zapobiec powstawaniu przykrych zapachów.
	\itemTriangle Raz w roku zaleca się dezynfekcję klimatyzacji.
\end{itemizeTriangle}

\subsubsection{Sposób działania}

Air Care

Funkcja Air Care zmniejsza przenikanie zanieczyszczeń do wnętrza pojazdu. Jednocześnie powietrze jest cyrkulowane i czyszczone.
Aby zapewnić prawidłowe działanie, drzwi i okna muszą być zamknięte.
Funkcję można włączyć/wyłączyć po naciśnięciu obszaru funkcjonalnego XX Air Care w menu obsługi systemu klimatyzacji.

Włączenie funkcji jest sygnalizowane przez zielone przebarwienie obszaru funkcjonalnego Air Care .

Wyłączenie funkcji zostanie przedstawione za pomocą białej barwy obszaru funkcjonalnego Air Care .

\subsubsection{Warunki pracy układu chłodzenia}
\begin{itemizeTick}
	\itemTick Temperatura zewnętrzna powyżej 2° C
	\itemTick Silnik pracuje
	\itemTick Wentylator włączony
\end{itemizeTick}


Obsługa
Menu do obsługi klimatyzacji

\begin{itemizeArrow}
	\itemArrow XX pod Infotainment.
\end{itemizeArrow}
Lub:
\begin{itemizeArrow}
	\itemArrow Dotknij XX na dolnym pasku stanu ekranu Infotainment.
\end{itemizeArrow}

Wyświetla się następujące menu:
XX Wyświetl inteligentną klimatyzację ze wstępnie ustawionymi trybami dystrybucji powietrza
XX Pokaż klasyczne ustawienie klimatyzatora
XX Air Care
XX Dalsze ustawienia klimatyzacji

\subsubsection{Ustawienia}

Ustaw inteligentną klimatyzację

\begin{itemizeArrow}
	\itemArrow Nacisnąć XX pod Infotainment; Stuknąć XX na ekranie Infotainment.
\end{itemizeArrow}
Lub:
\begin{itemizeArrow}
	\itemArrow Dotknij XX na dolnym pasku stanu ekranu Infotainment.
\end{itemizeArrow}

W zależności od wyposażenia wyświetlane jest menu z niektórymi z poniższych funkcji:
XX Ręczna klimatyzacja włączona / wyłączona
XX Pokaż menu ustawień z przodu
XX Pokaż menu ustawień z tyłu
XX Opcja preselekcji dla wentylacji/odszraniania szyby przedniej
XX Opcja preselekcji nawiewu ciepłego powietrza na nogi
XX Opcja preselekcji nawiewu ciepłego powietrza na dłonie i włączania ogrzewania kierownicy
XX Opcja preselekcji nawiewu zimnego powietrza na nogi
XX Opcja preselekcji nawiewu powietrza z zewnątrz do wnętrza
XX Włącz / wyłącz utrzymanie temperatury wnętrza zgodnie z ustawieniem temperatury po stronie kierowcy
XX Włącz / wyłącz ogrzewanie / wentylację fotela \guillemotright strona 61, Obsługa
XX Obniż temperaturę
XX wyższa temperatura

Wybrana opcja wstępnej selekcji zostanie włączona na czas określony.

Aby rozpocząć opcję wstępnego wyboru, należy spełnić określone warunki ze względu na stan pojazdu. Na te warunki nie można wpłynąć ani ich rozpoznać.

Podczas wyłączania opcji preselekcji XX ogrzewanie kierownicy jest również wyłączane. Dotyczy to również sytuacji, gdy ogrzewanie kierownicy zostało włączone ręcznie przed uruchomieniem tej opcji preselekcji.

Ustaw klasyczną klimatyzację

\begin{itemizeArrow}
	\itemArrow Naciśnij XX pod Infotainment; Stuknąć XX na ekranie Infotainment.
\end{itemizeArrow}
Lub:
\begin{itemizeArrow}
	\itemArrow Dotknij XX na dolnym pasku stanu ekranu Infotainment.
\end{itemizeArrow}

Wyświetla się następujące menu:
% FIXME: Rys.

A Pokaż menu ustawień z przodu
B Ustawianie kierunku nawiewu powietrza
C Pokaż menu ustawień z tyłu
D Włączanie trybu automatycznego i ustawianie obsługi ręcznej
E ustawianie intensywności nadmuchu
F Wyświetlanie menu, przegląd aktualnie wybranych funkcji
XX Włącz / wyłącz Climatronic
XX Włącz / wyłącz ogrzewanie szyby przedniej \guillemotright strona 60, Obsługa
XX Włącz / wyłącz utrzymanie temperatury wnętrza zgodnie z ustawieniem temperatury po stronie kierowcy
XX Włącz / wyłącz zamknięty obieg powietrza
XX Włącz / wyłącz chłodzenie
XX Włącz / wyłącz ogrzewanie / wentylację fotela \guillemotright strona 61, Obsługa
XX Zablokuj wzrost mocy grzewczej ogrzewania tylnych siedzeń i ustawianie temperatury z tyłu \guillemotright strona 61, Ustawienia \\ Funkcja jest wyświetlana po dotknięciu powierzchni funkcji C .
XX Obniż temperaturę
XX Zwiększ temperaturę

Temperaturę można również ustawić dwoma palcami na pasku postępu pod obszarem ustawiania temperatury.

Temperaturę tylną można również ustawić na wyświetlaczu w tylnej środkowej konsoli.

Przy ustawieniu temperatury poza zakresem liczbowym Climatronic wyświetla jedną z następujących ikon:
LO Maksymalna wydajność chłodzenia
HI Maksymalna moc grzewcza

Inne ustawienia klimatyzacji Climatronic

\begin{itemizeArrow}
	\itemArrow Stuknąć obszar funkcjonalny XX w menu obsługi klimatyzatora.
\end{itemizeArrow}

Wyświetla się następujące menu:
\begin{itemizeTriangle}
	\itemTriangle Automatyczny zamknięty obieg powietrza - włączanie/wyłączanie automatycznego obiegu powietrza
\end{itemizeTriangle}
Dalej w zależności od wyposażenia pojazdu:
\begin{itemizeTriangle}
	\itemTriangle Automatyczna nagrzewnica – włączanie/wyłączanie szybkiego nagrzania wnętrza
	\itemTriangle Automatyczne ogrzewanie przedniej szyby - włączanie/wyłączanie automatycznego ogrzewania przedniej szyby \guillemotright strona 61, Ustawienia
	\itemTriangle Włączanie / wyłączanie automatycznego uruchamiania ogrzewania i wentylacji fotela oraz ogrzewania kierownicy po uruchomieniu silnika (w zależności od temperatury wnętrza)
\end{itemizeTriangle}

Przyjazne użytkowanie Climatronic

Jeśli jedno z miejsc nie jest zajęte, Climatronic dostosuje odpowiednio temperaturę powietrza, aby zmniejszyć zużycie energii.

Na ekranie Infotainment pojawi się Eco.

Podczas pracy klimatyzatora może nastąpić automatyczny wzrost prędkości biegu jałowego silnika.

\subsubsection{Rozwiązywanie problemów}

Woda pod pojazdem

Po włączeniu układu chłodzenia woda może ścieknąć z systemu klimatyzacji. To nie jest przeciek.

Zaparowanie szyb

\begin{itemizeArrow}
	\itemArrow Włącz tryb automatyczny.
\end{itemizeArrow}
Lub:
\begin{itemizeArrow}
	\itemArrow Zwiększyć prędkość dmuchawy, włączyć układ chłodzenia i umieścić dystrybucję powietrza na przedniej szybie.
\end{itemizeArrow}

Automatyczne wyłączanie układu chłodzenia

Jeśli temperatura płynu chłodzącego jest zbyt wysoka, układ chłodzenia może się automatycznie wyłączyć. Zapewnia to wystarczające chłodzenie silnika.

\subsection{Ogrzewanie postojowe i wentylacja postojowa}

Ogrzewanie postojowe ogrzewa wnętrze samochodu oraz silnik.

Dodatkowa wentylacja umożliwia doprowadzenie świeżego powietrza do wnętrza pojazdu po wyłączeniu silnika. W rezultacie temperatura wnętrza jest
obniżana, np. w pojeździe zaparkowanym w słońcu.

Sposób działania

Wnętrze jest ogrzewane lub wentylowane zgodnie z temperaturą ustawioną w menu e-Managera \guillemotright strona 65, Ustawienia.

Do ogrzewania system wykorzystuje paliwo ze zbiornika paliwa.


Obsługa

Włącz / wyłącz za pomocą pilota zdalnego sterowania
\begin{itemizeArrow}
	\itemArrow Przytrzymaj odpowiedni przycisk.
\end{itemizeArrow}
% FIXME: Rys.
A Lampka kontrolna
B Antena
XX Wyłączanie
XX Włączanie

INFORMACJA
\begin{itemizeTriangle}
	\itemTriangle Chroń pilota przed wilgocią, silnymi wstrząsami i bezpośrednim światłem słonecznym.
\end{itemizeTriangle}

Wyświetlacz kontrolki pilota zdalnego sterowania
\begin{itemizeTriangle}
	\itemTriangle Świeci się na zielono przez 2 s – włącz.
	\itemTriangle Świeci na czerwono przez 2 s – wyłącz.
\end{itemizeTriangle}

Wyświetlanie lampki kontrolnej na tablicy rozdzielczej.
Świeci się, gdy włączony jest ogrzewanie postojowe XX.

Ustaw automatyczne włączanie
\begin{itemizeArrow}
	\itemArrow W Infotainment stuknij XX.
	\itemArrow Stuknij XX.
	\itemArrow Wybierz wstępnie ustawiony czas odjazdu i dotknij XX.
	\itemArrow Ustaw godzinę wyjazdu, dzień (dni) tygodnia i włącz chłodzenie / ogrzewanie wnętrza.
	\itemArrow Aktywuj ustawiony czas.
	\itemArrow Ustaw żądaną temperaturę i włącz ogrzewanie postojowe.
\end{itemizeArrow}

\subsubsection{Rozwiązywanie problemów}

Lampka kontrolna w pilocie zdalnego sterowania
\begin{itemizeTriangle}
	\itemTriangle Miga na zielono w wolnej sekwencji - sygnał włączenia nie został odebrany
	\itemTriangle Miga na czerwono w wolnej sekwencji - sygnał wyłączenia nie został odebrany
	\itemTriangle Szybko i nieregularnie miga na zielono - ogrzewanie postojowe jest zablokowane, np. ponieważ zbiornik paliwa jest prawie pusty lub wystąpił błąd
	\begin{itemizeArrow}
		\itemArrow Sprawdź ilość paliwa.
		\itemArrow Gdyby system nadal nie był dostępny, skorzystać z pomocy specjalistycznej stacji obsługi.
	\end{itemizeArrow}
	\itemTriangle Najpierw pomarańczowy, potem zielony / czerwony - bateria jest rozładowana, odebrano sygnał włączenia / wyłączenia
	\itemTriangle Najpierw pomarańczowy, następnie migający zielony / czerwony - bateria jest rozładowana, sygnał włączenia / wyłączenia nie został odebrany.
	\itemTriangle Miga na pomarańczowo - akumulator jest prawie rozładowany, sygnał włączenia / wyłączenia nie został odebrany
	\itemTriangle Nie świeci - bateria jest rozładowana, sygnał włączenia / wyłączenia nie został odebrany
	\begin{itemizeArrow}
		\itemArrow Wymienić baterię .
	\end{itemizeArrow}
\end{itemizeTriangle}

\subsection{Ogrzewanie szyb}

Ogrzewanie służy do odmrażania lub odparowywania szyby.

Obsługa

Ogrzewanie szyby tylnej

\begin{itemizeArrow}
	\itemArrow Nacisnąć przycisk XX pod Infotainment.
\end{itemizeArrow}

Ogrzewanie szyby przedniej

\begin{itemizeArrow}
	\itemArrow Nacisnąć przycisk XX pod Infotainment.
	\itemArrow Dotknąć przycisk funkcyjny XX / XX > Przód ; dotknij > XX na ekranie systemu Infotainment.
\end{itemizeArrow}

Ogrzewanie szyby wyłącza się automatycznie po kilku minutach.

Jeżeli przy włączonym ogrzewaniu silnik zostanie wyłączony i w ciągu 10 minut ponownie włączony, wówczas ogrzewanie jest kontynuowane.

Włącz / wyłącz tryb wentylacji / odszraniania szyby przedniej

\begin{itemizeArrow}
	\itemArrow Nacisnąć przycisk XX pod Infotainment.
\end{itemizeArrow}

Ustawienia

Automatyczne ogrzewanie przedniej szyby

Ogrzewanie przedniej szyby włącza się automatycznie, jeśli szyba przednia zaparuje.

\begin{itemizeArrow}
	\itemArrow Nacisnąć przycisk XX pod Infotainment; Stuknij XX na ekranie systemu Infotainment.
	\itemArrow Wybierz punkt menu dla automatycznego ogrzewania szyby przedniej.
\end{itemizeArrow}

\subsubsection{Rozwiązywanie problemów}

Kontrolka na przycisku lub pod przyciskiem miga

Ogrzewanie szyby nie działa z powodu niskiego poziomu naładowania akumulatora 12 V.

\subsection{Ogrzewanie i wentylacja foteli}

\subsection{Czego należy przestrzegać}

Ryzyko uszkodzenia siedzenia!
\begin{itemizeTriangle}
	\itemTriangle Nie klękaj na fotelach ani nie obciążaj ich punktowo w żaden inny sposób.
	\itemTriangle Nie włączaj ogrzewania w następujących sytuacjach:
		\itemTriangle Fotel nie jest zajęty.
		\itemTriangle Na siedzeniu są przedmioty, takie, np. fotelik dla
\end{itemizeTriangle}
	dziecka.
\begin{itemizeTriangle}
		\itemTriangle Na siedzeniu znajdują się dodatkowe pokrowce
\end{itemizeTriangle}
	na siedzenia lub osłony ochronne.

\subsubsection{Warunki}
\begin{itemizeTick}
	\itemTick Silnik pracuje
\end{itemizeTick}

\subsubsection{Obsługa}

Wariant bez wentylacji siedzenia

\begin{itemizeArrow}
	\itemArrow Stuknąć obszar funkcjonalny XX lub XX na ekranie urządzenia Infotainment, aby włączyć ogrzewanie fotela z przodu.
	\itemArrow Nacisnąć przycisk XX lub XX na konsoli środkowej z tyłu, aby włączyć ogrzewanie siedzeń z tyłu.
\end{itemizeArrow}

Ogrzewanie jest włączane z maksymalną mocą grzewczą. Dalsze naciskanie przycisku powoduje stopniowe zmniejszanie mocy grzewczej aż do wyłączenia.

Moc ogrzewania wyświetla się w postaci odpowiedniej liczby lampek kontrolnych na przycisku XX lub XX na ekranie Infotainment.

Wariant z wentylacją siedzenia
% FIXME: Posiadamy to?

Wyświetlanie poziomu mocy grzewczej / poziomu wentylacji w przycisku

XX Ogrzewanie siedzenia włączone
XX Wentylacja fotela włączona

Jeśli silnik zostanie zatrzymany z włączonym ogrzewaniem / wentylacją i ponownie uruchomiony w ciągu około 10 minut, ogrzewanie / wentylacja będzie kontynuowana zgodnie z ustawieniem przed zatrzymaniem silnika.

Ustawienia

Zablokuj zwiększenie mocy grzewczej ogrzewania tylnych foteli z tyłu

\begin{itemizeArrow}
	\itemArrow Nacisnąć przycisk XX pod Infotainment > XX > z tyłu ; Dotknąć XX na ekranie urządzenia Infotainment.
\end{itemizeArrow}

Gdy funkcja jest włączona, moc grzewcza może być regulowana tylko w dół.

Funkcja ta blokuje również możliwość ustawienia temperatury na wyświetlaczu w tylnej konsoli środkowej.


\subsection{Ogrzewanie kierownicy}

Warunki
\begin{itemizeTick}
	\itemTick Silnik pracuje
\end{itemizeTick}

Obsługa
\begin{itemizeArrow}
	\itemArrow Nacisnąć przycisk XX na kierownicy wielofunkcyjnej.
\end{itemizeArrow}

Ogrzewanie jest włączane z maksymalną mocą grzewczą. Dalsze naciskanie przycisku powoduje stopniowe zmniejszanie mocy grzewczej aż do wyłączenia.

Trzymając przycisk na kierownicy wielofunkcyjnej grzejnik można wyłączyć lub włączyć za pomocą mocy grzewczej ustawionej przed wyłączeniem ogrzewania.
% FIXME: O co chodzi w powyższym? Czy o płynną regulację grzania?

\section{System informacyjny kierowcy}

\subsection{Cyfrowy zestaw wskaźników}

Przegląd
% FIXME: Rys.
Przegląd zestawu wskaźników
A Panel z lampkami kontrolnymi
B Wskaźnik poziomu naładowania akumulatora wysokonapięciowego  % FIXME: Usunąć
C Wskaźnik poziomu paliwa
D Wyświetlacz
E Wyświetl obszar z wybranymi informacjami dla kierowcy
F Centralny obszar wyświetlania

\subsubsection{Obsługa}

Przegląd przycisków / pokręteł na kierownicy wielofunkcyjnej
% FIXME: Rys.

Obrót - przełączaj się pomiędzy pozycjami menu / dopasuj
wartości / ręcznie
zmieniaj skalę mapy
Naciśnięcie - potwierdź pozycję menu
Obrócenie i naciśnięcie – włączenie automatycznej zmiany skali mapy
XX Zmień warianty wyświetlania / pozycje menu wyświetlacza w podmenu po lewej stronie wyświetlacza
XX Zmień warianty wyświetlania / pozycje menu wyświetlacza w podmenu po prawej stronie wyświetlacza

Wyświetl menu z wybranymi informacjami na wyświetlaczu
\begin{itemizeArrow}
	\itemArrow Aby wyświetlić menu po prawej stronie wyświetlacza, nacisnąć pokrętło sterowania A .
	\itemArrow Aby wyświetlić menu po lewej stronie wyświetlacza, pokrętło sterowania A nacisnąć dwa razy.
\end{itemizeArrow}
W zależności od wyposażenia samochodu:
A Widok klasyczny
B Wyświetlacz nawigacji
\begin{itemizeArrow}
	\itemArrow Aby przełączać się między informacjami, obracać pokrętło sterowania A .
\end{itemizeArrow}

\begin{itemizeArrow}
	\itemArrow Aby potwierdzić wybór informacji, naciśnij pokrętło sterowania A .
\end{itemizeArrow}

Niektóre informacje zawierają podmenu.
\begin{itemizeArrow}
	\itemArrow Aby wyświetlić podmenu, nacisnąć przycisk XX lub XX.
	\itemArrow Aby przełączać między pozycjami menu w podmenu, obracać pokrętło sterowania A .
\end{itemizeArrow}

Opcja przełączania między informacjami jest oznaczona strzałkami. Jest to możliwe tylko przez krótki czas.

\subsubsection{Ustawienia}

Ustawianie języka

Język ustawia się w urządzeniu Infotainment w poniższym menu XX.

Nastawianie zegara

Jednostki ustawia się w urządzeniu Infotainment w poniższym menu XX.


Dostosuj jasność podświetlenia zestawu wskaźników

Poziom jasności oświetlenia wskaźników ustawiany jest automatycznie w zależności od panujących warunków oświetleniowych.

Ręczna regulacja jasności odbywa się przy włączonym świetle mijania w następujący sposób.

\begin{itemizeArrow}
	\itemArrow Wybierz w urządzeniu Infotainment w menu XX > wewnątrz > Oświetlenie ambientowe .
	\itemArrow Wykonaj ustawienia.
\end{itemizeArrow}

Ustaw wariant wyświetlacza

\begin{itemizeArrow}
	\itemArrow Naciśnij przycisk XX lub XX na kierownicy wielofunkcyjnej, aby zmienić wariant wyświetlacza.
\end{itemizeArrow}

Wybierz wyświetlane warianty
% FIXME: Rys.

W zależności od wyposażenia samochodu:
A Widok klasyczny
B Wyświetlacz nawigacji
C Systemy wspomagania jazdy
D Widok podstawowy
E Sportowy wyświetlacz (w zależności od wyposażenia)  % FIXME: pewnie niedostępny

\subsection{Wyświetlacz przezierny}

Sposób działania

Wyświetlacz przezierny wyświetla wybrane informacje na przedniej szybie w polu widzenia kierowcy.

Ustawienia

\begin{itemizeArrow}
	\itemArrow Menu XX > wewnątrz .
	\itemArrow Przesuwając palec bokiem do ekranu, wybierz element menu Wyświetlacz przezierny .
	\itemArrow Wybierz pozycję menu Wyświetlacz przezierny .
\end{itemizeArrow}
W wyświetlonym menu można ustawić następujące pozycje menu:
\begin{itemizeTriangle}
	\itemTriangle Aktywacja / dezaktywacja
	\itemTriangle Pozycja
	\itemTriangle Poziom jasności
	\itemTriangle Schemat kolorów
	\itemTriangle Wybór wyświetlanych informacji
\end{itemizeTriangle}

Pozycja może być również regulowana poprzez obrócenie pokrętła A na kierownicy wielofunkcyjnej.
% FIXME: Rys.


\subsection{Dane dotyczące jazdy}

Przegląd
Wyświetlacz danych jazdy działa przy włączonym zapłonie.

Na wyświetlaczu zestawu wskaźników

W zależności od wyposażenia na wyświetlaczu zestawu wskaźników jest wyświetlana np. prędkość, zużycie, informacje o zasięgu itp.

W urządzeniu Infotainment

W Infotainment wyświetlane są następujące dane:
XX Przebyta trasa
XX Czas jazdy
XX Średnia prędkość jazdy
XX Średnie zużycie paliwa
XX Zasięg jazdy

Pamięć

System zapisuje dane jazdy w następujących pamięciach:

Od uruchomienia
\begin{itemizeTriangle}
	\itemTriangle W pamięci zapisywane są dane jazdy od włączenia do wyłączenia zapłonu. Jeśli jazda zostanie przerwana na dłużej niż 2 godziny, pamięć zostanie zresetowana.
\end{itemizeTriangle}

Długookresowo
\begin{itemizeTriangle}
	\itemTriangle Pamięć zbiera informacje o dowolnej liczbie przejazdów o łącznym czasie trwania do 99 godzin i 59 minut lub przebiegu do 9999 km. Jeśli jedna z wymienionych wartości zostanie przekroczona, pamięć zostanie zresetowana.
\end{itemizeTriangle}

Od tankowania
\begin{itemizeTriangle}
	\itemTriangle W pamięci zapisywane są dane jazdy od ostatniego tankowania. Po następnym tankowaniu pamięć zostaje automatycznie skasowana.
\end{itemizeTriangle}


Obsługa

Wyświetl dane jazdy na wyświetlaczu zestawu wskaźników za pomocą kierownicy wielofunkcyjnej
\begin{itemizeArrow}
	\itemArrow Aby wyświetlić menu po prawej stronie wyświetlacza, nacisnąć pokrętło sterowania A.
	\itemArrow Aby wyświetlić menu po lewej stronie wyświetlacza, pokrętło sterowania A nacisnąć dwa razy.
	\itemArrow Aby przełączać się między informacjami, obracać pokrętło sterowania A .
\end{itemizeArrow}

Wyświetlaj dane jazdy w systemie Infotainment

Dane jazdy są wyświetlane w systemie Infotainment w menu XX.

Wybierz pamięć danych jazdy na wyświetlaczu zestawu wskaźników za pomocą kierownicy wielofunkcyjnej i zresetuj
\begin{itemizeArrow}
	\itemArrow Aby wybrać pamięć dla wyświetlanych po prawej stronie danych jazdy, nacisnąć przycisk XX. Obrócić pokrętło A , aby wybrać odpowiednią pamięć lub nacisnąć pokrętło, aby potwierdzić pamięć.
	\itemArrow Aby wybrać pamięć dla wyświetlanych po lewej stronie danych jazdy, nacisnąć przycisk . Obrócić pokrętło A , aby wybrać odpowiednią pamięć lub nacisnąć pokrętło, aby potwierdzić pamięć.
\end{itemizeArrow}

\begin{itemizeArrow}
	\itemArrow Aby zresetować pamięć dla wyświetlanych danych jazdy, nacisnąć przycisk lub . Obróć pokrętło A , aby wybrać resetowanie powiązanej pamięci lub nacisnąć pokrętło sterowania, aby potwierdzić resetowanie.
\end{itemizeArrow}

Wybierz pamięć w urządzeniu Infotainment

Wyboru pamięci dokonuje się, dotykając odpowiedniej karty na ekranie Infotainment w menu XX.

Zresetuj pamięć w urządzeniu Infotainment

Pamięć zostanie zresetowana w systemie Infotainment w następujący sposób:
XX > wewnątrz > Tablica rozdzielcza

\begin{itemizeArrow}
	\itemArrow Naciskając przycisk XX, zresetować powiązaną pamięć.
\end{itemizeArrow}

Ustawienia

Ustaw jednostki

Jednostki ustawia się w systemie Infotainment w poniższym menu XX.

\subsection{Stan samochodu}

Wskaźnik

Stan pojazdu wyświetlany jest w urządzeniu Infotainment w poniższym menu XX.
A Kolorowe obszary pojazdu wskazują powiązane ostrzeżenia.
B Terminy serwisowe i numer identyfikacyjny pojazdu (VIN)
C Kontrola ciśnienia powietrza w oponach
D Informacje dotyczące poziomu oleju silnikowego
E Komunikaty ostrzegawcze dotyczące stanu pojazdu i ich liczby
F Zasięg paliwa

\subsection{Przycisk SET}

Przegląd
Przycisk XX umożliwia szybki dostęp do ustawień następujących układów pojazdu (w zależności od wyposażenia pojazdu):
\begin{itemizeTriangle}
	\itemTriangle XX Wyłączanie kontroli trakcji ASR \guillemotright strona 141, Przegląd
	\itemTriangle XX Włączanie programu stabilizacji ESC Sport \guillemotright strona 141,
	\itemTriangle XX Układ monitorowania wnętrza \guillemotright strona 27, Ustawienia
	\itemTriangle XX Wskaźnik ciśnienia w oponach \guillemotright strona 183,
	\itemTriangle Dostęp do wyboru i regulacji systemów wspomagania kierowcy
	\itemTriangle Dostęp do innych ustawień pojazdu
\end{itemizeTriangle}

% FIXME: Bolero czy Columbus?

% FIXME: Skoda Connect (zrobić to ręcznie, nie umieszczać w instrukcji)

\section{Uruchamianie i jazda}

\subsection{Start}

Warunki działania
\begin{itemizeTick}
	\itemTick Klucz pojazdu znajduje się we wnętrzu.
\end{itemizeTick}

Włączyć i wyłączyć zapłon.
\begin{itemizeArrow}
	\itemArrow Naciśnij przycisk Start.
\end{itemizeArrow}

Uruchom pojazd
\begin{itemizeArrow}
	\itemArrow Zabezpieczyć pojazd hamulcem postojowym.
	\itemArrow Przytrzymaj wciśnięty pedał hamulca.
	\itemArrow Naciśnij przycisk Start.
\end{itemizeArrow}

% FIXME: HV-aku
W temperaturach powyżej ok. -10°C i wystarczającym stanie naładowania akumulatora wysokonapięciowego tylko silnik elektryczny uruchamia się po
uruchomieniu.
Początek pracy silnika elektrycznego jest wyświetlany w następujący sposób:
\begin{itemizeTriangle}
	\itemTriangle Rozlega się sygnał akustyczny.
	\itemTriangle Na wyświetlaczu zestawu wskaźników pokazuje się GOTOWY .
\end{itemizeTriangle}

Wyłączanie silnika
\begin{itemizeArrow}
	\itemArrow Zabezpieczyć pojazd hamulcem postojowym.
	\itemArrow Naciśnij przycisk Start.
\end{itemizeArrow}

OSTRZEŻENIE
Niebezpieczeństwo blokady kierownicy!
\begin{itemizeTriangle}
	\itemTriangle Jeżeli samochód porusza się z wyłączonym silnikiem, zapłon musi być cały czas włączony.
\end{itemizeTriangle}

\subsection{Problemy z uruchamianiem}

Silnik nie daje się uruchomić
\begin{itemizeArrow}
	\itemArrow Wyłączyć zapłon.
	\itemArrow Zaczekaj 30 sekund i powtórz procedurę uruchamiania.
	\itemArrow Jeśli silnik nie uruchamia się, spróbuj rozruchu z akumulatorem 12 V \guillemotright strona 173 lub zwrócić się o pomoc do specjalistycznej firmy.
\end{itemizeArrow}

Silnik nie uruchamia się, na wyświetlaczu pojawia się komunikat dotyczący immobilizera
\begin{itemizeArrow}
	\itemArrow Użyj drugiego klucza pojazdu.
	\itemArrow W przypadku pojawienia się komunikatu o błędzie zwrócić się o pomoc do specjalistycznej stacji obsługi.
\end{itemizeArrow}

Za pomocą przycisku startowego nie można uruchomić, system nie rozpoznał kluczyka XX.

\begin{itemizeArrow}
	\itemArrow Włóż klucz do uchwytu na kubek z przodu.
	\itemArrow Naciśnij przycisk Start.
	\itemArrow W przypadku pojawienia się komunikatu o błędzie
\end{itemizeArrow}
zwrócić się o pomoc do specjalistycznej stacji obsługi.
System może nie rozpoznać kluczyka, jeżeli bateria w kluczyku jest prawie rozładowana lub sygnał zakłócony.
Nie można wyłączyć silnika za pomocą przycisku rozrusznika
\begin{itemizeArrow}
	\itemArrow Przytrzymaj przycisk rozrusznika lub naciśnij dwa razy.
\end{itemizeArrow}

\subsection{Automatyczna skrzynia biegów}

\subsubsection{Tryby automatycznej skrzyni biegów}

\newcommand{\gearFmt}{\textbf}
\newcommand{\gearP}{\gearFmt{P}}
\newcommand{\gearR}{\gearFmt{R}}
\newcommand{\gearN}{\gearFmt{N}}
\newcommand{\gearDS}{\gearFmt{D/S}}
\newcommand{\gearD}{\gearFmt{D}}
\newcommand{\gearS}{\gearFmt{S}}

Wybierz tryb automatycznej skrzyni biegów
\begin{itemizeArrow}
	\itemArrow Aby zmienić tryb automatycznej skrzyni biegów, przesuń dźwignię zmiany biegów do przodu lub do tyłu.
\end{itemizeArrow}
Następnie dźwignia zmiany biegów powraca do pozycji wyjściowej.
\begin{itemizeArrow}
	\itemArrow Naciśnij przycisk P , aby przejść do trybu \gearP. Dźwignia zmiany biegów może być przesuwana do przodu lub do tyłu o dwa położenia. W rezultacie np. tryb XX można wybrać bezpośrednio z trybu i odwrotnie.
\end{itemizeArrow}

Wybrany tryb jest wyświetlany na zestawie wskaźników.

% FIXME: Rys.

\gearP Zaparkowany pojazd

Koła napędowe są mechanicznie zablokowane.
Tryb \gearP tylko przy nieruchomym pojeździe.

\gearR Bieg wsteczny

Tryb \gearR można wybrać przy prędkości poniżej 10 km/h.

\gearN Pozycja neutralna
Nie ma przekazywania mocy silnika na koła.

\gearDS Jazda do przodu / program sportowy
W trybie \gearS zmiana biegów odbywa się przy wyższych prędkościach niż w trybie \gearD.

Wybór między \gearD i \gearS wykonuje się, przesuwając dźwignię zmiany biegów do tyłu.

Aby wybrać tryb \gearD, \gearS i \gearR, silnik należy uruchomić wcześniej. Tylko tryb \gearN można wybrać przy wyłączonym silniku i włączonym zapłonie. To może być użyteczne np. podczas holowania pojazdu lub podczas jazdy przez myjnię samochodową.

Automatyczną skrzynię biegów odblokować z trybu \gearP

XX świeci się - automatyczna skrzynia biegów jest w trybie \gearP
\begin{itemizeArrow}
	\itemArrow Naciśnij pedał hamulca i wybierz żądany tryb.
\end{itemizeArrow}

\begin{itemizeTriangle}
	\itemTriangle Kiedy wybrany jest tryb \gearD, \gearS i \gearR lub \gearN, zabezpieczyć pojazd hamulcem.
\end{itemizeTriangle}

\subsubsection{Obsługa}

Ruszanie
\begin{itemizeArrow}
	\itemArrow Przytrzymaj wciśnięty pedał hamulca.
	\itemArrow Włączyć silnik.
	\itemArrow Za pomocą dźwigni wyboru wybierz żądany tryb.
	\itemArrow Zwolnij pedał hamulca i lekko naciśnij pedał przyspieszenia.
\end{itemizeArrow}

Maksymalne przyspieszenie podczas jazdy (kick-down)

\begin{itemizeArrow}
	\itemArrow Wcisnąć do końca pedał przyspieszenia.
\end{itemizeArrow}

Na wyświetlaczu zestawu wskaźników pokazuje się XX.

Tymczasowe zatrzymanie (np. na skrzyżowaniu)

\begin{itemizeArrow}
	\itemArrow Dźwignię zmiany biegów zostawić w położeniu \gearDS i zabezpieczyć pojazd pedałem hamulca.
\end{itemizeArrow}

Zatrzymanie
\begin{itemizeArrow}
	\itemArrow Przytrzymaj wciśnięty pedał hamulca.
	\itemArrow Zabezpieczyć pojazd hamulcem postojowym.
	\itemArrow Wybrać tryb \gearP.
	\itemArrow Wyłączyć silnik.
\end{itemizeArrow}

Podczas wyłączania silnika w trybie \gearDS lub \gearR automatycznie zostaje wybrany tryb \gearP.

Jazda na biegu jałowym
Jeśli system wykryje, że podczas jazdy nie trzeba włączać żadnego biegu, automatycznie przełącza się w położenie neutralne.
Będzie wyświetlany komunikat XX na zestawie wskaźników.

Wymagania dotyczące jazdy w pozycji neutralnej
\begin{itemizeTick}
	\itemTick Tryb \gearD jest wybrany.
	\itemTick Pedały gazu i hamulca są zwolnione.
	\itemTick Prędkość wynosi 20-130 km/h.
	\itemTick Nie ma urządzenia podłączonego do gniazda przyczepy.
\end{itemizeTick}

Maksymalne przyspieszenie przy uruchamianiu (procedura startowa)

Funkcja pozwala na maksymalne przyspieszenie podczas uruchamiania.

\begin{itemizeArrow}
	\itemArrow Wybrać tryb \gearS lub tryb ręcznej zmiany biegów.
	\itemArrow Dezaktywuj ASR \guillemotright strona 142 lub aktywuj ESC Sport \guillemotright strona 142.
	\itemArrow Przytrzymaj wciśnięty pedał hamulca.
	\itemArrow Wcisnąć do końca pedał przyspieszenia.
	\itemArrow Zwolnić pedał hamulca.
\end{itemizeArrow}

Pojazd porusza się z maksymalnym przyspieszeniem.

Ręczne przełączanie manetkami zmiany biegów na kierownicy

- Zmiana biegu na niższy (redukcja biegu)
+ Zmiana biegu na wyższy

\begin{itemizeArrow}
	\itemArrow Aby włączyć, nacisnąć manetkę - lub +
	\itemArrow Aby wyłączyć, przytrzymaj przełącznik kołyskowy + lub wybierz inny tryb.
\end{itemizeArrow}
Zaprzestanie obsługi za pomocą manetek -/+ w trybie przez określony czas spowoduje automatyczne wyłączenie ręcznej zmiany biegów.
W trybie nie ma automatycznego wyłączania trybu przełączania ręcznego.
Jeśli grozi przekroczenie prędkości obrotowej silnika
\begin{itemizeTriangle}
	\itemTriangle Przekładnia automatycznie przełącza się na wyższy bieg.
	\itemTriangle Przekładnia zapobiega przechodzeniu do następnego niższego biegu.
\end{itemizeTriangle}

Rozwiązywanie problemów

Tryb automatycznej skrzyni biegów nie może być ustawiony w zwykły sposób
\begin{itemizeArrow}
	\itemArrow Odblokuj manualnie dźwignię zmiany biegów \guillemotright strona 134.
\end{itemizeArrow}

Skrzynia biegów przegrzana

XX zapala się razem z XX

\begin{itemizeArrow}
	\itemArrow Dalsza jazda jest możliwa tylko przy zachowaniu odpowiedniej ostrożności.
\end{itemizeArrow}

XX zapala się razem z XX

\begin{itemizeArrow}
	\itemArrow Nie wolno kontynuować jazdy! Zatrzymać pojazd i wyłączyć silnik.
	\itemArrow Po zgaśnięciu lampki kontrolnej można wznowić jazdę.
	\itemArrow Jeżeli lampka kontrolna nie zgaśnie, nie kontynuować jazdy! Skorzystać z pomocy specjalistycznej stacji obsługi.
\end{itemizeArrow}

Skrzynia biegów była zakłócona

XX zapala się razem z XX

\begin{itemizeArrow}
	\itemArrow Dalsza jazda jest możliwa tylko przy zachowaniu odpowiedniej ostrożności.
	\itemArrow Niezwłocznie skorzystać z pomocy specjalistycznej stacji obsługi.
\end{itemizeArrow}

XX zapala się razem z XX

\begin{itemizeArrow}
	\itemArrow Nie wolno kontynuować jazdy! Skorzystać z pomocy specjalistycznej stacji obsługi.
\end{itemizeArrow}

Ruszanie zablokowanym pojazdem

Szybko przestawić dźwignię zmiany biegów między \gearDS i \gearR. Pojazd wchodzi w kołysanie i łatwiej nim ruszyć.

\subsubsection{Manualne odryglowanie automatycznej skrzyni biegów}

% FIXME: Rys.

\begin{itemizeArrow}
	\itemArrow Zabezpieczyć pojazd hamulcem postojowym.
	\itemArrow Włóż płaski śrubokręt lub podobne narzędzie do szczeliny w obszarze strzałki A .
	\itemArrow Ostrożnie poluzuj i podnieś osłonę jarzma przełączającego
\end{itemizeArrow}

\begin{itemizeArrow}
	\itemArrow Poluzuj cięgno w kierunku strzałki 1 .
	\itemArrow Wyjmij cięgno do oporu w kierunku strzałki 2 .
\end{itemizeArrow}

\subsection{Tryb prowadzenia pojazdu}

Tryb jazdy oferuje możliwość dostosowania zachowania podczas jazdy do pożądanego stylu jazdy.

\subsubsection{Przegląd}

Informacje o wybranym trybie jazdy są wyświetlane na zestawie wskaźników.

XX Tryb Eco
Tryb Eco jest odpowiedni do spokojnego sposobu jazdy i pomaga oszczędzać paliwo.

XX Tryb Comfort
Tryb ten nadaje się do jazdy po drogach o gorszej nawierzchni albo do długich jazd po autostradach.

XX Tryb Normal
Tryb Normal nadaje się do normalnego stylu jazdy.

XX Tryb Sport
Tryb Sport nadaje się do sportowego stylu jazdy.

XX Tryb Individual
Tryb Individual umożliwia indywidualne dopasowanie niektórych układów pojazdu.

\subsubsection{Obsługa}

Wybierz tryb jazdy

\begin{itemizeArrow}
	\itemArrow Nacisnąć przycisk XX.
\end{itemizeArrow}

Na ekranie urządzenia Infotainment ukazuje się menu trybu jazdy.

Ponowne naciśnięcie przycisku XX wyłącza system.

% FIXME: Rys.

A Informacje o ustawieniach systemowych aktualnie wybranego trybu / ustawień systemu w trybie Individual
B Menu trybu jazdy

\begin{itemizeArrow}
	\itemArrow Dotknij odpowiedniego obszaru funkcjonalnego B .
\end{itemizeArrow}

Po włączeniu zapłonu automatycznie ustawiany jest tryb Normal

Wybierając tryb \gearS automatycznej skrzyni biegów, automatycznie wybierany jest tryb jazdy Sport, gdy zezwala na to system.


\subsection{Ekonomiczny styl jazdy}

Wskazówki dotyczące ekonomicznej jazdy

\begin{itemizeTriangle}
	\itemTriangle Nie rozgrzewać silnika w stojącym pojeździe. Jeżeli to możliwe, należy ruszyć natychmiast po uruchomieniu silnika.
\end{itemizeTriangle}

Warunki jazdy przyjazne dla zużycia

Jeśli system wykryje, że podczas jazdy nie trzeba włączać żadnego biegu, system automatycznie przełącza się w położenie neutralne. Zmniejsza to zużycie paliwa.

XX w zestawie wskaźników zapala się.

\subsection{Ucha holownicze i holowanie}

\subsubsection{Gniazdo ucha holowniczego}

Usuń przednią nasadkę
% FIXME: Rys.
Naciśnij zatyczkę i wyjmij ją.

Usuń tylną nasadkę
% FIXME: Rys.
Naciśnij zatyczkę i wyjmij ją.

Montaż ucha holowniczego
% FIXME: Rys.
\begin{itemizeArrow}
	\itemArrow Wkręcić ucho holownicze.
	\itemArrow Włóż klucz do kół lub podobny przedmiot przez ucho holownicze.
	\itemArrow Dokręcić ucho holownicze.
\end{itemizeArrow}

Po holowaniu
\begin{itemizeArrow}
	\itemArrow Odkręć ucho holownicze.
	\itemArrow Włóż zaślepkę.
\end{itemizeArrow}

\subsubsection{Holowanie samochodu}

\begin{itemizeTriangle}
	\itemTriangle Lina holownicza nie może być skręcona.
	\itemTriangle Podczas holowania prędkością maksymalna wynosi 50 km/h.
	\itemTriangle Do holowania należy używać plecionej liny z włókien syntetycznych. Nie używaj skręcanej liny holowniczej.
	\itemTriangle Podczas holowania dbać o to, aby linka holownicza była stale naprężona.
	\itemTriangle Nie przekraczaj odległości holowania 50 km.
\end{itemizeTriangle}

Niebezpieczeństwo uszkodzenia skrzyni biegów!
\begin{itemizeTriangle}
	\itemTriangle Jeśli w skrzyni biegów nie ma już oleju, samochód może być holowany tylko z podniesioną osią przednią albo na specjalnym holowniku lub przyczepie.
	\itemTriangle Pojazdu nie wolno holować z podniesioną tylną osią.
	\itemTriangle Pojazdu nie wolno holować bez włączonego zapłonu.
\end{itemizeTriangle}

Informacje dla kierowcy pojazdu ciągniętego
\begin{itemizeArrow}
	\itemArrow Wybrać tryb \gearN automatycznej skrzyni biegów.
\end{itemizeArrow}

Gdy zapłon jest wyłączony, wspomaganie hamulców i wspomaganie kierownicy nie działają. Ponadto istnieje ryzyko zablokowania się blokady kierownicy.
\begin{itemizeTriangle}
	\itemTriangle Jeśli nie można uruchomić silnika, należy włączyć zapłon!
\end{itemizeTriangle}

\subsection{Hamulce}

Sprawdzanie poziomu płynu hamulcowego

Warunki sprawdzania
\begin{itemizeTick}
	\itemTick Pojazd stoi na równym terenie
	\itemTick Silnik jest wyłączony
\end{itemizeTick}

Sprawdzenie

Poziom napełnienia musi znajdować się w zaznaczonym zakresie.
\begin{itemizeArrow}
	\itemArrow Jeśli poziom jest poniżej znaku, nie kontynuuj jazdy.
	\itemArrow Nie dolewaj płynu.
	\itemArrow Skorzystać z pomocy specjalistycznej stacji obsługi.
\end{itemizeArrow}

W razie zbyt niskiego poziomu płynu hamulcowego na wyświetlaczu zestawu wskaźników świeci się lampka kontrolna XX.
Mimo to zalecamy regularnie sprawdzać poziom płynu hamulcowego bezpośrednio w zbiorniku.

OSTRZEŻENIE
Niebezpieczeństwo awarii hamulca!
Jeśli poziom płynu hamulcowego w krótkim czasie wyraźnie się obniży lub spadnie poniżej oznaczenia XX, może to oznaczać, że w układzie hamulcowym
powstała nieszczelność.
\begin{itemizeTriangle}
	\itemTriangle Nie wolno kontynuować jazdy! Skorzystać z pomocy specjalistycznej stacji obsługi.
\end{itemizeTriangle}

Specyfikacja

Płyn hamulcowy musi być zgodny z normą VW 501 14. Ta norma jest zgodna z wymaganiami normy FMVSS 116 DOT4.

Wymienić

Zlecić wymianę płynu w specjalistycznej stacji obsługi.

\subsubsection{Rozwiązywanie problemów}

Zbyt niski poziom płynu hamulcowego

XX świeci

\begin{itemizeArrow}
	\itemArrow Nie wolno kontynuować jazdy! Skorzystać z pomocy specjalistycznej stacji obsługi.
\end{itemizeArrow}

Zakłócony układ hamulcowy i układ ABS

XX zapala się razem z XX

\begin{itemizeArrow}
	\itemArrow Nie wolno kontynuować jazdy! Skorzystać z pomocy specjalistycznej stacji obsługi.
\end{itemizeArrow}

Zużyte klocki hamulcowe

XX świeci
\begin{itemizeArrow}
	\itemArrow Natychmiast jedź do najbliższej specjalistycznej stacji obsługi z zachowaniem należytej ostrożności.
\end{itemizeArrow}

Zmniejszony efekt hamowania

Wilgotne, zamarznięte, zabrudzone solą lub skorodowane hamulce mogą wpływać na efekt hamowania.

\begin{itemizeArrow}
	\itemArrow Wyczyść hamulce, kilkakrotnie hamując, jeśli pozwalają na to warunki ruchu drogowego.
\end{itemizeArrow}
Gdy silnik jest zatrzymany, wspomaganie hamulców nie działa
\begin{itemizeArrow}
	\itemArrow Wciśnij pedał hamulca mocniej.
\end{itemizeArrow}

\subsection{Elektryczny hamulec postojowy}

\subsubsection{Obsługa}

Niebezpieczeństwo blokady przycisku!
\begin{itemizeTriangle}
	\itemTriangle Wgłębiony uchwyt na palce przed przyciskiem musi być pusty.
\end{itemizeTriangle}

Ręczne włączanie
% FIXME: Rys.
\begin{itemizeArrow}
	\itemArrow Na przycisku XX pociągnij to i przytrzymaj.
\end{itemizeArrow}
Świecą się następujące symbole.
\begin{itemizeTriangle}
	\itemTriangle XX w przycisku.
	\itemTriangle XX w zestawie wskaźników.
\end{itemizeTriangle}

Wyłączanie automatyczne

Hamulec postojowy wyłącza się automatycznie podczas uruchamiania.

Jeśli automatyczne wyłączenie należy zablokować np. podczas uruchamiania na zboczu, pociągnąć i przytrzymać przycisk XX.

Warunki automatycznego wyłączenia

\begin{itemizeTick}
	\itemTick Drzwi kierowcy zamknięte, w niektórych przypadkach zapięty pas bezpieczeństwa kierowcy
\end{itemizeTick}

Wyłączanie ręczne

\begin{itemizeArrow}
	\itemArrow Przy włączonym zapłonie uruchomić pedał hamulca, naciskając jednocześnie przycisk XX.
\end{itemizeArrow}

\subsubsection{Rozwiązywanie problemów}

Błąd hamulca postojowego

XX świeci

Komunikat dotyczący błędu hamulca postojowego
\begin{itemizeArrow}
	\itemArrow Skorzystać z pomocy specjalistycznej stacji obsługi.
\end{itemizeArrow}

Parkowanie na zboczu o zbyt dużym nachyleniu

XX miga

Wiadomość dotycząca pozycji parkingowej z nadmiernym nachyleniem
\begin{itemizeArrow}
	\itemArrow Znajdź inny parking z mniejszym nachyleniem.
\end{itemizeArrow}

Hałas podczas korzystania z hamulca postojowego

Hałasy podczas korzystania z hamulca postojowego są normalne. Nie jest to usterka.

Akumulator 12 V pojazdu jest rozładowany, hamulec postojowy nie może być wyłączony

\begin{itemizeArrow}
	\itemArrow Akumulator 12 V podłącz do źródła zasilania, np. do akumulatora 12 V innego pojazdu.
\end{itemizeArrow}

\subsubsection{Hamowanie awaryjne w przypadku wadliwego układu hamulcowego}

Włączanie
% FIXME: Rys.
\begin{itemizeArrow}
	\itemArrow Na przycisku XX pociągnij to i przytrzymaj.
\end{itemizeArrow}

Pojazd zaczyna silnie hamować i brzmi sygnał akustyczny.

Wyłączanie
\begin{itemizeArrow}
	\itemArrow Zwolnić przycisk XX.
\end{itemizeArrow}
Lub:
\begin{itemizeArrow}
	\itemArrow Nacisnąć pedał gazu.
\end{itemizeArrow}

\subsection{Funkcja automatycznego zatrzymania Auto Hold}

Auto Hold automatycznie zabezpiecza pojazd przed stoczeniem się po zatrzymaniu.

\subsubsection{Sposób działania}

Zabezpiecz i zwolnij pojazd

Przy zatrzymaniu pojazd zostanie automatycznie zatrzymany przez Auto Hold .

Hamulec postojowy włączony. Pedał hamulca można zwolnić.
Podczas ruszania pojazd zostanie automatycznie zwolniony przez Auto Hold.

Automatyczna ochrona za pomocą hamulca postojowego

W pewnych okolicznościach pojazd można zabezpieczyć hamulcem postojowym.
XX Hamulec postojowy włączony.

\subsubsection{Warunki działania}

\begin{itemizeTick}
	\itemTick Drzwi kierowcy są zamknięte.
	\itemTick Silnik pracuje.
	\itemTick Funkcja Auto Hold jest wyłączona.
	\itemTick Tryb automatycznej skrzyni biegów \gearN nie jest wybrany.
\end{itemizeTick}

\subsubsection{Ustawienia}

Aktywacja / dezaktywacja
\begin{itemizeArrow}
	\itemArrow Nacisnąć przycisk XX.
\end{itemizeArrow}

Włączenie pokazywane jest przez podświetlenie symbolu XX w przycisku.

Wyłącz funkcję Auto Hold, aby umożliwić pojazdowi staczanie się w razie potrzeby podczas jazdy przez myjnię samochodową.

\section{Systemy wspomagania jazdy}

\subsection{Systemy hamowania i stabilizacji}

\subsubsection{Przegląd}

Stabilizacja toru jazdy (ESC)

ESC pomaga ustabilizować pojazd w sytuacjach granicznych (np. kiedy pojazd zaczyna się obracać). ESC hamuje poszczególne koła, aby zachować kierunek
jazdy.

XX miga - interweniuje ESC

ESC Sport

Funkcja ESC Sport umożliwia sportowy tryb jazdy.

XX świeci się - ESC Sport jest włączony

Kontrola trakcji (ASR)

ASR pomaga stabilizować pojazd podczas przyspieszania lub jazdy po drogach o niskiej przyczepności.
ASR redukuje siłę napędową przenoszoną na koła podczas obracania się kół w miejscu.

XX miga - interweniuje ASR

Układ ABS

Układ ABS pomaga w utrzymaniu kontroli nad pojazdem podczas hamowania awaryjnego. Ingerencję systemu ABS można rozpoznać po pulsowaniu pedału hamulca powiązanym z wyraźnymi odgłosami.

System zapobiegający poślizgowi kół napędowych podczas hamowania silnikiem (MSR)

MSR pomaga kontrolować pojazd przy gwałtownym zwolnieniu pedału przyspieszenia, np. na oblodzonej nawierzchni. Jeżeli koła napędowe są zablokowane, MSR zwiększa prędkość obrotową silnika. Powoduje to redukcję siły hamowania silnika i koła mogą znowu swobodnie się obracać.

Elektroniczna blokada mechanizmu różnicowego (EDS)

EDS pomaga stabilizować pojazd podczas jazdy po pasach o różnej przyczepności pod poszczególnymi kołami. EDS wyhamowuje obracające się koło i przekazuje moc do innego koła napędowego.

Elektroniczna blokada mechanizmu różnicowego (XDS+)

XDS+ pomaga ustabilizować pojazd podczas szybkiego pokonywania zakrętów poprzez hamowanie wewnętrznego koła napędzanej osi.

Aktywna pomoc w kierowaniu (DSR)

DSR pomaga kierowcy odzyskać panowanie nad samochodem w krytycznych sytuacjach podczas jazdy.

Wspomaganie podjazdu pod górę

Układ wspomagania ruszania pod górę pomaga w ruszaniu pod górę, hamując pojazd przez około 2 sekundy po zwolnieniu pedału hamulca.

Warunki działania
\begin{itemizeTick}
	\itemTick Nachylenie wynosi co najmniej 5%.
	\itemTick Drzwi kierowcy są zamknięte.
\end{itemizeTick}

Hamulec multikolizyjny (MCB)

MCB pomaga zwolnić i ustabilizować pojazd po uderzeniu w przeszkodę. Zmniejsza to ryzyko dalszych kolizji.

Warunki działania
\begin{itemizeTick}
	\itemTick Doszło do zderzenia czołowego, bocznego i uderzenia od tyłu o określonej energii.
	\itemTick Prędkość jazdy podczas zderzenia przekraczała 10 km/h.
	\itemTick Hamulce, ESC i pozostałe wymagane układy elektryczne pozostają sprawne po zderzeniu.
	\itemTick Nie wciśnięto pedału gazu.
\end{itemizeTick}

Elektromechaniczny wzmacniacz siły hamowania (eBKV)
% FIXME: Czy mamy tę opcję?

eBKV ułatwia działanie pedału hamulca.

Po wyłączeniu zapłonu funkcja eBKV jest ograniczona lub niedostępna.

Jeśli pojazd jest hamowany za pomocą układu asystenta, mogą wystąpić pulsujące ruchy pedału hamulca.


\subsubsection{Ustawienia}

Dezaktywacja / aktywacja ASR
\begin{itemizeArrow}
	\itemArrow Nacisnąć przycisk XX pod Infotainment.
\end{itemizeArrow}

W wyświetlonym menu urządzenia Infotainment
funkcję można wyłączyć lub włączyć.

Świeci po dezaktywacji XX w zestawie wskaźników zapala się.

XX gaśnie po ponownej aktywacji.

Wyłączenie ASR może być pomocne w następujących sytuacjach:
\begin{itemizeTriangle}
	\itemTriangle Jazda z łańcuchami śniegowymi.
	\itemTriangle Podczas jazdy w głębokim śniegu lub po sypkiej nawierzchni.
	\itemTriangle Ruszanie zablokowanym pojazdem
\end{itemizeTriangle}

Włącz / wyłącz ESC Sport

\begin{itemizeArrow}
	\itemArrow Nacisnąć przycisk XX pod Infotainment.
\end{itemizeArrow}

W wyświetlonym menu urządzenia Infotainment funkcję można wyłączyć lub włączyć.

Świeci po dezaktywacji XX w zestawie wskaźników zapala się.

XX gaśnie po ponownej aktywacji.

Rozwiązywanie problemów

ESC lub ASR zakłócony / wyłączony przez system

XX świeci
\begin{itemizeArrow}
	\itemArrow Zatrzymaj silnik i zacznij od nowa.
	\itemArrow Jeśli lampka sygnalizacyjna nie zgaśnie po przejechaniu niewielkiej odległości, skorzystaj z pomocy specjalistycznej stacji obsługi.
\end{itemizeArrow}

ABS zakłócony

XX świeci
\begin{itemizeArrow}
	\itemArrow Dalsza jazda jest możliwa tylko przy zachowaniu odpowiedniej ostrożności. Skorzystać z pomocy specjalistycznej stacji obsługi.
\end{itemizeArrow}

Zakłócony układ hamulcowy i układ ABS

XX zapala się razem z XX
\begin{itemizeArrow}
	\itemArrow Nie wolno kontynuować jazdy! Skorzystać z pomocy specjalistycznej stacji obsługi.
\end{itemizeArrow}

Elektromechaniczny wzmacniacz siły hamowania zakłócony

XX świeci
\begin{itemizeArrow}
	\itemArrow Nie wolno kontynuować jazdy! Skorzystać z pomocy specjalistycznej stacji obsługi.
\end{itemizeArrow}

\subsection{Front Assist}

Front Assist monitoruje odległość do poprzedzające-
go pojazdu i wskazuje, kiedy bezpieczna odległość
została przekroczona. W sytuacjach granicznych pomaga uniknąć kolizji poprzez hamowanie interwen-
cyjne.

Wskaźnik statusu na wyświetlaczu zestawu
wskaźników
XX świeci się - bezpieczna odległość przekroczona
XX świeci się - ostrzeżenie o niebezpieczeństwie kolizji

Automatyczne hamowanie w przypadku zagrożenia kolizją

W przypadku zagrożenia kolizją wyświetlany jest symbol XX. Jeśli nie zareagujesz na ostrzeżenie, pojazd zwolni.
Przy prędkości jazdy do 30 km/h pojazd jest hamowany bez uprzedniego ostrzeżenia.

Automatyczną ingerencję w hamowanie można przerwać przez użycie pedału gazu lub przez ingerencję kierownicą.

Przy automatycznym hamowaniu wzrasta ciśnienie w układzie hamulcowym. Pedału hamulca nie można obsługiwać za pomocą zwykłego skoku pedału.

Warunki działania

\begin{itemizeTick}
	\itemTick ASR jest aktywny.
	\itemTick Prędkość jazdy przekracza 5 km/h.
\end{itemizeTick}

Ograniczenie funkcjonowania

Działanie systemu może być znacznie ograniczone lub system może nie działać np. w następujących sytuacjach:

\begin{itemizeTriangle}
	\itemTriangle Około 30 sekund po uruchomieniu (XX wyświetlane jest na zestawie wskaźników)
\end{itemizeTriangle}
Dezaktywacja / aktywacja i ustawianie w urządze-
\begin{itemizeTriangle}
	\itemTriangle Podczas jazdy przez ostry zakręt
	\itemTriangle W przypadku interwencji ESC
\end{itemizeTriangle}

Ustawienia

Automatyczna aktywacja

Front Assist włącza się automatycznie po włączeniu zapłonu.

Aktywacja / dezaktywacja w zestawie wskaźników
\begin{itemizeArrow}
	\itemArrow Nacisnąć przycisk XX na kierownicy wielofunkcyjnej.
	\itemArrow Wybierz pozycję menu Front Assist .
\end{itemizeArrow}

Dezaktywacja / aktywacja i ustawianie w urządzeniu Infotainment

W menu XX
\begin{itemizeTriangle}
	\itemTriangle aktywne - Aktywacja/dezaktywacja Front Assist
	\itemTriangle Ostrzeżenie wstępne – Aktywacja/dezaktywacja i ustawienie poziomu odstępu, przy którym pojawia się ostrzeżenie
	\itemTriangle Pokaż ostrzeżenie o odległości - Aktywacja/dezaktywacja wyświetlania ostrzeżenia o odległości.
	\itemTriangle Obsługa unikania - jeśli funkcja jest aktywna, system może pomóc w uniknięciu przeszkody poprzez hamowanie i interwencję kierownicą
	\itemTriangle Funkcja przyhamowania - gdy funkcja jest aktywna, system może spowolnić pojazd podczas skrętu przy niskich prędkościach, aby uniknąć zderzenia z nadjeżdżającym pojazdem
\end{itemizeTriangle}

Wskazanie dezaktywacji systemu na wyświetlaczu zestawu wskaźników XX

Podczas przewożenia pojazdu na samochodzie ciężarowym, promie transportującym samochody itp.

\subsubsection{Rozwiązywanie problemów}

Front Assist jest niedostępny

XX zapala się razem z XX

\begin{itemizeArrow}
	\itemArrow Oczyść przedni czujnik radaru.
	\itemArrow Zatrzymaj silnik i zacznij od nowa.
	\itemArrow Jeśli Front Assist jest nadal niedostępny, sprawdź światła hamowania w pojeździe lub przyczepie.
	\itemArrow Jeśli światła hamowania działają, a Front Assist jest nadal niedostępny, należy skorzystać z pomocy specjalistycznej stacji obsługi.
\end{itemizeArrow}
Wystąpiło bezpodstawne ostrzeżenie lub interwencja systemu
\begin{itemizeArrow}
	\itemArrow Oczyść przedni czujnik radaru.
	\itemArrow Jeśli system nie działa poprawnie, wyłącz Front Assist i skorzystaj z pomocy specjalistycznej stacji obsługi.
\end{itemizeArrow}

\subsection{Wykrywanie pieszych}

Rozpoznawanie pieszych pomaga zapobiec kolizji z przechodzącymi lub poruszającymi się w kierunku wzdłużnym pieszymi przez automatyczne interwencje hamowania.

Wskaźnik statusu na wyświetlaczu zestawu wskaźników

XX zapala się - Niebezpieczeństwo kolizji

Ryzyko kolizji w zakresie prędkości 5-30 km/h

System włącza automatyczne hamowanie.

Ryzyko kolizji w zakresie prędkości 30-85 km/h

System najpierw ostrzega o zagrożeniu kolizją. Jeżeli kierowca nie odpowiada, pojazd jest automatycznie hamowany.

Warunki działania
\begin{itemizeTick}
	\itemTick Front Assist jest włączony.
	\itemTick Prędkość wynosi 5-85 km/h.
\end{itemizeTick}

Ogranicznik prędkości

Ogranicznik prędkości steruje maksymalną prędkością jazdy zgodnie z ustawionym ograniczeniem prędkości.

Jeśli limit zostanie przekroczony podczas zjeżdżania z góry, należy wyhamować pojazd za pomocą pedału hamulca.

Wskaźnik statusu na wyświetlaczu zestawu wskaźników
XX gdy świeci – układ wspomagania zjeżdżania ze wzniesienia jest aktywowany
gdy miga - ustawiony limit jest przekroczony

Kiedy zaczyna się regulacja, XX jest podświetlony i wyświetlany jest ustawiony limit.

Warunki działania
\begin{itemizeTick}
	\itemTick Prędkość jazdy przekracza 30 km/h.
\end{itemizeTick}

\subsubsection{Obsługa}
% FIXME: w zależności od wariantu

\subsubsection{Rozwiązywanie problemów}

Ograniczenie prędkości zostało zakłócone

XX świeci

\begin{itemizeArrow}
	\itemArrow Skorzystać z pomocy specjalistycznej stacji obsługi.
\end{itemizeArrow}

\subsection{Tempomat}

Tempomat samoczynnie utrzymuje wybraną, stałą prędkość jazdy, kierowca nie musi naciskać pedału gazu.

Jest to możliwe oczywiście tylko w zakresie, na jaki pozwala moc silnika lub moment hamujący silnika.

OSTRZEŻENIE
Niebezpieczeństwo niezamierzonego uruchomienia tempomatu!
\begin{itemizeTriangle}
	\itemTriangle Wyłącz tempomat po użyciu.
\end{itemizeTriangle}

Wskaźnik statusu na wyświetlaczu zestawu wskaźników
XX świeci się - tempomat jest aktywny.
XX świeci się - regulacja prędkości jest aktywna
Po uruchomieniu regulacji wyświetlana jest ustawiona prędkość.

Warunki działania
\begin{itemizeTick}
	\itemTick Prędkość jazdy przekracza 20 km/h.
\end{itemizeTick}

Obsługa

Obsługa dźwignią

% FIXME: Rys.

A ON Aktywacja tempomatu (regulacja nieaktywna)

OFF Wyłączenie tempomatu

CANCEL przerwanie regulacji (pozycja sprężynowana)

B XX Ponowne przejęcie regulacji\footnote{Jeżeli nie została ustawiona żadna prędkość, przejmowana jest prędkość aktualna.} / zwiększenie prędkości
C XX Rozpocznij regulację przy bieżącej prędkości / zmniejsz prędkość
D XX pokaż menu systemów wspomagających - możliwość przełączania między GRA i ogranicznikiem prędkości

Chwilowe przyspieszenie
\begin{itemizeArrow}
	\itemArrow Nacisnąć pedał gazu.
\end{itemizeArrow}
Po zwolnieniu pedału gazu prędkość jazdy zmniejszy się do wcześniej zapamiętanej wartości.
Przerwanie działania tempomatu
\begin{itemizeTriangle}
	\itemTriangle Po naciśnięciu pedału hamulca
	\itemTriangle Po interwencji ESC
\end{itemizeTriangle}

Rozwiązywanie problemów

Usterka tempomatu

XX świeci
\begin{itemizeArrow}
	\itemArrow Skorzystać z pomocy specjalistycznej stacji obsługi.
\end{itemizeArrow}

\subsection{Aktywny tempomat (ACC)}

Kontrola ACC
ACC utrzymuje ustawioną prędkość. Jeśli pojazd zbliża się do pojazdu z przodu, ACC automatycznie zacznie utrzymywać ustaloną odległość od tego pojazdu.
Utrzymanie ustawionej prędkości i odległości będzie dalej zwane regulacja.

OSTRZEŻENIE
ACC nie reaguje na obiekty przecinające drogę lub jadące z naprzeciwka.

ACC przewidziany jest do stosowania przede wszystkim na autostradach.

pACC (predykcyjny tempomat)

pACC stanowi rozszerzenie systemu ACC.

pACC dostosowuje prędkość do rozpoznanych ograniczeń prędkości i trasy, np. B. zakręty, skrzyżowania lub ronda.

System wykorzystuje następujące źródła do oceny sytuacji
\begin{itemizeTriangle}
	\itemTriangle Dane nawigacyjne
	\itemTriangle System identyfikacji znaków drogowych
	\itemTriangle Czujniki, radary i kamery
\end{itemizeTriangle}

OSTRZEŻENIE
pACC nie reaguje na przejazdy kolejowe.

Zakres prędkości

W zależności od wyposażenia, ACC umożliwia ustawienie prędkości w zakresie 30 - 210 km/h.
Jeżeli uruchamiane jest sterowanie z prędkością mniejszą niż 30 km/h, prędkość jest automatycznie zwiększana do 30 km/h lub regulowana zgodnie z prędkością poprzedzającego pojazdu.

Stopień odstępu

Odległość do poprzedzającego pojazdu jest regulowana na pięciu różnych poziomach.

OSTRZEŻENIE
\begin{itemizeTriangle}
	\itemTriangle Zachowaj minimalną odległość zgodnie z przepisami krajowymi.
\end{itemizeTriangle}

Automatyczne zatrzymanie i uruchomienie

Przy użyciu ACC samochody można całkowicie zatrzymać i ponownie ruszyć. W przypadku dłuższego postoju regulacja jest przerywana.

Aby wznowić regulację:
\begin{itemizeArrow}
	\itemArrow Nacisnąć pedał gazu.
\end{itemizeArrow}
Lub:
\begin{itemizeArrow}
	\itemArrow Ustaw dźwignię w pozycji XX.
\end{itemizeArrow}
Lub:
\begin{itemizeArrow}
	\itemArrow Dotyczy pojazdów z wykrywaniem dłoni na kierownicy: złap ponownie kierownicę.
\end{itemizeArrow}

Sterowanie zostaje przerwane, gdy pedał hamulca jest wciśnięty.

Wyprzedzanie

Jeżeli samochód zjedzie na pas wyprzedzania i nie zostanie rozpoznany żaden poprzedzający go samochód, ACC powoduje, że samochód przyspiesza do ustawionej prędkości i utrzymuje tę prędkość.

Wskaźnik statusu na wyświetlaczu zestawu wskaźników

XX świeci się - ACC jest aktywowany
XX świeci się - Kontrola jest aktywna

Po uruchomieniu sterowania wyświetlana jest ustawiona prędkość.

XX świeci się - ACC nie zwalnia wystarczająco
\begin{itemizeArrow}
	\itemArrow Nacisnąć pedał hamulca.
\end{itemizeArrow}
XX świeci się - ACC reguluje prędkość jazdy zgodnie z dopuszczalną prędkością
XX świeci się - ACC kontroluje prędkość jazdy w zależności od zbliżającego się ruchu okrężnego
XX świeci się - ACC reguluje prędkość jazdy w zależności od zbliżającego się ruchu okrężnego
XX świeci się - ACC kontroluje prędkość jazdy w zależności od trasy

\subsubsection{Obsługa}

Obsługa dźwignią

% FIXME: Rys.

XX Aktywacja ACC (regulacja nieaktywna)
XX Rozpoczęcie regulacji (ponowne uruchomienie) / zwiększanie prędkości skokowo o 1 km/h (położenie sprężynujące)
XX Przerwanie regulacji (pozycja sprężynowania)
XX Dezaktywacja ACC
XX Zwiększanie prędkości skokowo o 10 km/h
XX Zmniejszanie prędkości skokowo o 10 km/h
XX Ustawianie poziomu odstępu
XX Rozpoczęcie regulacji / zmniejszanie prędkości skokowo o 1 km/h

Rozpocznij sterowanie z bieżącą prędkością

\begin{itemizeArrow}
	\itemArrow Stuknąć SET
\end{itemizeArrow}

Lub:
\begin{itemizeArrow}
	\itemArrow Jeśli nie jest ustawiona żadna prędkość, przesuń dźwignię do położenia sprężynującego XX.
\end{itemizeArrow}

Ustawianie poziomu odstępu

\begin{itemizeArrow}
	\itemArrow Przełącznik XX ustaw w pozycji sprężynowej XX lub XX.
\end{itemizeArrow}

% FIXME: Rys.

Linia pojawi się na wyświetlaczu A , wskazując przesunięcie odległości.
\begin{itemizeArrow}
	\itemArrow Za pomocą przełącznika XX ustaw żądany poziom odległości.
\end{itemizeArrow}

Dla chwilowego przyspieszenia
\begin{itemizeArrow}
	\itemArrow Nacisnąć pedał gazu.
\end{itemizeArrow}

Pojazd przyspiesza, regulacja zostaje chwilowo przerwana. Po zwolnieniu pedału gazu regulacja automatycznie uruchamia się ponownie.

Przerwanie działania tempomatu
\begin{itemizeTriangle}
	\itemTriangle Po naciśnięciu pedału hamulca
	\itemTriangle Po interwencji ESC
	\itemTriangle Po wyłączeniu ASR
\end{itemizeTriangle}

Ponowne uruchomienie regulacji
\begin{itemizeArrow}
	\itemArrow Gdy prędkość jest ustawiona, przesuń dźwignię do położenia sprężynowania XX.
\end{itemizeArrow}

Pojazd jest sterowany z ustawioną prędkością. Ustawiona prędkość jest wyświetlana na wyświetlaczu zestawu wskaźników.

\subsubsection{Ograniczenie funkcjonowania}

Nie ACC w następujących przypadkach.

\begin{itemizeTriangle}
	\itemTriangle Pojazd wykonuje ostry zakręt.
	\itemTriangle Pojazd porusza się po pochyłej drodze lub w koleinach.
	\itemTriangle Pojazd jedzie po wąskim pasie ruchu.
	\itemTriangle Pojazd przejeżdża przez plac budowy.
\end{itemizeTriangle}


\begin{itemizeTriangle}
	\itemTriangle Bądź szczególnie czujny i przygotowany do działania w następujących sytuacjach.
\end{itemizeTriangle}

Regulacja odpowiednio do pojazdu jadącego sąsiednim pasem

Jeśli Twój pojazd porusza się szybciej niż pojazd na bocznym pasie ruchu po stronie kierowcy z prędkością powyżej 80 km/h, Twój pojazd może być kontrolowany zgodnie z tym pojazdem.

Podczas pokonywania zakrętów
% FIXME: Rys.

Podczas pokonywania zakrętów pojazd na sąsiednim pasie może znaleźć się w wykrywanym obszarze.
Twój pojazd będzie regulowany zgodnie z tym pojazdem.

Wąski lub jadący z przesunięciem pojazd
% FIXME: Rys.

ACC wykrywa wąski lub przesunięty pojazd tylko wtedy, gdy znajduje się w obszarze objętym radarem.

Zmiana pasa innych pojazdów
% FIXME: Rys.

Pojazdy, które zmieniają pas ruchu w bliskiej odległości, mogą nie zostać na czas rozpoznane przez ACC.

Stojący pojazd
% FIXME: Rys.

Jeżeli rozpoznany przez ACC samochód skręca albo zjeżdża z pasa, a przed tym samochodem znajduje się pojazd stojący, ACC nie musi zareagować na pojazd
stojący.
Samochody ze szczególnym ładunkiem lub specjalnym nadwoziem

Ładunek lub nadbudówka, które wystają poza pojazd, nie muszą być wykrywane przez ACC.

Ustawienia

Dezaktywacja / aktywacja i ustawianie

W systemie Infotainment w menu XX

\begin{itemizeTriangle}
	\itemTriangle Profil jazdy: – ustawienie przyspieszenia samochodu przy włączonym ACC (w pojazdach z możliwością wyboru trybu jazdy dokonuje się takiego ustawienia \guillemotright strona 135)
	\itemTriangle Ostatnia wybrana odległość – Aktywacja/dezaktywacja ostatniego wybranego poziomu odstępu
	\itemTriangle Odległość: – Ustawianie odstępu od poprzedzających pojazdów
	\itemTriangle Prognozowana trasa - Aktywacja/dezaktywacja tempomatu zgodnie z trasą (np. zmniejszenie prędkości przed zakrętem)
	\itemTriangle Przewidywalna dopuszczalna prędkość - Aktywacja/dezaktywacja regulacji prędkości zgodnie z dopuszczalną prędkością
\end{itemizeTriangle}

Rozwiązywanie problemów

ACC niedostępny

XX zapala się razem z XX

\begin{itemizeArrow}
	\itemArrow Zatrzymaj silnik i zacznij od nowa.
	\itemArrow Jeśli ACC jest nadal niedostępny, sprawdź światła hamowania w pojeździe lub przyczepie.
	\itemArrow Jeśli światła hamowania działają, a ACC jest nadal niedostępny, należy skorzystać z pomocy specjalistycznej stacji obsługi.
\end{itemizeArrow}

\subsection{Asystent utrzymania pasa ruchu (Lane Assist)}

Lane Assist pomaga utrzymać pojazd w pasie. Opiera się na liniach granicznych. Mogą to być linie graniczne, krawędzie drogi lub obiekty, takie jak krawężniki, ewentualnie stożki drogowe.

System przewidziany jest do stosowania na autostradach.

\subsubsection{Sposób działania}

Gdy pojazd zbliża się do rozpoznanej linii odgraniczającej, system wykonuje lekki ruch kierownicą w kierunku przeciwnym do linii odgraniczającej. Ingerencję systemu można przezwyciężyć ręcznie.
Przy zmianie pasa z włączonym kierunkowskazem nie ma interwencji systemu.

Wskaźnik statusu na wyświetlaczu zestawu wskaźników

XX świeci - system został aktywowany i jest gotowy do ingerencji.
XX świeci - system ingeruje

Wyświetlacz
% FIXME: Rys.
Granica pasa ruchu podświetlona po prawej stronie: System ingeruje w momencie zbliżenia się do linii odgraniczającej po prawej stronie.

Ostrzeżenie przez wibracje kierownicy

Drgania kierownicy są wyzwalane w następujących sytuacjach:
\begin{itemizeTriangle}
	\itemTriangle Pojazd przejeżdża przez linię graniczną bez włączonego kierunkowskazu.
	\itemTriangle System nie jest w stanie utrzymać pojazdu na pasie ruchu poprzez ingerencję w kierownicę.
	\begin{itemizeArrow}
		\itemArrow W przypadku wibracji skoryguj ruch kierownicy.
	\end{itemizeArrow}
\end{itemizeTriangle}

Warunki działania
\begin{itemizeTick}
	\itemTick Prędkość jazdy przekracza 60 km/h.
	\itemTick Linie graniczne są wyraźnie widoczne.
\end{itemizeTick}

Ograniczenie funkcjonowania

Działanie systemu może być znacznie ograniczone lub system może nie działać np. w następujących sytuacjach:
\begin{itemizeTriangle}
	\itemTriangle Ograniczenie pola widzenia czujnika przez zabrudzoną szybę, przeszkodę lub poprzedzający pojazd.
	\itemTriangle Jeśli warunki pogodowe są niekorzystne.
	\itemTriangle Pojazd wykonuje ostry zakręt.
	\itemTriangle Pojazd porusza się po pochyłej drodze lub w koleinach.
	\itemTriangle Pojazd jedzie po wąskim pasie ruchu.
	\itemTriangle Pojazd przejeżdża przez plac budowy.
\end{itemizeTriangle}

OSTRZEŻENIE
Niebezpieczeństwo błędnej interwencji układu kierowniczego!
Pewne obiekty lub oznaczenia na drodze mogą zostać błędnie uznane za linie graniczne.

\begin{itemizeTriangle}
	\itemTriangle Zawsze trzymaj ręce na kierownicy i bądź przygotowany, aby przejąć kontrolę nad kierownicą.
\end{itemizeTriangle}

Ustawienia

Automatyczna aktywacja

Lane Assist zawsze włącza się automatycznie po włączeniu zapłonu.

Aktywacja / dezaktywacja w zestawie wskaźników

\begin{itemizeArrow}
	\itemArrow Nacisnąć przycisk XX na kierownicy wielofunkcyjnej.
	\itemArrow Wybierz pozycję menu Lane Assist .
\end{itemizeArrow}

Aktywacja / dezaktywacja w Infotainment

W menu XX

Rozwiązywanie problemów

Komunikat dotyczący niedostępności systemu

\begin{itemizeArrow}
	\itemArrow Wyczyść szybę przednią w obszarze czujnika.
	\itemArrow Zatrzymaj silnik i uruchom ponownie po krótkim czasie.
	\itemArrow Gdyby system nadal nie był dostępny, skorzystać z pomocy specjalistycznej stacji obsługi.
\end{itemizeArrow}

Asystent podróży

Travel Assist pomaga utrzymać pojazd na pasie, jednocześnie kontrolując prędkość pojazdu.

Sposób działania

Travel Assist korzysta z funkcji Lane Assist i ACC.

OSTRZEŻENIE
Niebezpieczeństwo wypadku!
\begin{itemizeTriangle}
	\itemTriangle Zawsze trzymaj ręce na kierownicy i bądź przygotowany, aby przejąć kontrolę nad kierownicą.
	\itemTriangle Należy przestrzegać uwag w opisie funkcjonalnym Lane Assist i ACC.
\end{itemizeTriangle}

Adaptacyjne prowadzenie po torze

Funkcja utrzymuje wybraną pozycję kierowcy w pasie.

Wskaźnik statusu na wyświetlaczu zestawu wskaźników
XX świeci się - system jest włączony, tempomat i śledzenie adaptacyjne są aktywne
XX świeci się - system jest włączony, tempomat jest aktywny
XX świeci się - system jest włączony, aktywne jest prowadzenie adaptacyjne
XX świeci się - kierowca puścił kierownicę i przejął sterowanie
XX zapala się - kierowca zwolnił kierownicę i od razu przejął sterowanie

Warunki działania
\begin{itemizeTick}
	\itemTick ACC jest aktywny.
	\itemTick Prędkość wynosi maks. 210 km/h.
	\itemTick Linie graniczne są wyraźnie widoczne.
\end{itemizeTick}

Ograniczenie funkcjonowania

Systemy wspomagania służą jedynie jako pomoc i nie zwalniają kierowcy z odpowiedzialności za prowadzenie pojazdu.

Działanie systemu może być znacznie ograniczone lub system może nie działać np. w następujących sytuacjach:
\begin{itemizeTriangle}
	\itemTriangle Ograniczenie pola widzenia czujnika przez zabru-
\end{itemizeTriangle}
dzoną szybę, przeszkodę lub poprzedzający pojazd.
\begin{itemizeTriangle}
	\itemTriangle Jeśli warunki pogodowe są niekorzystne.
	\itemTriangle Pojazd wykonuje ostry zakręt.
	\itemTriangle Pojazd porusza się po pochyłej drodze lub w koleinach.
	\itemTriangle Pojazd jedzie po wąskim pasie ruchu.
\end{itemizeTriangle}

Oprócz tych ograniczeń, przestrzegać również ograniczeń ACC \guillemotright strona 147.

OSTRZEŻENIE
Niebezpieczeństwo błędnej interwencji układu kierowniczego!
Pewne obiekty lub oznaczenia na drodze mogą zostać błędnie uznane za linie graniczne.
\begin{itemizeTriangle}
	\itemTriangle Zawsze trzymaj ręce na kierownicy i bądź przygotowany, aby przejąć kontrolę nad kierownicą.
\end{itemizeTriangle}

Ustawienia

Aktywacja
\begin{itemizeArrow}
	\itemArrow Nacisnąć przycisk XX na kierownicy wielofunkcyjnej.
\end{itemizeArrow}
Rozpoczyna się regulacja z bieżącą prędkością i śledzenie adaptacyjne.

Dezaktywacja
\begin{itemizeArrow}
	\itemArrow Nacisnąć pedał hamulca.
\end{itemizeArrow}
Lub:
\begin{itemizeArrow}
	\itemArrow Ustaw dźwignię ACC w pozycji XX >> strona 147
\end{itemizeArrow}

Pozostałe ustawienia są identyczne z ACC i Lane Assist.

\subsection{Asystent zmiany pasa ruchu Side Assist}

System wykorzystuje sygnały optyczne w lusterku zewnętrznym do ostrzegania o pojazdach, które mogą być niebezpieczne podczas zmiany pasów.

\subsubsection{Sposób działania}

System monitoruje obszar obok i za pojazdem.

System ostrzega o zbliżającym się pojeździe za pomocą lampki kontrolnej XX w osłonie lusterka zewnętrznego po stronie pojazdu, gdzie pojazd jest rozpoznawany.

Sytuacje, w których występuje ostrzeżenie systemowe
% FIXME: Rys.

Twój pojazd jest wyprzedzany przez pojazd A . Im szybciej pojazd A zbliża się, tym wcześniej ostrzeżenie jest sygnalizowane przez lampkę ostrzegawczą.

Pojazd B jest wyprzedzany przez twój pojazd z prędkością większą o maks. 10 km/h. Jeśli prędkość podczas wyprzedzania jest wyższa, lampka ostrzegawcza nie ostrzeże.

Lampka kontrolna
XX świeci - za pojazdem znajduje się inny pojazd
XX miga - za pojazdem znajduje się inny pojazd, migające światło jest włączone po tej samej stronie

W samochodach z Lane Assist lampka ostrzegawcza zaświeci się, gdy Twój pojazd przekroczy linię odgraniczenia w kierunku do nadjeżdżającego pojazdu.
System wyzwala wibracje kierownicy.
W tym celu system Lane Assist musi być włączony i musi być rozpoznana linia rozgraniczająca między pojazdami.

Warunki działania
\begin{itemizeTick}
	\itemTick Prędkość jazdy przekracza 15 km/h.
	\itemTick Żadne akcesoria nie są podłączone do gniazda przyczepy.
\end{itemizeTick}

Ograniczenie funkcjonowania

System nie może wykryć szerokości pasa. Dlatego może on np. w podanych niżej przypadkach reagować na pojazd jadący po dalej położonym pasie:
\begin{itemizeTriangle}
	\itemTriangle Pojazd jedzie po drodze o wąskich pasach ruchu lub przy krawędzi pasa.
	\itemTriangle Pojazd przejeżdża przez zakręt.
\end{itemizeTriangle}
System może być reagować na obiekty wzdłuż drogi, np. wysokie barierki ochronne lub ekrany akustyczne.

Ostrzeżenie o zbliżającym się obiekcie nie musi być wysyłane przez system w następujących przypadkach lub ostrzeżenie może być niepoprawnie sformułowane:
\begin{itemizeTriangle}
	\itemTriangle Jeśli warunki pogodowe są niekorzystne.
	\itemTriangle Bardzo szybko zbliżający się pojazd.
	\itemTriangle Pojazd przejeżdża przez ostry zakręt lub rondo.
\end{itemizeTriangle}

Ustawienia

Na wyświetlaczu zestawu wskaźników

W menu Asystenci.

W urządzeniu Infotainment

W menu XX

Rozwiązywanie problemów

Komunikat dotyczący niedostępności systemu
\begin{itemizeArrow}
	\itemArrow Zatrzymaj silnik i zacznij od nowa.
	\itemArrow Gdyby system nadal nie był dostępny, skorzystać z pomocy specjalistycznej stacji obsługi.
\end{itemizeArrow}

\subsection{Rozpoznawanie znaków drogowych}

Rozpoznawanie znaków drogowych wyświetla znaki drogowe na wyświetlaczu zestawu wskaźników.
Sygnalizuje również ograniczenie prędkości.

Sposób działania

Wyświetlane znaki drogowe
\begin{itemizeTriangle}
	\itemTriangle Ograniczenia prędkości
	\itemTriangle Zakazy wyprzedzania
	\itemTriangle Zakazy prowadzenia pojazdów - ruch jednokierunkowy
	\itemTriangle Znaki ostrzegawcze
	\itemTriangle Dodatkowe znaki, np. ograniczenie prędkości przy mokrej nawierzchni
\end{itemizeTriangle}

Ograniczenie

Działanie systemu może być znacznie ograniczone lub system może nie działać np. w następujących sytuacjach:
\begin{itemizeTriangle}
	\itemTriangle Zaburzenia pracy czujnika przez słońce lub światła samochodów jadących z naprzeciwka.
	\itemTriangle Ograniczenie pola widzenia czujnika przez przeszkodę lub poprzedzający pojazd.
	\itemTriangle Jeśli warunki pogodowe są niekorzystne.
	\itemTriangle Pojazd porusza się z dużą prędkością.
	\itemTriangle Znaki drogowe są ukryte, uszkodzone lub niezgodne z normą.
	\itemTriangle Znaki drogowe przymocowane do migających tablic świetlnych.
	\itemTriangle Dokumenty map nawigacyjnych są nieaktualne lub niedostępne.
\end{itemizeTriangle}

Ustawienia

Dezaktywacja / aktywacja i ustawianie

W systemie Infotainment w menu XX

◼ System identyfikacji znaków drogowych
\begin{itemizeTriangle}
	\itemTriangle Wyświetl w zestawie wskaźników – Aktywacja/dezaktywacja dodatkowego wyświetlania znaków drogowych na wyświetlaczu tablicy rozdzielczej
	\itemTriangle Ostrzeżenie o prędkości: – ustawianie ostrzeżenia o przekroczeniu dozwolonej prędkości
	\itemTriangle Ostrzeżenie przy więcej niż - zwiększenie ograniczenia prędkości ostrzeżenia po przekroczeniu dopuszczalnej prędkości
\end{itemizeTriangle}

\subsection{Proaktywna ochrona pasażerów Crew Protect Assist}

System proaktywnej ochrony pasażerów uruchamia działania mające na celu ochronę pasażerów pojazdu w niebezpiecznych sytuacjach.

Sposób działania

Jeśli system oceni sytuację jako krytyczną, uruchamia następujące działania bezpieczeństwa:
\begin{itemizeTriangle}
	\itemTriangle włączają się światła awaryjne,
	\itemTriangle Otwarte okna zostają zamknięte, z wyjątkiem małej szczeliny.
	\itemTriangle Dach przesuwno-uchylny zostanie zamknięty.
	\itemTriangle przednie pasy bezpieczeństwa są naprężone na ciele.
\end{itemizeTriangle}
Reakcja systemu zależy od wybranego trybu jazdy.

Niebezpieczeństwo uderzenia czołowego

Dotyczy pojazdów wyposażonych w system Front Assist.

System uruchamia środki bezpieczeństwa, gdy wykrywa zbliżające się niebezpieczeństwo zderzenia czołowego.

Niebezpieczeństwo uderzenia z tyłu

System uruchamia środki bezpieczeństwa, gdy wykrywa zbliżające się niebezpieczeństwo uderzenia z tyłu.

Zagrożenie bezczynności kierowcy

Dotyczy pojazdów wyposażonych w funkcję Emergency Assist.

System uruchamia środki bezpieczeństwa, gdy wykrywa zbliżające się niebezpieczeństwo, gdy kierowca jest nieaktywny.

Ograniczenie funkcjonowania

Działanie systemu może być znacznie ograniczone lub system może nie działać np. w następujących sytuacjach:
\begin{itemizeTriangle}
	\itemTriangle Podczas dezaktywacji ASR lub aktywacji ESC Sport
	\itemTriangle Podczas cofania
	\itemTriangle W przypadku nieprawidłowego działania ESC, odwracalnego pasa bezpieczeństwa lub systemu poduszek powietrznych
	\itemTriangle W przypadku nieprawidłowego działania lub ograniczenia systemu Front Assist
	\itemTriangle W przypadku nieprawidłowego działania lub ograniczenia asystenta zmiany pasa Side Assist
	\itemTriangle W przypadku nieprawidłowego działania lub ograniczenia funkcji Emergency Assist
\end{itemizeTriangle}

Ustawienia

Automatyczna aktywacja

System jest automatycznie aktywowany zawsze po włączeniu zapłonu.

Rozwiązywanie problemów

Zakłócenie proaktywnego systemu ochrony pasażerów

XX zapala się razem z XX

Komunikat dotyczący niedostępności lub ograniczenia funkcjonalnego
\begin{itemizeArrow}
	\itemArrow Skorzystać z pomocy specjalistycznej stacji obsługi.
\end{itemizeArrow}

\subsection{Asystent wykrywania zmęczenia Driver Alert}

Asystent wykrywania zmęczenia ocenia zachowanie kierowcy podczas prowadzenia pojazdu. Jeśli wykryje zmęczenie u kierowcy, zaleca przerwę. Na wyświetlaczu zestawu wskaźników świeci się symbol XX.

OSTRZEŻENIE
W przypadku tzw. mikrozaśnięć ostrzeżenie systemowe nie jest generowane.

Zresetuj zalecenie pauzy

Zresetowanie zalecenia pauzy następuje w następujących przypadkach:
\begin{itemizeTriangle}
	\itemTriangle Samochód zostanie zatrzymany, a zapłon wyłączony
	\itemTriangle Samochód zostanie zatrzymany, pas bezpieczeństwa odpięty, a drzwi kierowcy otwarte
	\itemTriangle Samochód zostanie zatrzymany na ponad 15 minut
\end{itemizeTriangle}

W niektórych sytuacjach system może błędnie wydać zalecenie pauzy.

Warunki działania
\begin{itemizeTick}
	\itemTick Prędkość jazdy między 60 - 200 km/h.
\end{itemizeTick}

Ustawienia

Aktywacja / dezaktywacja

W systemie Infotainment w menu XX

\subsection{Asystent w sytuacjach awaryjnych Emergency Assist}

Asystent sytuacji awaryjnych (zwany dalej krótko systemem) wykrywa brak aktywności kierowcy spowodowany np. nagłą utratą przytomności. System podejmuje działania zmierzające do możliwie bezpiecznego zatrzymania pojazdu.

Sposób działania

Gdy asystent wykryje brak aktywności kierowcy
\begin{itemizeTriangle}
	\itemTriangle Rozlega się akustyczny sygnał ostrzegawczy i na wyświetlaczu zestawu wskaźników pojawia się komunikat.
	\itemTriangle Jeśli po wielokrotnym ostrzeżeniu kierowca nie przejmie sterowania, światła awaryjne zostaną włączone i pojazd zostanie automatycznie zahamowany.
	\itemTriangle Po zatrzymaniu pojazdu włącza się hamulec postojowy. W zależności od wyposażenia można zainicjować połączenie alarmowe.
\end{itemizeTriangle}

Przerwanie automatycznego hamowania
\begin{itemizeTriangle}
	\itemTriangle Wciśnięto pedał hamulca lub sprzęgła.
	\itemTriangle Przez ruch kierownicą
\end{itemizeTriangle}

Warunki działania
\begin{itemizeTick}
	\itemTick Asystent w sytuacjach awaryjnych jest włączony.
	\itemTick System Lane Assist jest aktywny i linie ograniczające po obu stronach pasa są rozpoznawane.
\end{itemizeTick}

Ustawienia

Automatyczna aktywacja

Asystent w sytuacjach awaryjnych włącza się automatycznie po włączeniu zapłonu.

Aktywacja / dezaktywacja

W systemie Infotainment w menu XX

\section{Systemy asystentów parkowania}

\subsection{System wspomagania parkowania Park Pilot}

Jeśli przeszkoda zostanie wykryta, urządzenie Infotainment wyświetli komunikat graficzny i rozlegnie się sygnał dźwiękowy.

W miarę zmniejszania się odległości od przeszkody skraca się przerwa pomiędzy kolejnymi dźwiękami.
W odległości mniejszej niż 30 cm od przeszkody rozlega się ciągły dźwięk.

Jeżeli kierowca nie reaguje na ostrzeżenie, system uruchamia automatyczne hamowanie awaryjne przy prędkościach poniżej 8 km/h, aby zmniejszyć uderzenie.

Automatyczne hamowanie można włączyć i wyłączyć w urządzeniu Infotainment.

Wskaźnik

Rejestrowane obszary różnią się w zależności od sprzętu.
% FIXME: Rys.

XX Włączanie / wyłączanie sygnałów akustycznych
XX Jednorazowa dezaktywacja / aktywacja hamowania automatycznego
XX Asystent parkowania
XX Dostosowanie niektórych systemów asystenta parkowania
XX Usterka systemu
XX Przejście do wskazania kamery do jazdy wstecz
XX Przeszkoda w odległości mniejszej niż 30 cm
XX Przeszkoda w odległości ponad 30 cm
XX Przeszkoda na zewnątrz podjazdu

Warunki działania
\begin{itemizeTick}
	\itemTick Prędkość jazdy jest mniejsza niż 15 km/h.
	\itemTick Żadne akcesoria nie są podłączone do gniazda przyczepy.
\end{itemizeTick}

Włączanie / wyłączanie

Włączanie
\begin{itemizeArrow}
	\itemArrow Włączyć bieg wsteczny.
\end{itemizeArrow}
Lub:
\begin{itemizeArrow}
	\itemArrow Nacisnąć przycisk XX
\end{itemizeArrow}

Automatyczne włączanie podczas jazdy do przodu

Zbliżając się do przeszkody z prędkością mniejszą niż 10 km/h.

Automatyczne włączenie nie uruchamia automatycznego hamowania.

Wyłączanie

\begin{itemizeArrow}
	\itemArrow Wyłączyć bieg wsteczny.
\end{itemizeArrow}
Lub:
\begin{itemizeArrow}
	\itemArrow Nacisnąć przycisk XX.
\end{itemizeArrow}

Automatyczne wyłączanie

Prędkość jazdy przekracza 15 km/h.

Ograniczenie funkcjonowania

Ostrzeżenie o zbliżającej się przeszkodzie nie musi być wysyłane przez system w następujących przypadkach lub ostrzeżenie może być niepoprawnie sformułowane.
\begin{itemizeTriangle}
	\itemTriangle Jeśli warunki pogodowe są niekorzystne.
	\itemTriangle Wykryte przeszkody są w ruchu.
	\itemTriangle Sygnały z czujników nie odbijają się od powierzchni przeszkód.
	\itemTriangle Jest to mniejsza przeszkoda, np. kamień lub słup.
\end{itemizeTriangle}

Ustawienia

Menu ustawień systemu wyświetla się w następujący sposób:
\begin{itemizeArrow}
	\itemArrow Nacisnąć przycisk XX pod Infotainment; Stuknij XX na ekranie systemu Infotainment.
\end{itemizeArrow}
Lub:
\begin{itemizeArrow}
	\itemArrow Nacisnąć przycisk XX pod Infotainment; Stuknąć XX na ekranie systemu Infotainment.
\end{itemizeArrow}
Lub:
\begin{itemizeArrow}
	\itemArrow Włączyć bieg wsteczny > na ekranie Infotainment dotknąć przycisku funkcyjnego XX
\end{itemizeArrow}

Rozwiązywanie problemów

Po aktywacji systemu przez około 3 s rozlega się sygnał dźwiękowy (w pobliżu pojazdu nie znajduje się żadna przeszkoda).
\begin{itemizeArrow}
	\itemArrow Skorzystać z pomocy specjalistycznej stacji obsługi.
\end{itemizeArrow}

Po włączeniu nie wszystkie zeskanowane obszarysą wyświetlane na ekranie systemu Infotainment 
\begin{itemizeArrow}
	\itemArrow Przesuń pojazd kilka metrów w przód lub w tył.
	\itemArrow Jeśli wykrywane obszary nadal nie będą wyświetlane, skorzystać z pomocy specjalistycznej stacji obsługi.
\end{itemizeArrow}

\subsection{Kamera cofania}

Sposób działania

Podczas cofania obszar za samochodem z liniami orientacyjnymi jest wyświetlany w systemie Infotainment.

Kamera wyposażona jest w instalację do czyszczenia . Mycie odbywa się automatycznie wraz z myciem tylnej szyby lub za pomocą powierzchni funkcyjnej XX na ekranie Infotainment.

Linie orientacyjne i przyciski funkcyjne

Linie orientacji
%FIXME: Rys.

A Odległość około 40 cm
B Odległość około 100 cm
C Odległość około 200 cm
D Linie wskazują tor, który jest kontrolowany przy aktualnym kącie skrętu. Odstęp między bocznymi liniami orientacyjnymi odpowiada szerokości samochodu wraz z lusterkami zewnętrznymi.

Przyciski funkcyjne
%FIXME: Rys.

A Tryb parkowania poprzecznego
B Tryb dojazdu do przyczepy
C Tryb obserwacji obszaru za pojazdem (szeroki obraz)
D wyłączanie / włączanie sygnału dźwiękowego systemu czujników parkowania
E Ustawianie pomocy przy parkowaniu
F Ustawienia jasności, kontrastu i kolorów wyświetlanego obrazu
Funkcjonalna powierzchnia do czyszczenia kamery cofania
G Wspomaganie parkowania - widok

Aktywacja pełnego ekranu pomocy parkowania odbywa się poprzez dotknięcie widoku.

Warunki działania
\begin{itemizeTick}
	\itemTick Zapłon włączony
	\itemTick Prędkość jazdy poniżej 15 km/h
\end{itemizeTick}

Obsługa

Włączanie systemu
\begin{itemizeArrow}
	\itemArrow Włączyć bieg wsteczny.
\end{itemizeArrow}
Lub:
\begin{itemizeArrow}
	\itemArrow Nacisnąć przycisk XX pod Infotainment; Stuknij XX na ekranie systemu Infotainment.
\end{itemizeArrow}

Tryb parkowania poprzecznego
% FIXME: Rys.
\begin{itemizeArrow}
	\itemArrow Zatrzymaj pojazd przed odpowiednim miejscem parkingowym.
	\itemArrow Włącz cofanie tak, aby żółte linie prowadziły na miejsce parkingowe.
	\itemArrow Zatrzymaj się najpóźniej, gdy czerwona linia dotknie tylnej granicy (np. krawężnika).
\end{itemizeArrow}

Tryb monitorowania za pojazdem

W tym trybie obszar za pojazdem jest wyświetlany na ekranie w trybie ekranowym.

Wyłączanie systemu

\begin{itemizeArrow}
	\itemArrow Nacisnąć przycisk XX pod Infotainment.
\end{itemizeArrow}
Lub:
\begin{itemizeArrow}
	\itemArrow Nacisnąć przycisk funkcyjny XX na ekranie urządzenia Infotainment.
\end{itemizeArrow}
Lub:
\begin{itemizeArrow}
	\itemArrow Wybrać tryb \gearP.
\end{itemizeArrow}

Wyłączanie automatyczne

Automatyczne wyłączanie systemu występuje podczas jazdy do przodu z prędkością powyżej 15 km/h.

Ograniczenie

Obraz rejestrowany przez kamerę w porównaniu z obrazem widzianym ludzkim okiem jest zniekształcony. Używaj wskazania tylko warunkowo do oszacowania odległości.
Niektóre elementy mogą być niewystarczająco wyświetlane na ekranie. Na przykład wąskie kolumny, ogrodzenia z siatki drucianej, siatki lub wyboje drogowe.

\subsection{System obserwacji otoczenia (Top View)}

Widok otoczenia wspomaga kierowcę w parkowaniu i manewrowaniu przez wyświetlanie otoczenia pojazdu.

\subsection{Przegląd}

Zdjęcie całego samochodu
% FIXME: Rys.

Wybór obrazu z kamery

Wybór danego widoku obrazu kamery następuje poprzez dotknięcie palcem ekranu w obszarze obok, przed lub za sylwetką samochodu. Wybrany obszar jest zaznaczony na ekranie przez żółte obramowanie.

Widok z kamery z tyłu
% FIXME: Rys.
A Tryb parkowania poprzecznego
B Tryb dojazdu do przyczepy
C Tryb obserwacji obszaru za pojazdem (szeroki obraz)
D wyłączanie / włączanie sygnału dźwiękowego systemu czujników parkowania
E Ustawianie pomocy przy parkowaniu
F Ustawienia jasności, kontrastu i kolorów wyświetlanego obrazu

Funkcjonalna powierzchnia do czyszczenia kamery cofania

Widok kamery z przodu
% FIXME: Rys.
A Tryb parkowania poprzecznego
B Tryb obserwacji obszaru za pojazdem (szeroki obraz)
C wyłączanie / włączanie sygnału dźwiękowego systemu czujników parkowania
D Ustawianie pomocy przy parkowaniu
E Ustawienia jasności, kontrastu i kolorów wyświetlanego obrazu
Funkcjonalna powierzchnia do czyszczenia kamery cofania

Widok z kamery bocznej
% FIXME: Rys.
Lewa i prawa strona.
Żółte linie wyświetlane są w momencie, gdy odstęp do pojazdu wynosi ok. 40 cm.
Warunki działania
\begin{itemizeTick}
	\itemTick Zapłon włączony
	\itemTick Prędkość jazdy poniżej 15 km/h
\end{itemizeTick}

\subsubsection{Obsługa}

Włączanie

Sposób działania
\begin{itemizeArrow}
	\itemArrow Włączyć bieg wsteczny.
	\itemArrow Nacisnąć przycisk XX pod Infotainment; Stuknij XX na ekranie systemu Infotainment.
\end{itemizeArrow}

Wybór obrazu z kamery
\begin{itemizeArrow}
	\itemArrow W systemie Infotainment dotknij obszaru obok sylwetki pojazdu, przed lub za nią. Wybrany obszar jest zaznaczony na ekranie przez żółte obramowanie.
\end{itemizeArrow}

Wyłączanie

\begin{itemizeArrow}
	\itemArrow Nacisnąć przycisk XX pod Infotainment.
\end{itemizeArrow}
Lub:
\begin{itemizeArrow}
	\itemArrow Nacisnąć przycisk funkcyjny XX na ekranie urządzenia Infotainment.
\end{itemizeArrow}
Lub:
\begin{itemizeArrow}
	\itemArrow Wybrać tryb \gearP.
\end{itemizeArrow}

Wyłączanie automatyczne

Automatyczne wyłączanie systemu występuje podczas jazdy do przodu z prędkością powyżej 15 km/h.

Ograniczenie

Obraz rejestrowany przez kamerę w porównaniu z obrazem widzianym ludzkim okiem jest zniekształcony. Używaj wskazania tylko warunkowo do oszacowania odległości.
Niektóre przedmioty (np. cienkie słupki, płoty z siatki, kraty czy nierówności nawierzchni drogi) mogą nie być wystarczająco widoczne na ekranie ze względu
na jego rozdzielczość.
Obiekty, które znajdują się w bezpośredniej bliskości pojazdu, mogą leżeć poza obszarem obejmowanym przez kamerę i dlatego mogą nie być wyświetlane na ekranie.

Rozwiązywanie problemów

Komunikat dotyczący niedostępności systemu
\begin{itemizeArrow}
	\itemArrow Skorzystać z pomocy specjalistycznej stacji obsługi.
\end{itemizeArrow}

\subsection{Asystent wyjazdu}

Sposób działania

% FIXME: Rys.

Asystent wyjazdu z parkingu ostrzega podczas cofania o nadjeżdżających pojazdach.

Jeżeli kierowca nie zareaguje na ostrzeżenie, pojazd hamuje automatycznie przy prędkości poniżej 10 km/h.

Pojazd ze wspomaganiem parkowania

W przypadku wykrycia zbliżającego się pojazdu w obszarze za pojazdem, w urządzeniu Infotainment pojawia się notatka graficzna. Jednocześnie słychać ciągły dźwięk.

% FIXME: Rys.

XX Pojazd w obszarze kolizji – niebezpieczeństwo kolizji!
XX Zbliżający się pojazd

Warunki działania
\begin{itemizeTick}
	\itemTick Żadne akcesoria nie są podłączone do gniazda przyczepy.
\end{itemizeTick}

Ograniczenie funkcjonowania
Funkcja asystenta wyjazdu z parkingu może być ograniczona w niesprzyjających warunkach pogodowych.

Ustawienia

Aby wyświetlić menu aktywacji/dezaktywacji systemu, wykonaj następujące czynności.
\begin{itemizeArrow}
	\itemArrow Nacisnąć przycisk XX pod Infotainment Stuknij XX na ekranie systemu Infotainment.
	\itemArrow Wybrać pozycję menu Asystent wyjazdu z parkingu.
\end{itemizeArrow}
Lub:
\begin{itemizeArrow}
	\itemArrow Nacisnąć przycisk XX na kierownicy wielofunkcyjnej.
	\itemArrow Wybrać pozycję menu Asystent wyjazdu z parkingu.
\end{itemizeArrow}
Lub:
\begin{itemizeArrow}
	\itemArrow Włączyć bieg wsteczny na ekranie Infotainment dotknąć przycisku funkcyjnego XX
	\itemArrow Wybrać pozycję menu Asystent wyjazdu z parkingu.
\end{itemizeArrow}

Rozwiązywanie problemów

Komunikat dotyczący niedostępności systemu
\begin{itemizeArrow}
	\itemArrow Zatrzymaj silnik i zacznij od nowa.
	\itemArrow Gdyby system nadal nie był dostępny, skorzystać z pomocy specjalistycznej stacji obsługi.
\end{itemizeArrow}

\subsection{Lampka ostrzegawcza wyjścia}

Asystent ostrzega podczas otwierania drzwi przed zbliżającymi się obiektami, aby uniknąć możliwej kolizji.

Sposób działania

Side Assist monitoruje obszar obok i za pojazdem do odległości około 35 m.

Sytuacja, w której pojawia się ostrzeżenie systemowe
% FIXME: Rys.

Ostrzeżenie w przypadku zbliżającego się zderzenia


\begin{itemizeTriangle}
	\itemTriangle Lampka kontrolna XX w osłonie lusterka zewnętrznego po stronie, w której wykryto obiekt, miga, a następnie świeci.
	\itemTriangle Rozlega się sygnał ostrzegawczy.
\end{itemizeTriangle}

Ostrzeżenie, jeśli nie ma bezpośredniego zagrożenia kolizją

\begin{itemizeTriangle}
	\itemTriangle Lampka kontrolna XX w pokrywie lusterka zewnętrznego po stronie, w której wykryto pojazd, zapala się.
\end{itemizeTriangle}

Warunki
\begin{itemizeTick}
	\itemTick Pojazd się nie porusza.
	\itemTick Zapłon jest włączony.
	\itemTick Prędkość zbliżającego się obiektu jest wyższa niż 2 km/h.
	\itemTick Żadne akcesoria nie są podłączone do gniazda przyczepy.
\end{itemizeTick}
Po wyłączeniu zapłonu asystent pozostaje aktywny przez około 3 minuty.

Ograniczenie

Ostrzeżenie o zbliżającym się obiekcie nie musi być wysyłane przez system w następujących przypadkach lub ostrzeżenie może być niepoprawnie wyprowadzone:
\begin{itemizeTriangle}
	\itemTriangle Jeśli warunki pogodowe są niekorzystne.
\end{itemizeTriangle}

Cel stosowania
\begin{itemizeTriangle}
	\itemTriangle Pole widzenia czujników jest ograniczone przeszkodą.
	\itemTriangle Bardzo szybko zbliżający się pojazd.
\end{itemizeTriangle}

Ustawienia

Aby wyświetlić menu aktywacji / dezaktywacji systemu, wykonaj następujące czynności:
\begin{itemizeArrow}
	\itemArrow Nacisnąć przycisk XX pod Infotainment Stuknij XX na ekranie systemu Infotainment.
	\itemArrow Wybrać pozycję menu Lampka ostrzegawcza wyjścia
\end{itemizeArrow}
Lub:
\begin{itemizeArrow}
	\itemArrow Włączyć bieg wsteczny na ekranie Infotainment dotknąć przycisku funkcyjnego XX 
	\itemArrow Wybrać pozycję menu Lampka ostrzegawcza wyjścia
\end{itemizeArrow}

Rozwiązywanie problemów

Komunikat dotyczący niedostępności systemu
\begin{itemizeArrow}
	\itemArrow Zatrzymaj silnik i zacznij od nowa.
	\itemArrow Gdyby system nadal nie był dostępny, skorzystać z pomocy specjalistycznej stacji obsługi.
\end{itemizeArrow}

\subsection{Asystent parkowania}

Asystent parkowania (zwany dalej krótko systemem) wspomaga kierowcę podczas parkowania w odpowiednich miejscach parkowania równoległego i prostopadłego oraz podczas wyjeżdżania z miejsc parkowania równoległego.

Sposób działania

System wyszukuje miejsce parkingowe i przejmuje sterowanie tylko podczas parkowania lub wyjazdu z parkingu. Kierowca obsługuje pedały i wybierak automatycznej skrzyni biegów.

System wyświetla informacje i wskazówki na wyświetlaczu zestawu wskaźników oraz systemie Infotainment.

Jeżeli podczas manewru parkowania system rozpozna niebezpieczeństwo zderzenia, wówczas, aby uniknąć zderzenia, następuje automatyczne hamowanie awaryjne.

Warunki działania
\begin{itemizeTick}
	\itemTick Prędkość jazdy jest mniejsza niż 7 km/h.
	\itemTick ASR jest włączony i nie ma interwencji.
	\itemTick Brak ingerencji kierowcy w automatyczne kierowanie.
	\itemTick Żadne akcesoria nie są podłączone do gniazda przyczepy.
\end{itemizeTick}

Obsługa

Włączanie
\begin{itemizeArrow}
	\itemArrow Nacisnąć przycisk XX pod Infotainment Pomoc przy parkowaniu dotknąć ekranu systemu Infotainment. Wybierz stronę pasa dla procedury parkowania System wyszukuje automatycznie miejsce do zaparkowania po stronie pasażera.
	\itemArrow Włącz kierunkowskaz po tej stronie, po której chcesz znaleźć miejsce parkingowe.
\end{itemizeArrow}

Przebieg wyszukiwania miejsca do zaparkowania
\begin{itemizeArrow}
	\itemArrow Przejedź obok kilku zaparkowanych pojazdów w odległości 0,5 - 1,5 m.
	\itemArrow Aby wyszukać miejsce parkingowe poprzecznie do pasa jezdni, jedź wolniej niż 20 km/h.
	\itemArrow Aby wyszukać miejsce parkingowe wzdłuż drogi, jedź wolniej niż 40 km h.
\end{itemizeArrow}

Pokaż wybrane miejsce parkingowe
% FIXME: Rys.

A Parkowanie tyłem poprzecznie
B Parkowanie do tyłu wzdłużnie
C Parkowanie przodem poprzecznie
XX Symbol dotykowy - zmiana trybu parkowania

Wyłączanie
\begin{itemizeArrow}
	\itemArrow Nacisnąć przycisk XX pod Infotainment.
\end{itemizeArrow}
Lub:
\begin{itemizeArrow}
	\itemArrow Symbol XX na ekranie infotainment.
\end{itemizeArrow}

Przed rozpoczęciem parkowania

\begin{itemizeArrow}
	\itemArrow Jeśli zostanie znalezione odpowiednie miejsce do parkowania, zatrzymaj się i jedź w tył lub do przodu zgodnie ze strzałką na wyświetlaczu.
	\itemArrow Po wyświetleniu komunikatu o zaprzestaniu kierowania puść kierownicę. Kierowanie zostaje przejęte przez system.
\end{itemizeArrow}

Przebieg parkowania

OSTROŻNIE
Niebezpieczeństwo zranienia!
\begin{itemizeTriangle}
	\itemTriangle Nie należy sięgać między ramiona koła kierownicy podczas parkowania.
	\itemTriangle Używaj tylko pedałów i wybieraka automatycznej skrzyni biegów.
	\begin{itemizeArrow}
		\itemArrow Obserwować otoczenie pojazdu i jechać do tyłu lub do przodu zgodnie ze strzałką na wyświetlaczu.
		\itemArrow Gdy na wyświetlaczu zestawu wskaźników wyświetli się symbol XX i pojawi się sygnał dźwiękowy, należy zatrzymać pojazd. Kierownica zostanie odpowiednio obrócona. Symbol XX świeci.
		\itemArrow Jechać dalej do tyłu lub do przodu zgodnie ze strzałką na wyświetlaczu.
	\end{itemizeArrow}
\end{itemizeTriangle}
Zaraz po zakończeniu parkowania wyświetlany jest odpowiedni komunikat i rozlega się sygnał akustyczny.

Prowadzenie do celu można wyłączyć na jeden z poniższych sposobów:

\begin{itemizeTriangle}
	\itemTriangle Przez naciśnięcie przycisku XX na urządzeniu Infotainment
	\itemTriangle Dotykając ikony XX na ekranie Infotainment
	\itemTriangle Poprzez interwencję kierownicą
\end{itemizeTriangle}

Aby zakończyć procedurę parkowania za pomocą asystenta parkowania

Parkowanie przodem

\begin{itemizeArrow}
	\itemArrow Częściowo zaparkować w odpowiednim poprzecznym miejscu parkingowym do przodu.
	\itemArrow Nacisnąć przycisk XX pod Infotainment Pomoc przy parkowaniu dotknąć ekranu systemu Infotainment.
	\itemArrow Postępuj zgodnie z instrukcjami wyświetlanymi na ekranie systemu Infotainment.
\end{itemizeArrow}

Parkowanie do tyłu

\begin{itemizeArrow}
	\itemArrow Zatrzymaj pojazd za odpowiednim równoległym lub prostopadłym miejscem parkingowym.
	\itemArrow Nacisnąć przycisk XX pod Infotainment Pomoc przy parkowaniu dotknąć ekranu systemu Infotainment.
	\itemArrow Postępuj zgodnie z instrukcjami wyświetlanymi na ekranie systemu Infotainment.
\end{itemizeArrow}

Wyjazd z miejsca parkingowego wzdłuż pasa jezdni

\begin{itemizeArrow}
	\itemArrow Nacisnąć przycisk XX pod Infotainment Pomoc przy parkowaniu dotknąć ekranu systemu Infotainment. \\ Po aktywacji systemu w przycisku świeci się symbol XX.
	\itemArrow Postępuj zgodnie z instrukcjami wyświetlanymi na wyświetlaczu urządzenia diagnostycznego.
\end{itemizeArrow}

Automatyczna redukcja prędkości

Jeśli podczas postoju zostanie przekroczona prędkość 7 km/h, system zmniejszy prędkość.
Przy drugim przekroczeniu prędkości 7 km/\ h proces parkowania zostaje zakończony.

Ograniczenie funkcjonowania

Jeżeli miejsce parkowania jest za małe, manewr wyjeżdżania z miejsca parkingowego przy pomocy systemu nie jest możliwy. Na wyświetlaczu tablicy rozdzielczej pojawia się odpowiedni komunikat.

Nie używaj asystenta parkowania w następujących przypadkach:
\begin{itemizeTriangle}
	\itemTriangle Pojazd porusza się po nieutwardzonych lub śliskich nawierzchniach.
	\itemTriangle Jeśli zamontowane są łańcuchy śnieżne lub koło zapasowe.
	\itemTriangle Jeśli system oferuje nieodpowiednie miejsce parkingowe do parkowania.
\end{itemizeTriangle}

Rozwiązywanie problemów

Komunikat dotyczący niedostępności systemu
\begin{itemizeArrow}
	\itemArrow Zatrzymaj silnik i zacznij od nowa.
	\itemArrow Gdyby system nadal nie był dostępny, skorzystać z pomocy specjalistycznej stacji obsługi.
\end{itemizeArrow}

Nieprawidłowa wynikowa pozycja pojazdu na parkingu

Prawidłowa procedura parkowania zależy od wielkości kół. Jeśli zamontowano inne koła zatwierdzone przez ŠKODA AUTO, system powinien zostać ponownie ustawiony przez specjalistyczną stację obsługi.

\subsection{Olej silnikowy}


Sprawdzanie i uzupełnianie poziomu

Warunki sprawdzania
\begin{itemizeTick}
	\itemTick Pojazd stoi na równym terenie
	\itemTick Unieruchomiony silnik jest ciepły
\end{itemizeTick}

Sprawdź poziom za pomocą prętowego wskaźnika poziomu oleju
\begin{itemizeArrow}
	\itemArrow Odczekać kilka minut, aż olej silnikowy spłynie do miski olejowej.
	\itemArrow Wyjąć bagnet do pomiaru poziomu oleju i wytrzeć czystą ściereczką.
	\itemArrow Wsunąć bagnet do oporu i ponownie wyjąć.
	\itemArrow Odczytać poziom oleju i ponownie wsunąć bagnet.
\end{itemizeArrow}

% FIXME: Rys.

Poziom oleju musi znajdować się w zaznaczonym zakresie.

INFORMACJA
Niebezpieczeństwo uszkodzenia silnika i układu wydechowego!
\begin{itemizeTriangle}
	\itemTriangle Poziom oleju nie może znajdować się poza oznaczonym zakresem. Jeżeli uzupełnienie oleju silnikowego nie jest możliwe lub poziom oleju znajduje się poza zaznaczonym zakresem, nie wolno kontynuować jazdy Wyłączyć silnik i zwrócić się o pomoc do specjalistycznej stacji obsługi.
	\itemTriangle Nie używaj żadnych dodatków do oleju.
\end{itemizeTriangle}

Zużycie

Silnik zużywa, w zależności od sposobu jazdy i warunków eksploatacji, pewną ilość oleju (do 0,5~l/1000 km). Przez pierwsze 5000 km zużycie oleju może być
nawet większe.
Dolewanie
\begin{itemizeArrow}
	\itemArrow Odkręcić korek wlewu oleju do silnika.
	\itemArrow Dolewać olej zgodny ze specyfikacją w porcjach półlitrowych.
	\itemArrow Sprawdzić poziom oleju.
	\itemArrow Dokręcić korek wlewu oleju silnikowego.
\end{itemizeArrow}

Specyfikacja
Zapytaj o odpowiednią specyfikację oleju silnikowego do Twojego pojazdu w specjalistycznej stacji obsługi.
Jeśli nie ma oleju o prawidłowej specyfikacji, można do następnej wymiany oleju zastosować maks. 0,5 l oleju o następującej specyfikacji:
\begin{itemizeTriangle}
	\itemTriangle VW 504 00, VW 508 00, ACEA C3, ACEA C5
\end{itemizeTriangle}

Wymienić

Zlecić wymianę oleju w specjalistycznej stacji obsługi.

Rozwiązywanie problemów

Zbyt niskie ciśnienie oleju silnikowego

XX miga, świeci jednocześnie XX

\begin{itemizeArrow}
	\itemArrow Sprawdź poziom oleju w silniku.
\end{itemizeArrow}

Poziom oleju jest OK, lampka ostrzegawcza nadal miga:
\begin{itemizeArrow}
	\itemArrow Wyłączyć silnik i zwrócić się o pomoc do specjalistycznej stacji obsługi.
\end{itemizeArrow}

Poziom oleju silnikowego za niski

XX zapala się razem z XX

Komunikat dotyczący koniecznego uzupełnienia oleju silnikowego
\begin{itemizeArrow}
	\itemArrow Sprawdź poziom oleju w silniku, w razie potrzeby uzupełnij olej.
\end{itemizeArrow}

Poziom oleju silnikowego za wysoki

XX zapala się razem z XX

Komunikat dotyczący koniecznej redukcji poziomu oleju silnikowego
\begin{itemizeArrow}
	\itemArrow Sprawdź poziom oleju w silniku.
\end{itemizeArrow}

Poziom oleju jest za wysoki:

XX zapala się razem z XX

\begin{itemizeArrow}
	\itemArrow Kontynuuj jazdę z odpowiednią ostrożnością.
	\itemArrow Skorzystać z pomocy specjalistycznej stacji obsługi.
\end{itemizeArrow}

Uszkodzony czujnik poziomu oleju silnika

XX zapala się razem z XX

Komunikat dotyczący czujnika oleju silnikowego
\begin{itemizeArrow}
	\itemArrow Kontynuuj jazdę z odpowiednią ostrożnością.
	\itemArrow Skorzystać z pomocy specjalistycznej stacji obsługi.
\end{itemizeArrow}


Płyn chłodzący

Sprawdzanie i uzupełnianie poziomu

Zbiornik płynu chłodzącego

Pojazd ma dwa obwody chłodzące z dwoma zbiornikami wyrównawczymi płynu chłodzącego w komorze silnika.
\begin{itemizeTriangle}
	\itemTriangle Obwód chłodzenia układu wysokiego napięcia z mniejszym zbiornikiem wyrównawczym płynu chłodzącego
	\itemTriangle Obwód chłodzenia silnika spalinowego z większym zbiornikiem wyrównawczym płynu chłodzącego
\end{itemizeTriangle}

Warunki sprawdzania
\begin{itemizeTick}
	\itemTick Pojazd stoi na równym terenie
	\itemTick Silnik jest wyłączony i schłodzony
\end{itemizeTick}

Sprawdź poziom w zbiorniku - układ wysokiego napięcia
Poziom napełnienia musi znajdować się w zaznaczonym zakresie.
\begin{itemizeArrow}
	\itemArrow Jeżeli poziom płynu chłodzącego leży poniżej zaznaczenia , należy uzupełnić płyn chłodzący.
	\itemArrow Skorzystać z pomocy specjalistycznej stacji obsługi.
\end{itemizeArrow}

OSTRZEŻENIE
Niebezpieczeństwo zwarcia i zapłonu akumulatora wysokonapięciowego!
\begin{itemizeTriangle}
	\itemTriangle Nigdy nie otwieraj zbiornika obwodu chłodzenia układu wysokiego napięcia, nie uzupełniaj płynu chłodzącego.
\end{itemizeTriangle}

Sprawdź poziom w zbiorniku - silnik spalinowy

% FIXME: Rys.

Poziom napełnienia musi znajdować się w zaznaczonym zakresie.
\begin{itemizeArrow}
	\itemArrow Jeżeli poziom płynu chłodzącego leży poniżej zaznaczenia XX, należy uzupełnić płyn chłodzący.
\end{itemizeArrow}

INFORMACJA
Niebezpieczeństwo uszkodzenia komory silnikowej!
\begin{itemizeTriangle}
	\itemTriangle Nie uzupełniaj płynu chłodzącego powyżej zaznaczonego obszaru. Po podgrzaniu chłodziwo mogłoby zostać wypchnięte z układu chłodzenia.
\end{itemizeTriangle}

INFORMACJA
W zbiorniku płynu chłodzącego musi zawsze znajdować się niewielka ilość płynu chłodzącego.
\begin{itemizeTriangle}
	\itemTriangle Jeżeli zbiornik wyrównawczy jest pusty, nie dolewać płynu chłodzącego.
	\itemTriangle Nie wolno kontynuować jazdy! Skorzystać z pomocy specjalistycznej stacji obsługi.
\end{itemizeTriangle}

Dolewanie

OSTROŻNIE
Niebezpieczeństwo poparzenia!
Układ chłodzenia znajduje się pod ciśnieniem.
\begin{itemizeTriangle}
	\itemTriangle Nigdy nie odkręcać korka zbiornika wyrównawczego układu chłodzenia, dopóki silnik nie ostygnie -- zaczekać, aż silnik ostygnie.
	\begin{itemizeArrow}
		\itemArrow Na korek zbiornika wyrównawczego środka chłodzącego połóż ścierkę i ostrożnie odkręć korek.
		\itemArrow Dolewać nowego płynu chłodzącego o prawidłowej specyfikacji.
		\itemArrow Zakręcić korek zbiornika, aż zaskoczy.
	\end{itemizeArrow}
\end{itemizeTriangle}

Specyfikacja
Aby uzupełnić dodatek do płynu chłodzącego, użyć G12evo (TL 774 L).

Rozwiązywanie problemów

Zbyt niski poziom płynu chłodzącego

XX zapala się razem z XX

Komunikat dotyczący koniecznego testu płynu chłodzącego
\begin{itemizeArrow}
	\itemArrow Sprawdzić poziom płynu chłodzącego.
\end{itemizeArrow}
Jeśli poziom płynu chłodzącego jest prawidłowy:
\begin{itemizeArrow}
	\itemArrow Sprawdzić bezpiecznik wentylatora chłodnicy i w razie potrzeby wymienić \guillemotright strona 175, Wymiana bezpiecznika.
\end{itemizeArrow}
Jeśli bezpiecznik jest sprawny, ale lampka kontrolna zapala się ponownie:
\begin{itemizeArrow}
	\itemArrow Nie wolno kontynuować jazdy! Skorzystać z pomocy specjalistycznej stacji obsługi.
\end{itemizeArrow}

Temperatura płynu chłodzącego za wysoka

XX zapala się razem z XX

Komunikat dotyczący przegrzania silnika
\begin{itemizeArrow}
	\itemArrow Nie wolno kontynuować jazdy.
	\itemArrow Wyłączyć silnik i odczekać, aż się schłodzi.
	\itemArrow Kontynuuj jazdę po zgaśnięciu kontrolki.
\end{itemizeArrow}

Usterka silnika

XX zapala się razem z XX

\begin{itemizeArrow}
	\itemArrow Nie wolno kontynuować jazdy!
	\itemArrow Wyłączyć silnik i zwrócić się o pomoc do specjalistycznej stacji obsługi.
\end{itemizeArrow}

Zakłócenie w obwodzie chłodzenia układu wysokiego napięcia

XX zapala się razem z XX
\begin{itemizeArrow}
	\itemArrow Nie wolno kontynuować jazdy!
	\itemArrow Wyłączyć silnik i zwrócić się o pomoc do specjalistycznej stacji obsługi.
\end{itemizeArrow}

Nie ma płynu chłodzącego o właściwej specyfikacji
\begin{itemizeArrow}
	\itemArrow Dodaj destylowaną lub zdemineralizowaną wodę.
	\itemArrow Popraw prawidłowy stosunek mieszania środka chłodzącego przez specjalistyczną stację obsługi tak szybko, jak to możliwe.
\end{itemizeArrow}

Zastosowano do napełnienia inną ciecz niż woda destylowana lub zdemineralizowana:
\begin{itemizeArrow}
	\itemArrow Zlecić sprawdzenie pojazdu w specjalistycznej stacji obsługi.
\end{itemizeArrow}

Napełnienie wystarczającą ilością płynu chłodzącego nie jest możliwe

\begin{itemizeArrow}
	\itemArrow Nie wolno kontynuować jazdy.
	\itemArrow Wyłączyć silnik i zwrócić się o pomoc do specjalistycznej stacji obsługi.
\end{itemizeArrow}

Występuje utrata płynu chłodzącego
\begin{itemizeArrow}
	\itemArrow Uzupełnić płyn chłodzący i zwrócić się o pomoc do specjalistycznej stacji obsługi.
\end{itemizeArrow}

Sterownik silnika

Rozwiązywanie problemów

Uszkodzone sterowanie silnika benzynowego

XX świeci

Napęd w trybie awaryjnym jest możliwy -- może dojść do zauważalnej redukcji mocy silnika.

\begin{itemizeArrow}
	\itemArrow Niezwłocznie skorzystaj z pomocy specjalistycznej stacji obsługi.
\end{itemizeArrow}

Filtr cząstek stałych (GPF)

Rozwiązywanie problemów

Zatkany filtr cząstek stałych

XX zapala się razem z XX
\begin{itemizeArrow}
	\itemArrow Wyczyść filtr.
\end{itemizeArrow}
Gdy lampka kontrolna świeci się, należy spodziewać się zwiększonego zużycia paliwa i obniżonej mocy silnika.

Czyszczenie filtra

Obsługa czyszczenia
\begin{itemizeTick}
	\itemTick Silnik jest rozgrzany.
\end{itemizeTick}

Przebieg czyszczenia
\begin{itemizeArrow}
	\itemArrow Jechać z prędkością co najmniej 80 km/h przy prędkościach obrotowych silnika między 3000-5000 obr./min.
	\itemArrow Zwolnij pedał przyspieszenia i pozwól, aby pojazd przetoczył się przez kilka sekund z włączonym biegiem.
	\itemArrow Powtórz tę procedurę kilka razy.
\end{itemizeArrow}

Jeżeli uda się oczyścić filtr, lampka kontrolna XX zgaśnie.

Jeśli lampka kontrolna XX nie zgaśnie w ciągu 30 minut, nie było czyszczenia filtra.
\begin{itemizeArrow}
	\itemArrow Niezwłocznie skorzystaj z pomocy specjalistycznej stacji obsługi.
\end{itemizeArrow}

System kontroli spalin

Rozwiązywanie problemów

System kontroli emisji został zakłócony

XX świeci

Napęd w trybie awaryjnym jest możliwy -- może dojść do zauważalnej redukcji mocy silnika.

\begin{itemizeArrow}
	\itemArrow Niezwłocznie skorzystaj z pomocy specjalistycznej stacji obsługi.
\end{itemizeArrow}

Przesłony chłodnicy

Żaluzje chłodnicy redukują szkodliwe dla środowiska emisje.

Rozwiązywanie problemów

Jeśli na wyświetlaczu zestawu wskaźników pojawi się komunikat dotyczący ograniczenia funkcji żaluzji, maksymalna prędkość pojazdu jest ograniczona do 160~km/h.
Przyczyną może być lód lub śnieg w obszarze żaluzji.
Po stopieniu się lodu lub śniegu żaluzje będą znowu sprawne.
\begin{itemizeArrow}
	\itemArrow Jeżeli przyczyną ograniczenia działania żaluzji nie jest lód czy śnieg, należy zwrócić się o pomoc do specjalistycznej stacji obsługi.
\end{itemizeArrow}

Klapka wlewu paliwa

Otwieranie pokrywy wlewu paliwa

\begin{itemizeArrow}
	\itemArrow Nacisnąć przycisk XX w drzwiach kierowcy.
\end{itemizeArrow}
Klapka wlewu paliwa jest odblokowywana po kilku sekundach i częściowo otwarta.
\begin{itemizeArrow}
	\itemArrow Otwórz klapkę.
\end{itemizeArrow}

Benzyna

INFORMACJA
\begin{itemizeTriangle}
	\itemTriangle Nigdy nie opróżniaj całkowicie zbiornika paliwa!
\end{itemizeTriangle}

Przepisy
Normy
Benzyna musi spełniać europejską normę EN 228.
Stosować tylko benzynę bezołowiową zawierającą maksymalnie 10\% bioetanolu (E10).

Liczba oktanowa
Użyj benzyny o liczbie oktanowej (LO) co najmniej 95.

\begin{itemizeTriangle}
	\itemTriangle W przypadku stosowania benzyny o LO niższej niż zalecana (92 lub 93) kontynuować jazdę tylko ze średnimi prędkościami obrotowymi i minimalnym obciążeniem silnika.
	\itemTriangle Jak najszybciej należy zatankować paliwo o zalecanej LO.
	\itemTriangle Nie stosować benzyny o LO niższej niż 91!
\end{itemizeTriangle}


Wymagania dotyczące tankowania
\begin{itemizeTick}
	\itemTick Pojazd odblokowany
	\itemTick Ogrzewanie postojowe wyłączone
	\itemTick Zapłon wyłączony
	\itemTick Klapka pokrywy wlewu paliwa odblokowana
\end{itemizeTick}

Dolewanie
\begin{itemizeArrow}
	\itemArrow Otworzyć pokrywę wlewu paliwa.
	\itemArrow Przekręć korek w kierunku strzałki i zdejmij go.
	\itemArrow Załóż korek wlewu paliwa na klapkę wlewu paliwa.
	\itemArrow Włożyć pistolet dystrybutora do oporu do króćca wlewu paliwa i zatankować.
	\itemArrow Nie należy kontynuować tankowania po wyłączeniu dyszy paliwowej.
	\itemArrow Wyjąć pistolet dystrybutora z króćca wlewu paliwa i zawiesić ponownie w dystrybutorze.
	\itemArrow Włożyć korek na króciec wlewu paliwa i obrócić w kierunku przeciwnym do kierunku strzałki do wyraźnego zatrzaśnięcia.
	\itemArrow Zamknij klapkę wlewu paliwa do zatrzaśnięcia.
\end{itemizeArrow}

Rozwiązywanie problemów

Jeżeli zatankowano paliwo inne niż benzyna bezołowiowa zgodna z normą
\begin{itemizeArrow}
	\itemArrow Nie uruchamiaj silnika ani nie włączaj zapłonu.
	\itemArrow Skorzystać z pomocy specjalistycznej stacji obsługi.
\end{itemizeArrow}

Dane techniczne
Paliwo przepisane do Twojego pojazdu jest wskazane na naklejce na wewnętrznej stronie klapki wlewu paliwa.
% FIXME: Rys.
A Benzyna bezołowiowa
B Procent dodatków organicznych
Pojemność zbiornika paliwa wynosi około 40 litrów, z czego ok. 7 litrów stanowi rezerwę.

XX świeci się - zapas paliwa dotarł do poziomu rezerwowego.

Akumulator 12 V i bezpieczniki

Akumulator 12 V

Czego należy przestrzegać

OSTRZEŻENIE
Akumulator pojazdu 12 V podlega naturalnemu zużyciu. Przy dużym zużyciu istnieje ryzyko ograniczenia funkcjonalnego lub awarii niektórych układów pojazdu, np. wzmacniacz hamulca, wspomaganie kierownicy lub układ poduszek powietrznych.

\begin{itemizeTriangle}
	\itemTriangle Oddać akumulator pojazdu 12 V do wymiany specjalistycznej firmie co najmniej co 4 lata.
\end{itemizeTriangle}

Sposób działania – zabezpieczenie przed rozładowaniem akumulatora pojazdu 12 V

Możliwe przyczyny rozładowania akumulatora pojazdu 12 V.
\begin{itemizeTriangle}
	\itemTriangle Zużyty akumulator pojazdu 12 V
	\itemTriangle Niskie temperatury
	\itemTriangle Długoterminowy postój pojazdu
\end{itemizeTriangle}

Ładowanie akumulatora pojazdu 12 V

Przebieg ładowania
Akumulator pojazdu 12 V jest automatycznie ładowany po naładowaniu akumulatora wysokonapięciowego.

Akumulator pojazdu 12 V można ładować za pomocą ładowarki 12 V.

Wymagania dotyczące ładowania akumulatora pojazdu o napięciu 12 V.
\begin{itemizeTick}
	\itemTick Zapłon wyłączony
	\itemTick Odbiorniki energii wyłączone
\end{itemizeTick}

Ładowanie akumulatora pojazdu 12 V

Dla pełnego naładowania akumulatora pojazdu 12~V ustawić prąd ładowania maks. 0,1 x pojemność akumulatora.
% FIXME: Rys.
A Punkt masy
+ Biegun XX do ładowania akumulatora pojazdu 12 V.

Biegun XX akumulatora pojazdu 12 V znajduje się w komorze silnika pod pokrywą skrzynki bezpieczników
\begin{itemizeArrow}
	\itemArrow Zdjąć pokrywę skrzynki bezpieczników \guillemotright strona 177.
	\itemArrow Zacisk ładowarki XX podłącz do bieguna XX akumulatora pojazdu.
	\itemArrow Zacisk ładowarki XX podłączyć do punktu masy A.
	\itemArrow Podłączyć przewód sieciowy urządzenia do ładowania do sieci i włączyć je.
	\itemArrow Po naładowaniu wyłącz ładowarkę i odłącz przewód zasilający od gniazda ściennego.
	\itemArrow Odłączyć zaciski ładowarki od akumulatora pojazdu 12 V.
	\itemArrow Założyć i zatrzasnąć pokrywę skrzynki bezpieczników.
\end{itemizeArrow}

OSTRZEŻENIE
Niebezpieczeństwo wybuchu!
\begin{itemizeTriangle}
	\itemTriangle Podczas ładowania uwalniany jest wodór. Nawet iskra, powstająca np. przy odłączaniu lub luzowaniu przewodów, może wywołać wybuch.
	\itemTriangle W żadnym razie nie ładować zamrożonego lub odtajałego akumulatora pojazdu 12 V.
	\itemTriangle Nie należy wykonywać tak zwanego szybkiego ładowania akumulatora pojazdu 12 V, ale zlecić to specjalistycznej firmie.
\end{itemizeTriangle}

INFORMACJA
Rozładowany akumulator pojazdu 12 V może łatwo zamarznąć!

Rozwiązywanie problemów

Usterka silnika

XX zapala się razem z XX
\begin{itemizeArrow}
	\itemArrow Nie wolno kontynuować jazdy!
	\itemArrow Wyłączyć silnik i zwrócić się o pomoc do specjalistycznej stacji obsługi.
\end{itemizeArrow}

Nieprawidłowe ładowanie 12-woltowego akumulatora pojazdu

XX zapala się razem z XX
\begin{itemizeArrow}
	\itemArrow Nie wolno kontynuować jazdy!
	\itemArrow Wyłączyć zapłon i zwrócić się o pomoc do specjalistycznej stacji obsługi.
\end{itemizeArrow}

Awaria 12- woltowego akumulatora pojazdu

XX zapala się razem z XX

Komunikat na wyświetlaczu zestawu wskaźników dotyczący usterki akumulatora 12 V pojazdu lub układów wtórnych.

\begin{itemizeArrow}
	\itemArrow Skorzystać z pomocy specjalistycznej stacji obsługi.
\end{itemizeArrow}

Stan naładowania 12-woltowego akumulatora pojazdu jest niewystarczający

XX zapala się razem z XX

Komunikat na wyświetlaczu zestawu wskaźników dotyczący słabego lub rozładowanego akumulatora 12 V pojazdu.
\begin{itemizeArrow}
	\itemArrow Przejedź kilka kilometrów, aby naładować 12-woltowy akumulator pojazdu.
\end{itemizeArrow}
Lub:
\begin{itemizeArrow}
	\itemArrow Naładuj pojazd za pomocą ładowarki.
\end{itemizeArrow}
Lub:
\begin{itemizeArrow}
	\itemArrow Naładuj 12-woltowy akumulator samochodowy za pomocą ładowarki.
\end{itemizeArrow}

Zużyty akumulator samochodowy 12 V.

XX zapala się razem z XX

Komunikat na wyświetlaczu zestawu wskaźników dotyczący wymiany 12-woltowego akumulatora pojazdu.
\begin{itemizeArrow}
	\itemArrow Skorzystać z pomocy specjalistycznej stacji obsługi.
\end{itemizeArrow}

Odłącz, podłącz i zmień

Akumulator pojazdu 12 V znajduje się w bagażniku i nie jest dostępny. Akumulator pojazdu 12 V może być odłączany i podłączany tylko przez specjalistyczną firmę.

Wymiana

Nowy akumulator pojazdu 12 V musi mieć takie same parametry jak oryginalny akumulator. Wymianę przeprowadzi wyspecjalizowana firma.

Używanie kabli rozruchowych

Za pomocą przewodów rozruchowych można włączyć pojazd z rozładowanym lub uszkodzonym akumulatorem pojazdu 12 V za pomocą akumulatora pojazdu 12 V innego samochodu.

Czego należy przestrzegać

OSTRZEŻENIE
Niebezpieczeństwo wybuchu i poparzenia!
\begin{itemizeTriangle}
	\itemTriangle Nie należy uruchamiać silnika z akumulatora 12 V innego samochodu w następujących warunkach:
		\itemTriangle Rozładowany akumulator pojazdu 12 V jest zamrożony. Rozładowany akumulator pojazdu 12 V może zamarznąć już w temperaturze niewiele niższej od 0°C.
		\itemTriangle Poziom kwasu w akumulatorze pojazdu 12 V jest zbyt niski \guillemotright strona 172.
\end{itemizeTriangle}

Użyj kabli połączeniowych o wystarczającym przekroju i izolowanych szczypcach.

Napięcie znamionowe obu akumulatorów musi wynosić 12 V. Pojemność (w Ah) akumulatora rozruchowego 12 V nie może być dużo mniejsza niż pojemność akumulatora rozładowanego 12 V.

Uruchomić samochód za pomocą akumulatora 12 V innego samochodu

OSTRZEŻENIE
Niebezpieczeństwo zranienia! Niebezpieczeństwo uszkodzenia pojazdu!
\begin{itemizeTriangle}
	\itemTriangle Poprowadź kable rozruchowe tak, aby nie zostały pochwycone przez obracające się części w komorze silnika.
\end{itemizeTriangle}

INFORMACJA
Niebezpieczeństwo zwarcia!
\begin{itemizeTriangle}
	\itemTriangle Nieizolowane części szczypiec nie mogą się stykać.
	\itemTriangle Przewód rozruchowy połączony z dodatnim biegunem akumulatora pojazdu 12 V nie może się stykać z częściami samochodu przewodzącymi prąd – ryzyko zwarcia.
	\itemTriangle Pojazdy nie mogą się stykać.
\end{itemizeTriangle}

Podłącz przewód rozruchowy
% FIXME: Rys.
\begin{itemizeArrow}
	\itemArrow Wyłączyć zapłon.
	\itemArrow Zacisnąć końcówki kabli rozruchowych zgodnie z kolejnością w legendzie.
\end{itemizeArrow}
XX rozładowany akumulator pojazdu 12 V
XX Akumulator rozruchowy 12 V
1 Biegun rozładowanego akumulatora pojazdu 12 V
2 Biegun akumulatora rozruchowego 12 V
3 Biegun akumulatora rozruchowego 12 V (lub punkt masy)
4 Punkt masy samochodu z rozładowanym akumulatorem pojazdu 12 V

% FIXME: Rys.

Komora silnika: Punkt masy / biegun XX akumulatora pojazdu 12 V (w skrzynce bezpieczników)

Uruchomić silnik.
\begin{itemizeArrow}
	\itemArrow Uruchomić silnik samochodu rozruchowego i pozostawić go na biegu jałowym (dotyczy samochodów z silnikiem spalinowym).
\end{itemizeArrow}
Lub:
\begin{itemizeArrow}
	\itemArrow Włączyć napęd elektryczny samochodu rozruchowego (dotyczy pojazdów z silnikiem elektrycznym).
	\itemArrow Uruchomić samochód z rozładowanym akumulatorem pojazdu 12 V.
	\itemArrow Jeśli silnik samochodu z rozładowanym akumulatorem pojazdu 12 V nie uruchomi się w ciągu 10 sekund, powtórzyć procedurę rozruchu po około 30 sekundach.
\end{itemizeArrow}

Odłącz kable

\begin{itemizeArrow}
	\itemArrow Odłącz kable rozruchowe w odwrotnej kolejności.
\end{itemizeArrow}

Bezpieczniki

\begin{itemizeTriangle}
	\itemTriangle Jeśli nowo użyty bezpiecznik ponownie się przepali, zwróć się o pomoc do specjalistycznej firmy.
\end{itemizeTriangle}

Do jednego bezpiecznika może być przyporządkowanych kilka odbiorników. Jednemu odbiornikowi może odpowiadać kilka bezpieczników.

Bezpiecznik systemu wysokiego napięcia (bezpiecznik dla ratowników)

Bezpiecznik systemu wysokiego napięcia jest zaopatrzony w żółtą naklejkę, aby służby ratunkowe mogły jak najszybciej wyłączyć wysokie napięcie w pojeździe.

OSTRZEŻENIE
Zagrożenie życia lub ryzyko porażenia prądem i poważnych oparzeń!
\begin{itemizeTriangle}
	\itemTriangle Nie zmieniaj bezpiecznika systemu wysokonapięciowego!
	\itemTriangle Skorzystać z pomocy specjalistycznej stacji obsługi.
\end{itemizeTriangle}

Sposób działania

% FIXME: Rys.

Warunki wymiany bezpiecznika
\begin{itemizeTick}
	\itemTick Zapłon wyłączony
	\itemTick Drzwi kierowcy otwarte
	\itemTick Wyłączone wszystkie odbiorniki elektryczne
\end{itemizeTick}

Wymiana bezpiecznika

% FIXME: Rys.

\begin{itemizeArrow}
	\itemArrow Wymień bezpiecznik zaciskiem znajdującym się pod pokrywą skrzynki bezpieczników w komorze silnika.
	\itemArrow Użyj odpowiedniego końca zacisku zgodnie z wymiarami bezpiecznika.
\end{itemizeArrow}

Bezpieczniki w tablicy rozdzielczej

Przegląd

Dostęp do bezpieczników - pojazdy z kierownicą po lewej stronie

% FIXME: Rys.

\begin{itemizeArrow}
	\itemArrow Otworzyć schowek po stronie kierowcy.
	\itemArrow Naciśnij przycisk i otwórz schowek.
	\itemArrow Zmień bezpiecznik.
	\itemArrow Zamknąć schowek.
\end{itemizeArrow}

Przegląd bezpieczników

% FIXME: Rys.

Numer bezpiecznika	Odbiornik
1 wolny
2 wolny
3 Hak holowniczy
4 wolny
5 Dźwignia sterująca automatycznej skrzyni biegów
6 Oświetlenie wnętrza
7 Ogrzewanie siedzeń przednich
8 wolny
9 Centralne ryglowanie + podnośnik szyb (lewa strona pojazdu), lewe lusterko zewnętrzne (ogrzewanie, funkcja składania, ustawienie powierzchni lusterka)
10 wolny
11 Hak holowniczy
12 Światła zewnętrzne pojazdu, reflektor przeciwmgłowy prawy, podniesione światła hamowania, oświetlenie tablicy rejestracyjnej
13 Zamek centralny (drzwi tylne i pokrywa bagażnika, klapka wlewu paliwa), spryskiwacz szyby przedniej, system czyszczenia reflektorów
14 wzmacniacz muzyki
15 wolny
16 Poduszki bezpieczeństwa
17 wolny
18 Blokada kolumny kierownicy, KESSY (system bezkluczykowego rozruchu)
19 Zestaw wskaźników, połączenie alarmowe, usługi online
20 Phonebox, porty USB
21 Kamera cofania, system obserwacji otoczenia (Top View)
22 Wentylacja siedziska z przodu
23 Oświetlenie wnętrza z przodu, mikrofon, obsługa dachu przesuwno-uchylnego
24 wolny
25 Napinacz pasa – z przodu z lewej
26 Centralne ryglowanie + podnośnik szyb (prawa strona pojazdu), prawe lusterko zewnętrzne (ogrzewanie, funkcja składania, ustawienie powierzchni lusterka)
27 Napinacz pasa – z przodu z prawej
28 System kontroli akumulatorów wysokonapięciowych – bezpieczne wyłączanie systemu wysokonapięciowego [Bezpiecznik może wymieniać tylko specjalistyczna firma!]
29 Hak holowniczy
30 urządzenie Infotainment
31 Hak holowniczy
32 ogrzewanie tylnych siedzeń
33 wolny
34 gniazdo 230 V
35 Światła zewnętrzne pojazdu, reflektor przeciwmgłowy lewy
36 Klimatyzacja
37 Elektryczna pokrywa bagażnika
38 wolny
39 ogrzewanie kierownicy
40 wolny
41 magistrala danych
42 Automatyczna skrzynia biegów, wskaźnik biegów
43 Klimatyzacja przednia, klimatyzacja tylna, ogrzewanie dodatkowe (ogrzewanie postojowe), ogrzewanie tylnej szyby
44 Asystent zmiany pasa ruchu (Side Assist), złącze diagnostyczne, czujnik deszczu do świateł, automatyczna regulacja zasięgu reflektorów, hamulec postojowy, alarm, włącznik światła, przyciski w konsoli środkowej
45 Elektronika kolumny kierownicy, dźwignia obsługowa pod kierownicą
46 Ekran Infotainment, wyświetlacz przezierny
47 Układ aktywnego zawieszenia (DCC)
48 wolny
49 wolny
50 wolny
51 wolny
52 gniazdo 12 V w bagażniku
53 Bezdotykowa obsługa pokrywy bagażnika
54 wolny
55 wolny
56 wolny
57 wolny
58 Asystent parkowania, pomoc w parkowaniu, przedni czujnik radarowy, kamera przednia do systemów asystujących
59 Hamulec postojowy, klimatyzacja, czujnik biegu wstecznego, lusterko wewnętrzne, generator dźwięku silnika
60 Złącze diagnostyczne
61 Przełącznik pedału sprzęgła, układ wysokowoltowy, elektronika do napędu elektrycznego
62 USB tylne, USB na lusterku wewnętrznym
63 wolny
64 wolny
65 Generator dźwięku silnika
66 wycieraczka szyby tylnej
67 Ogrzewanie szyby tylnej

Bezpieczniki w komorze silnika

Przegląd

Dostęp do bezpieczników

% FIXME: Rys.

\begin{itemizeArrow}
	\itemArrow Jednocześnie naciśnij przyciski blokujące na pokrywie skrzynki bezpieczników i zdejmij pokrywę.
	\itemArrow Zmień bezpiecznik.
	\itemArrow Załóż pokrywę i zatrzaśnij.
\end{itemizeArrow}

INFORMACJA
Niebezpieczeństwo przedostania się wody do skrzynki bezpieczników!
\begin{itemizeTriangle}
	\itemTriangle Dopasuj prawidłowo pokrywę i zaczep mocno.
\end{itemizeTriangle}

Przegląd bezpieczników

% FIXME: Rys.

Numer bezpiecznika	Odbiornik
1	Wspomaganie kierownicy
Bezpiecznik może wymieniać tylko specjalistyczna firma!
2	System pomocy ESC, układ sterowania silnika, elementy silnika
3	Ładowarka do akumulatora wysokiego napięcia, napęd elektryczny
4 Lewy reflektor przedni
5 Prawy przedni reflektor
6 System alarmowy
7 wolny
8 Urządzenie wspomagania hamulców
9 sygnał dźwiękowy
10 wycieraczki przednie
11 Klimatyzacja
12 wolny
13 System pomocy ESC
14 Ogrzewanie dodatkowe (ogrzewanie postojowe)
15 System pomocy ESC
16 Automatyczna skrzynia biegów
17 Ogrzewanie
18 Ogrzewanie
19 wolny
20 Blokada poprzeczna przedniej osi (VAQ)
21 układ sterowania silnika
22 wolny
23 układ sterowania silnika
24 części silnika
25 części silnika
26 części silnika
27 Sondy lambda
28 części silnika
29 Pompa paliwa
30 części silnika
31 wolny
32 Ogrzewanie szyby przedniej
33 Ogrzewanie

Koła

Opony i felgi

Czego należy przestrzegać
INFORMACJA
\begin{itemizeTriangle}
	\itemTriangle Koła lub opony przechowywać w miejscu chłodnym, suchym i możliwie ciemnym. Opony bez felg należy przechowywać w pozycji stojącej.
	\itemTriangle Felgi ze stopów lekkich są uszkadzane przez materiał, którym posypywane są drogi.
	\itemTriangle Nie używaj obręczy ze stopów lekkich z błyszczącą powierzchnią w zimowych warunkach pogodowych lub z łańcuchami śniegowymi
\end{itemizeTriangle}

Przyczyny nierównomiernego zużycia opon

\begin{itemizeTriangle}
	\itemTriangle Niewłaściwe ciśnienie w oponach.
	\itemTriangle Sposób jazdy (np. szybkie pokonywanie zakrętów, ostre przyspieszanie / ostre hamowanie).
	\itemTriangle Złe ustawienie kół.
	\itemTriangle Nieprawidłowe wyważenie kół.
\end{itemizeTriangle}

Zamiana kół

W celu równomiernego zużycia opon zalecamy zamianę kół co 10 000 km według schematu:
% FIXME: Rys.

Porada dotycząca montażu opon
\begin{itemizeArrow}
	\itemArrow W przypadku opon kierunkowych należy zwrócić uwagę na prawidłowy kierunek obrotów. Kierunek obrotu jest zaznaczony strzałkami na boku opony.
	\itemArrow Zawsze wymieniaj opony osiami.
\end{itemizeArrow}

Informacje uzupełniające

Na wszystkich czterech kołach powinny być założone dopuszczalne opony radialne tego samego rodzaju, rozmiaru (obwodu) i o tym samym bieżniku na osi.

Przegląd etykietowania opon

Objaśnienie opisów opon
Np. 205/60 R 16 92 H
205	szerokość opony w mm
60	stosunek wysokości do szerokości w \%
R Oznaczenie konstrukcji opony – opona radialna
16	średnica nominalna obręczy w calach
92	Indeks obciążenia
H	Indeks prędkości

Wskaźniki zużycia opon

% FIXME: Rys.

W podstawie bieżnika opony umieszczone są wskaźniki zużycia, które wskazują minimalną głębokość bieżnika.

Oznaczenia na boku opony – litery TWI lub inne symbole, np. XX, wskazują położenie wskaźników zużycia.

Opona uznawana jest za zużytą, gdy wskaźnik ten zrówna się z bieżnikiem opony.

Głębokość bieżnika można zmierzyć za pomocą miernika głębokości bieżnika na skrobaczce. Skrobaczka znajduje się po wewnętrznej stronie pokrywy wlewu paliwa.

Opona - data produkcji

Data produkcji jest podana na bocznej ścianie opony.

Np. DOT ... 18 20... oznacza, że opona została wyprodukowana w 18. tygodniu 2020 roku.

OSTRZEŻENIE
Niebezpieczeństwo wypadku!
\begin{itemizeTriangle}
	\itemTriangle Nie używać opon starszych niż 6 lat.
\end{itemizeTriangle}

Rozwiązywanie problemów

Pomoc przy awarii opony

\begin{itemizeTriangle}
	\itemTriangle Zestaw do naprawy opon \guillemotright strona 181.
\end{itemizeTriangle}

Zmiana ciśnienia w oponach.

XX .
\begin{itemizeArrow}
	\itemArrow Zatrzymać samochód.
	\itemArrow Sprawdź opony i ciśnienie w oponach.
\end{itemizeArrow}

Pokaż oponę ze zmienionym ciśnieniem

\begin{itemizeArrow}
	\itemArrow Wybierz pozycję menu do wyświetlania stanu opon w menu XX.
\end{itemizeArrow}

Opony całoroczne lub zimowe

Opony całoroczne lub zimowe poprawiają prowadzenie w warunkach zimowych. Są one oznaczone jako M+S z symbolem szczytu górskiego i płatka śniegu XX.

Czego należy przestrzegać

W celu zachowania optymalnych właściwości jezdnych minimalna głębokość bieżnika opon całorocznych lub zimowych musi wynosić 4 mm.

Łańcuchy przeciwśnieżne

Zamontowane obręcze kół (R18) nie dopuszczają stosowania łańcuchów przeciwśnieżnych.

Wymiana koła i podnoszenie samochodu

Czego należy przestrzegać

Przed wymianą

\begin{itemizeArrow}
	\itemArrow Zaparkować pojazd bezpiecznie i zabezpieczyć przed stoczeniem.
	\itemArrow Wyłączyć silnik.
	\itemArrow Poprosić wszystkich pasażerów o opuszczenie samochodu. W czasie wymiany koła pasażerowie powinni przebywać poza jezdnią (np. za barierkami).
	\itemArrow Wszystkie drzwi, bagażnik i pokrywa komory silnika są zamknięte.
	\itemArrow Odczepić przyczepę, jeżeli jest przyczepiona.
\end{itemizeArrow}

Podnoszenie samochodu

OSTRZEŻENIE
Zabezpieczyć podstawę podnośnika przed przemieszczaniem się.
\begin{itemizeTriangle}
	\itemTriangle Na grząskim podłożu (np. żwirowym) ustawić pod podnośnikiem stabilną podkładkę o dużej powierzchni.
	\itemTriangle Na śliskim podłożu (np. na kostce brukowej) ustawić pod podnośnikiem podkładkę antypoślizgową (np. gumową matę).
	\itemTriangle Dopilnować, aby w podnoszonym samochodzie zawsze były zamknięte drzwi.
	\itemTriangle Nie wkładaj żadnych części ciała pod podniesiony pojazd.
	\itemTriangle Nie uruchamiaj silnika podniesionego pojazdu.
\end{itemizeTriangle}

INFORMACJA
Niebezpieczeństwo uszkodzenia pojazdu!
\begin{itemizeTriangle}
	\itemTriangle Upewnij się, że podnośnik jest prawidłowo przymocowany do belki podwozia.
\end{itemizeTriangle}

Po wymianie
\begin{itemizeArrow}
	\itemArrow Sprawdź ciśnienie w oponie zamontowanego koła, w razie potrzeby skoryguj.
	\itemArrow W pojazdach ze wskaźnikiem kontroli ciśnienia w oponach należy zapisać ciśnienie w układzie.
	\itemArrow Gdy tylko będzie to możliwe, sprawdzić moment dokręcenia śrub koła. Prawidłowy moment dokręcania wynosi 140 Nm.
\end{itemizeArrow}

Jedź ostrożnie, aż sprawdzony zostanie moment dokręcania.

OSTRZEŻENIE
Niebezpieczeństwo wypadku!
\begin{itemizeTriangle}
	\itemTriangle Zbyt mały moment dokręcania może spowodować odłączenie koła podczas jazdy.
	\itemTriangle Zbyt wysoki moment dokręcania może uszkodzić gwinty i spowodować odkształcenie obręczy.
\end{itemizeTriangle}

Wymiana koła i podnoszenie samochodu

Odkręcić śruby

\begin{itemizeArrow}
	\itemArrow Jeśli pojazd ma nakładki na śruby kół lub pełne kołpaki, należy je zdjąć.
\end{itemizeArrow}

OSTRZEŻENIE
\begin{itemizeTriangle}
	\itemTriangle Gdy naciskasz nogą koniec klucza odkręcając śrubę, przytrzymaj się pojazdu, aby uzyskać lepszą stabilność.
\end{itemizeTriangle}

% FIXME: Mamy to?
Niektóre pojazdy mogą mieć śruby zabezpieczające przed kradzieżą, które chronią koła przed kradzieżą.
\begin{itemizeArrow}
	\itemArrow Nałożyć nasadkę do śrub zabezpieczających na śrubę zabezpieczającą do oporu.
	\itemArrow Włóż klucz na śrubę koła lub na nasadkę.
	\itemArrow Przekręcić śrubę maksymalnie o pół obrotu, aby koło nie poluzowało się i nie odpadło.
\end{itemizeArrow}


Miejsca do zakładania podnośnika
% FIXME: Rys.
A 18 cm
B 30 cm

Ustaw podnośnik i unieś pojazd
% FIXME: Rys.
\begin{itemizeArrow}
	\itemArrow Podnośnik ustawić w punkcie podparcia możliwie najbliżej wymienianego koła.
	\itemArrow Należy się upewnić, że podstawa podnośnika jest ustawiona całą powierzchnią na równym podłożu, tak aby podnośnik stał pionowo względem miejsca podparcia.
	\itemArrow Podnieść podnośnik za pomocą korby na tyle, aby jego szczęka obejmowała próg.
	\itemArrow Podnosić samochód, aż koło oderwie się od podłoża.
\end{itemizeArrow}

Wymiana koła
\begin{itemizeArrow}
	\itemArrow Odkręć śruby i umieść je na czystej powierzchni.
	\itemArrow Ostrożnie zdjąć koło.
	\itemArrow Załóż koło.
	\itemArrow Lekko wkręcić śruby.
	\itemArrow Opuść pojazd.
\end{itemizeArrow}

INFORMACJA
\begin{itemizeTriangle}
	\itemTriangle W przypadku fabrycznie dostarczonych kołpaków lub kołpaków z oferty oryginalnych akcesoriów ŠKODA śrubę zabezpieczającą przed kradzieżą koła montować odpowiednio do pozycji oznaczonej na odwrocie kołpaka.
\end{itemizeTriangle}

Dokręcić śruby
\begin{itemizeArrow}
	\itemArrow Dokręcić kolejno naprzemiennie śruby koła, łącznie ze śrubą przeciwkradzieżową.
	\itemArrow Ponownie założyć nakładki śrub kół lub pełne kołpaki kół.
\end{itemizeArrow}

Zachowaj etykietę z numerem kodu śrub zabezpieczających przed kradzieżą. Na podstawie tego kodu możliwe jest uzyskanie nasadki zastępczej z oferty oryginalnych części ŠKODA.

Zestaw awaryjny

Przegląd treści

Zestaw jest umieszczony w skrzynce pod wykładziną podłogową bagażnika.
% FIXME: Rys.
A Naklejka ze wskaźnikiem prędkości
B wkrętak do wkładu zaworu
C wężyk do napełniania z korkiem
D Sprężarka (rozmieszczenie elementów obsługowych zależy od typu sprężarki dostarczonej wraz
z samochodem)
E wężyk do napełniania opony
F Przycisk redukcji ciśnienia
G Wskaźnik ciśnienia
H wtyczka do kabla 12 V
I przełącznik WŁ. i WYŁ.
J Wężyk do napełniania opony
K wymienny wkład zaworu

Deklaracja zgodności jest załączona do sprężarki lub kompletu dokumentów.

Warunki użytkowania

Naprawa wykonana za pomocą zestawu awaryjnego w żadnym razie nie zastępuje fachowej naprawy opony.

Służy wyłącznie do dojechania do najbliższej specjalistycznej stacji obsługi.

Zmień oponę naprawioną zestawem awaryjnym tak szybko, jak to możliwe.

Przed użyciem zestawu
\begin{itemizeArrow}
	\itemArrow Zaparkować pojazd bezpiecznie i zabezpieczyć przed stoczeniem.
	\itemArrow Wyłączyć silnik.
	\itemArrow Poprosić wszystkich pasażerów o opuszczenie samochodu. W czasie naprawy koła pasażerowie powinni przebywać poza jezdnią (np. za barierkami).
	\itemArrow Zamknąć wszystkie drzwi, jak również pokrywę komory silnika i pokrywę bagażnika.
	\itemArrow Odczepić przyczepę, jeżeli jest przyczepiona.
\end{itemizeArrow}

Po użyciu zestawu

Jeśli nie można osiągnąć ciśnienia 2,0 bar, opona jest zbyt zniszczona i nie można jej uszczelnić za pomocą zestawu naprawczego.
\begin{itemizeArrow}
	\itemArrow Nie wolno kontynuować jazdy. Skorzystać z pomocy specjalistycznej stacji obsługi.
\end{itemizeArrow}
Po uzyskaniu ciśnienia 2,0-2,5 bar można kontynuować jazdę z maks. prędkością 80 km/h.
\begin{itemizeArrow}
	\itemArrow Unikać gwałtownych przyspieszeń, ostrego hamowania i szybkiego pokonywania zakrętów.
\end{itemizeArrow}

Wskazówki dotyczące jazdy z naprawioną oponą
\begin{itemizeArrow}
	\itemArrow Po 10 minutach jazdy na naprawionej oponie skontrolować ciśnienie w oponie.
	\itemArrow Jeśli ciśnienie w oponach wynosi 1,3 bara i mniej, nie kontynuuj! Skorzystać z pomocy specjalistycznej stacji obsługi.
	\itemArrow Jeśli ciśnienie w oponach jest wyższe niż 1,3 bara, należy skorygować ciśnienie do wartości co najmniej 2 barów i kontynuować jazdę.
\end{itemizeArrow}

INFORMACJA
Niebezpieczeństwo uszkodzenia sprężarki!
\begin{itemizeTriangle}
	\itemTriangle Po maksymalnym czasie pracy sprężarki, pozwól jej ostygnąć przez kilka minut.
\end{itemizeTriangle}

OSTRZEŻENIE
Niebezpieczeństwo poparzenia!
Podczas pompowania wężyk i sprężarka mogą się mocno nagrzewać.
\begin{itemizeTriangle}
	\itemTriangle Pozwól sprężarce i wężowi ostygnąć przez kilka minut.
\end{itemizeTriangle}

Ograniczenie stosowania

Nie używaj zestawu w następujących przypadkach:
\begin{itemizeTriangle}
	\itemTriangle Obręcz koła jest uszkodzona.
	\itemTriangle Temperatura zewnętrzna jest niższa od temperatury minimalnej podanej w instrukcji obsługi butelki ze środkiem do napełniania opon.
	\itemTriangle Nakłucia opony są większe niż 4 mm.
	\itemTriangle Uszkodzony jest bok opony.
	\itemTriangle Data przydatności do użytku podana na butelce ze środkiem do napełniania opon jest przekroczona.
\end{itemizeTriangle}

Samodzielne naprawy

Uszczelnij opony
\begin{itemizeArrow}
	\itemArrow Odkręcić kapturek ochronny zaworu uszkodzonej opony.
	\itemArrow Wkrętakiem do wkładów zaworowych wykręcić wkład zaworu i umieścić go na czystej powierzchni.
	\itemArrow Kilka razy potrząsnąć energicznie butelką ze szczeliwem do napełniania oponę.
	\itemArrow Mocno przykręcić wężyk do napełniania na butelce ze środkiem do napełniania opon. Folia na zamknięciu butelki zostanie przebita automatycznie.
	\itemArrow Zdjąć korek z wężyka do napełniania i włożyć na zawór opony.
	\itemArrow Butelkę trzymać do góry dnem i wprowadzić do opony całą jej zawartość.
	\itemArrow Zdjąć wężyk do napełniania z zaworu opony.
	\itemArrow Wkręcić wkładkę zaworu.
\end{itemizeArrow}

Napompuj opony
\begin{itemizeArrow}
	\itemArrow Założyć wąż do pompowania powietrza w sprężarce powietrza na zawór opony.
	\itemArrow Zabezpieczyć pojazd hamulcem postojowym.
	\itemArrow Włączyć silnik.
	\itemArrow Podłączyć wtyk sprężarki do gniazda 12 V.
	\itemArrow Włączyć sprężarkę powietrza.
	\itemArrow Gdy ciśnienie w oponie osiągnie wartość 2,0-2,5 bar, wyłączyć sprężarkę. \\ Przestrzegać maksymalnego czasu pracy sprężarki zgodnie z instrukcją producenta zestawu awaryjnego.
	\itemArrow Jeżeli nie można osiągnąć ciśnienia powietrza 2,0-2,5 bar, odkręcić wężyk z zaworu opony.
	\itemArrow Przejechać samochodem ok. 10 metrów w przód lub w tył, aby umożliwić rozprowadzenie środka uszczelniającego w oponie.
	\itemArrow Wężyk sprężarki powietrza ponownie przykręcić mocno na zaworze opony i powtórzyć pompowanie.
\end{itemizeArrow}

Ciśnienie w oponach

Czego należy przestrzegać
INFORMACJA
\begin{itemizeTriangle}
	\itemTriangle Zawsze dostosowuj ciśnienie w oponach do ładunku.
	\itemTriangle Ciśnienie w oponach (również w oponie koła zapasowego) należy sprawdzać co najmniej raz w miesiącu i dodatkowo przed każdą dłuższą podróżą.
	\itemTriangle Sprawdź ciśnienie na zimnych oponach. W rozgrzanych oponach jest wyższe ciśnienie – nie upuszczać powietrza.
	\itemTriangle Po każdej zmianie ciśnienia w oponach zapisz nowe wartości na wyświetlaczu sterowania oponami.
\end{itemizeTriangle}

OSTRZEŻENIE
Niebezpieczeństwo wypadku!
\begin{itemizeTriangle}
	\itemTriangle Jeśli utrata ciśnienia w oponie jest bardzo szybka, staraj się ostrożnie zatrzymać pojazd bez gwałtownych ruchów kierownicy lub gwałtownego hamowania.
\end{itemizeTriangle}

Naklejka z zalecanymi ciśnieniami w oponach

Naklejka z zalecanymi ciśnieniami w oponach znajduje się po wewnętrznej stronie klapki wlewu paliwa.
% FIXME: Rys.
A Ciśnienie dla połowicznego obciążenia
B Ciśnienie dla pełnego obciążenia
C Średnica opon w calach
Dane te służą jedynie jako informacja na temat zalecanego ciśnienia w oponach. Nie stanowią one wykazu opon dopuszczonych dla danego pojazdu. Informacje takie znajdują się w dokumentacji technicznej pojazdu oraz w deklaracji zgodności (tzw. dokumencie COC).
D Odczyt ciśnienia opon dla opon przedniej osi
E Odczyt ciśnienia opon dla opon tylnej osi

Wskaźnik ciśnienia w oponach

Wskaźnik kontroli opon wskazuje zmianę ciśnienia w oponach.

Ostrzeżenie, gdy zmienia się ciśnienie
XX.
\begin{itemizeArrow}
	\itemArrow Zatrzymać samochód.
	\itemArrow Sprawdź opony i ciśnienie w oponach.
\end{itemizeArrow}

Pokaż oponę ze zmienionym ciśnieniem

\begin{itemizeArrow}
	\itemArrow Wybierz pozycję menu do wyświetlania stanu opon w menu XX.
\end{itemizeArrow}

Ograniczenie funkcjonowania

System może nie zadziałać w przypadku nagłego spadku ciśnienia w oponie, np. z powodu przebicia.

Działanie systemu może być znacznie ograniczone lub system może nie działać np. w następujących sytuacjach:
\begin{itemizeTriangle}
	\itemTriangle Nierówne obciążenie kół, np. przy eksploatacji przyczepy
	\itemTriangle Jazda po drogach gruntowych
\end{itemizeTriangle}

Zapis wartości ciśnienia w oponach

Zapis wartości ciśnienia w oponach
\begin{itemizeArrow}
	\itemArrow Nacisnąć przycisk XX pod Infotainment.
	\itemArrow W wyświetlonym menu w systemie Infotainment stuknij obszar funkcji XX.
	\itemArrow Postępować zgodnie z dalszymi wskazówkami podawanymi na ekranie urządzenia Infotainment.
\end{itemizeArrow}

Zapis ciśnienia w oponach w następujących przypadkach:
\begin{itemizeTriangle}
	\itemTriangle Zmiana ciśnienia w oponach.
	\itemTriangle Wymiana jednego lub kilku kół
	\itemTriangle Zmiana pozycji jednego koła w pojeździe
	\itemTriangle Zawsze po przejechaniu dystansu 10 000 km lub raz w roku
\end{itemizeTriangle}

Rozwiązywanie problemów

System kontroli opon ma zakłócenie

XX miga przez około 1 minutę i pozostaje włączony

\begin{itemizeArrow}
	\itemArrow Zatrzymać pojazd, wyłączyć zapłon i ponownie uruchomić silnik. \\ Jeśli symbol XX po rozruchu silnika dalej miga, w systemie jest usterka.
	\itemArrow Niezwłocznie skorzystaj z pomocy specjalistycznej stacji obsługi.
\end{itemizeArrow}

Nakładki śrub kół

Zdejmowanie i zakładanie nakładek śrub

Odłączenie
% FIXME: Rys.
\begin{itemizeArrow}
	\itemArrow Nasunąć klucz klamrę do ściągania na nakładkę śruby koła.
	\itemArrow Ściągnąć kapturek.
\end{itemizeArrow}

Zamontowanie
\begin{itemizeArrow}
	\itemArrow Wsunąć nakładkę śrub do kół do oporu na śrubę koła.
\end{itemizeArrow}

Pełna osłona koła

Wymontowanie i zamontowanie kołpaka

Dotyczy kołpaków zamontowanych fabrycznie lub zakupionych jako oryginalne akcesoria ŠKODA.

Odłączenie
\begin{itemizeArrow}
	\itemArrow Zaczepić pałąk do zdejmowania kołpaków za krawędź jednego z otworów w kołpaku.
	\itemArrow Klucz do kół przełożyć przez pałąk, oprzeć o oponę i pociągnięciem zdjąć kołpak.
\end{itemizeArrow}

Zamontowanie
\begin{itemizeArrow}
	\itemArrow Nałożyć kołpak na obręcz, zaczynając przy wyprofilowaniu przewidzianym na zawór powietrza.
\end{itemizeArrow}

W przypadku użycia śruby zabezpieczającej przed kradzieżą, należy ją umieścić w miejscu wskazanym na kołpaku.
\begin{itemizeArrow}
	\itemArrow Następnie kołpak docisnąć, zaczynając od zaworu, do obręczy tak, aby zaczepy prawidłowo zaskoczyły na całym obwodzie.
\end{itemizeArrow}

INFORMACJA
\begin{itemizeTriangle}
	\itemTriangle Użyj nacisku dłoni, nie uderzaj w kołpak koła.
\end{itemizeTriangle}

Schowki i wyposażenie wnętrza

Wyposażenie w bagażniku

% FIXME: Rys.
A Zdejmowana przegroda w schowku \\ Obciążenie maks. 2,5 kg
B Dźwignia do składania oparć tylnych siedzeń
C Gniazdo zasilające 12-Volt
D Pod wykładziną podłogową: Schowek na kabel ładujący
E Przełącznik przyczepy
F Pod wykładziną podłogową: Schowek na elementy Cargo \\ Obciążenie wykładziny podłogowej maks. 75 kg

Wyposażenie awaryjne (w bagażniku)

% FIXME: Rys.

A Umieszczenie trójkąta ostrzegawczego (w zależności od wyposażenia)
B Umieszczenie apteczki pierwszej pomocy (w zależności od wyposażenia)
C Zestaw narzędzi
\begin{itemizeArrow}
	\itemArrow Aby uzyskać dostęp do zestawu narzędzi pojazdu, złożyć tylne siedzenie w prawo.
\end{itemizeArrow}

Schowek na kamizelkę odblaskową

Schowek na kamizelkę odblaskową znajduje się w schowku przednich drzwi.
% FIXME: Rys.

Gaśnica pod siedzeniem pasażera
% FIXME: Rys.

Zestaw narzędzi

% FIXME: Rys.
W zależności od wyposażenia nie wszystkie wymienione narzędzia muszą znajdować się w zestawie narzędzi.
A nasadka do śrub zabezpieczających przed kradzieżą koła
B gniazdo ucha holowniczego
C pałąk do ściągania kołpaków
D szczypce do zdejmowania kapturków śrub koła
E Zestaw do naprawy opon

Deklaracja zgodności jest załączona do podnośnika lub kompletu dokumentów.

Elementy mocujące w bagażniku
% FIXME: Rys.

% FIXME: Który wariant?

Wariant 1
A Hak do mocowania siatki mocującej
B Rozkładany podwójny hak do torby
Obciążenie maks. 5 kg z każdej strony podwójnego haka
C Łączniki do mocowania siatek mocujących
D Ucha do mocowania ładunku i siatek mocujących
Obciążenie maks. 350 kg
E Ucha do mocowania ładunku i siatek mocujących
Obciążenie maks. 350 kg
Wariant 2
A Rozkładany haczyk na torbę
Obciążenie maks. 7,5 kg
B Hak do mocowania siatki mocującej
C Łączniki do mocowania siatek mocujących
D Ucha do mocowania ładunku i siatek mocujących
Obciążenie maks. 350 kg
Inne elementy mocujące
A Elementy Cargo
Obciążenie maks. 8 kg
\begin{itemizeArrow}
	\itemArrow Złożyć element cargo i przymocować go do
\end{itemizeArrow}
wykładziny podłogowej w bagażniku.
Schowek na elementy Cargo znajduje się po
wewnętrznej stronie bocznego schowka lub
pod wykładziną podłogową w bagażniku.
B Elementy Cargo można usunąć ręcznie.

Haczyki na torby w bagażniku
% FIXME: Który wariant?

Wariant 1
Maksymalne obciążenie po obu stronach podwójnego haka wynosi każdorazowo 5 kg.
Wariant 2
Maksymalne dozwolone obciążenie haka wynosi 7,5 kg.


Schowek z elementami Cargo w bagażniku

% FIXME: Rys.

Maksymalne obciążenie elementów Cargo wynosi 8 kg.

Elementy Cargo w bagażniku

% FIXME: Rys.

Maksymalne obciążenie elementów Cargo wynosi 8 kg.

Siatki do mocowania bagażu

% FIXME: Rys.

% FIXME: Który wariant?

Wariant 1
Wariant 2
Maksymalne dozwolone obciążenie każdej siatki mocującej wynosi 1,5 kg.
Kieszeń wielofunkcyjna
Obsługa
Maksymalne dozwolone obciążenie haka wynosi 3
kg.
Rozkładanie
\begin{itemizeArrow}
	\itemArrow Odłącz tylną listwę, obracając w kierunku
\end{itemizeArrow}
strzałki.
\begin{itemizeArrow}
	\itemArrow Tylną listwę włożyć w
\end{itemizeArrow}
zagłębienia A .
INFORMACJA
Niebezpieczeństwo uszkodzenia pasów bezpieczeństwa.
\begin{itemizeTriangle}
	\itemTriangle Zawartość torby wielofunkcyjnej nie może wystawać poza górną krawędź listew.
	\begin{itemizeArrow}
		\itemArrow Rozłożyć przednie haki
	\end{itemizeArrow}
\end{itemizeTriangle}
po obu stronach bagażnika do dołu.
\begin{itemizeArrow}
	\itemArrow Oddziel listwę w tylnej
\end{itemizeArrow}
części, obracając ją do
przodu w kierunku
strzałki i umieść na hakach.
Składanie
\begin{itemizeArrow}
	\itemArrow Zdjąć tylną listwę z zagłębień.
	\itemArrow Umieść tylną listwę na
\end{itemizeArrow}
przedniej szynie i złóż
obie części razem, obracając.
\begin{itemizeArrow}
	\itemArrow Zdjąć tylną listwę z haków.
	\itemArrow Umieść tylną listwę na
\end{itemizeArrow}
przedniej szynie i złóż
obie części razem, obracając.
Wyjmowanie i wkładanie
Wyjmowanie
\begin{itemizeArrow}
	\itemArrow Wyjmij pokrywę bagażnika.
\end{itemizeArrow}
Wariant 1
Wariant 2
Wariant 1
Wariant 2
186 Schowki i wyposażenie wnętrza › Elementy Cargo w bagażniku
\begin{itemizeArrow}
	\itemArrow Wyjmij złożoną torbę do góry.
\end{itemizeArrow}
Wkładanie
\begin{itemizeArrow}
	\itemArrow Koniec listwy A nałożyć na uchwyt B .
	\itemArrow Postępuj analogicznie
\end{itemizeArrow}
po lewej stronie.
\begin{itemizeArrow}
	\itemArrow Włożyć końcówkę listwy oznaczoną w
\end{itemizeArrow}
prawe mocowanie, a
końcówkę listwy oznaczoną w lewe mocowanie.


Siatka oddzielająca
% FIXME: Czy posiadamy?

Mocowanie siatki działowej
Rozkładanie / składanie
\begin{itemizeArrow}
	\itemArrow Otwórz ramiona poprzeczki, aż usłyszysz
\end{itemizeArrow}
kliknięcie przycisku
blokującego.
\begin{itemizeArrow}
	\itemArrow Naciśnij przycisk blokady i złóż ramiona poprzeczki razem.
\end{itemizeArrow}
Składaną przegrodę siatkową można schować
pod podłogą o zmiennej wysokości.
Wariant 1
Wariant 2
Mocowanie z tyłu
\begin{itemizeArrow}
	\itemArrow Aby zamontować, wyjąć zwijaną osłonę bagażnika
\end{itemizeArrow}
lub złożyć do przodu oparcia kanapy tylnej .
\begin{itemizeArrow}
	\itemArrow Poprzeczkę wsunąć w uchwyt A po jednej stronie
\end{itemizeArrow}
i docisnąć ją w przód.
\begin{itemizeArrow}
	\itemArrow Zamocować pręt w ten sam sposób po drugiej
\end{itemizeArrow}
stronie.
\begin{itemizeArrow}
	\itemArrow Zatrzasnąć karabińczyki B w zaczepach.
	\itemArrow Napiąć pasy, pociągając za swobodne końcówki C .
\end{itemizeArrow}
Mocowanie z przodu
Procedura jest analogiczna do procedury z tyłu.
Zaczepy do zahaczenia karabińczyków znajdują się
poniżej środkowych słupków nadwozia.
Odkręcanie
Wyjmowanie odbywa się w kolejności odwrotnej do
mocowania.


Sztywna pokrywa bagażnika

Wyjmowanie i wkładanie

Maksymalne dozwolone obciążenie osłony wynosi 1 kg.

Wyjmowanie
% FIXME: Rys.
\begin{itemizeArrow}
	\itemArrow Odczep taśmy mocujące.
\end{itemizeArrow}

% FIXME: Rys.
\begin{itemizeArrow}
	\itemArrow Przytrzymać podniesioną osłonę i nacisnąć po obu stronach jej dolną część po obu stronach
	\itemArrow Wyjąć półkę.
\end{itemizeArrow}

Wkładanie
% FIXME: Rys.
\begin{itemizeArrow}
	\itemArrow Mocowanie A naprzeciwko uchwytu B włożyć o obu stronach bagażnika.
	\itemArrow Naciśnij pokrywę od góry, aż się zatrzaśnie.
	\itemArrow Zaczepić paski mocujące.
\end{itemizeArrow}

Przechowuj osłonę za tylnymi siedzeniami
% FIXME: Rys.
\begin{itemizeArrow}
	\itemArrow Wcisnąć osłonę między tylnymi siedzeniami a uchwytem A .
\end{itemizeArrow}

Zwijana pokrywa bagażnika

Obsługa

Rozwijanie
% FIXME: Rys.
\begin{itemizeArrow}
	\itemArrow Włóż osłonę, aż zatrzaśnie się na swoim miejscu.
\end{itemizeArrow}

Zwijanie
% FIXME: Rys.
\begin{itemizeArrow}
	\itemArrow Naciśnij osłonę w obszarze uchwytu.
\end{itemizeArrow}

Osłona zwija się automatycznie do pozycji pośredniej A .

Naciśnij ponownie, aby całkowicie otworzyć osłonę.

Istnieje możliwość, że zwijana osłona bagażnika w warunkach zimowych będzie się zwijać wolniej.


Ustawienia

Automatyczne nawijanie osłony do pozycji pośredniej

Po otwarciu bagażnika osłona automatycznie zwija się do pozycji pośredniej.

Dezaktywacja lub aktywacja funkcji odbywa się w systemie Infotainment w poniższym menu:
XX na zewnątrz Otwieranie i zamykanie
Lub:
XX na zewnątrz
\begin{itemizeArrow}
	\itemArrow Przesuwając palec bokiem do ekranu, wybierz element menu Centralne ryglowanie .
	\itemArrow Wybierz pozycję menu Centralne ryglowanie .
\end{itemizeArrow}

Wyjmowanie i wkładanie

Wyjmij i włóż zwiniętą pokrywę
% FIXME: Rys.

\begin{itemizeArrow}
	\itemArrow Naciśnij koniec poprzeczki i wyjmij lub włóż pokrywę.
\end{itemizeArrow}

Schować osłonę pod wykładziną podłogową w bagażniku
% FIXME: Rys.
\begin{itemizeArrow}
	\itemArrow Podnieść tylną część wykładziny podłogowej.
	\itemArrow Usuń boczne przegrody schowków.
	\itemArrow Włożyć boczne ścianki do schowka na kabel ładujący.
	\itemArrow Włożyć osłonę w tylne zagłębienia bocznych ścianek.
	\itemArrow Złożyć wolną część osłony do przodu za pomocą rączki.
\end{itemizeArrow}


Wyposażenie wewnętrzne z przodu

% FIXME: Rys.

A Lusterko podręczne
B Uchwyt na bilet parkingowy
C Uchwyt na bilet parkingowy
D Schowek na okulary \\ Obciążenie maks. 0,25 kg
\begin{itemizeArrow}
	\itemArrow Aby otworzyć, wcisnąć przycisk.
\end{itemizeArrow}
E Złącze USB
Port USB służy tylko do ładowania.
F Schowek z dyszą wylotową powietrza
Obciążenie maks. 3 kg
W schowku znajduje się uchwyt na karty i uchwyt na pisak.
\begin{itemizeArrow}
	\itemArrow Aby otworzyć schowek, pociągnij za uchwyt.
	\itemArrow Aby otworzyć dyszę wylotową powietrza, obróć pokrętło do położenia XX.
\end{itemizeArrow}
G Schowek
\begin{itemizeTriangle}
	\itemTriangle Schowek na butelkę o maks. pojemności 1,5 l
	\itemTriangle Schowek na pojemnik na odpady
	\itemTriangle Schowek na kamizelkę odblaskową
\end{itemizeTriangle}
H Schowek (w zależności od wyposażenia pojazdu):
\begin{itemizeTriangle}
	\itemTriangle Phonebox
	\itemTriangle Złącza USB
\end{itemizeTriangle}
Złącza USB mogą służyć do ładowania oraz transmisji danych.
I Schowek
W schowku znajduje się uchwyt na napoje.
\begin{itemizeArrow}
	\itemArrow Aby otworzyć schowek, pociągnij za listwę.
\end{itemizeArrow}
J Otwierany i regulowany podłokietnik ze schowkiem
\begin{itemizeArrow}
	\itemArrow Aby otworzyć schowek, podnieś podłokietnik.
\end{itemizeArrow}
K Schowek
Obciążenie maks. 0,5 kg
W schowku znajduje się uchwyt na karty.
\begin{itemizeArrow}
	\itemArrow Pociągnij za uchwyt, aby otworzyć.
\end{itemizeArrow}

INFORMACJA
\begin{itemizeTriangle}
	\itemTriangle W komorze na okulary D nie pozostawiać żadnych wrażliwych na ciepło przedmiotów.
\end{itemizeTriangle}

Wyposażenie wewnętrzne z tyłu

% FIXME: Rys.

A Haczyki na ubrania \\ Obciążenie maks. 2 kg
B Schowek
\begin{itemizeTriangle}
	\itemTriangle Schowek na butelkę o maks. pojemności 1,5 l
	\itemTriangle Schowek na kamizelkę odblaskową
\end{itemizeTriangle}
C Kieszeń
\begin{itemizeTriangle}
	\itemTriangle Kieszeń na telefon
\end{itemizeTriangle}
D W zależności od wyposażenia samochodu:
\begin{itemizeTriangle}
	\itemTriangle Gniazdo 230 V i port USB
	\itemTriangle Złącza USB
	\itemTriangle Schowek
\end{itemizeTriangle}

OSTRZEŻENIE
\begin{itemizeTriangle}
	\itemTriangle Zawieszaj tylko lekkie ubrania na haku. W kieszeniach wieszanej odzieży nie zostawiać ciężkich ani ostrych przedmiotów.
	\itemTriangle Do zawieszania odzieży nie używać wieszaków.
\end{itemizeTriangle}

Kieszeń do przechowywania telefonu
% FIXME: Rys.

Uchwyt na bilet parkingowy
% FIXME: Rys.

Stojak na butelki w schowku przednich drzwi
% FIXME: Rys.

Półka przeznaczona jest na butelki o zawartości maks. 1,5 l.

Półka na butelki w schowku tylnych drzwi

% FIXME: Rys.

Półka przeznaczona jest na butelki o zawartości maks. 1,5 l.

Złącza USB
% FIXME: Rys.
Złącze USB służy tylko do ładowania.
% FIXME: Rys.
Złącza USB mogą służyć do ładowania oraz transmisji danych.
% FIXME: Rys.
Złącza USB służą tylko do ładowania.

Hak na środkowym słupku nadwozia
% FIXME: Rys.
Maksymalne dozwolone obciążenie haka wynosi 2 kg.

OSTRZEŻENIE
\begin{itemizeTriangle}
	\itemTriangle Zawieszaj tylko lekkie ubrania na haku. W kieszeniach wieszanej odzieży nie zostawiać ciężkich ani ostrych przedmiotów.
	\itemTriangle Do zawieszania odzieży nie używać wieszaków.
\end{itemizeTriangle}

Schowek na okulary
% FIXME: Rys.
Maksymalne dozwolone obciążenie schowka wynosi 0,25 kg.

INFORMACJA
\begin{itemizeTriangle}
	\itemTriangle W schowku na okulary nie pozostawiać żadnych przedmiotów wrażliwych na ciepło.
\end{itemizeTriangle}

Schowek pod fotelem przednim

Otwieranie schowka

% FIXME: Rys.
Maksymalne dozwolone obciążenie schowka wynosi 1,5 kg.

Schowek na telefon

Czego należy przestrzegać

OSTROŻNIE
Ryzyko poparzenia podczas ładowania.
\begin{itemizeTriangle}
	\itemTriangle Telefon może się rozgrzać, wyjmij go ostrożnie z schowka.
	\itemTriangle Nie zostawiaj żadnych metalowych przedmiotów w schowku pod telefonem. Jeżeli w schowku znajduje się rozgrzany metalowy przedmiot, należy wyjąć telefon, a rozgrzany przedmiot pozostawić w schowku do ostygnięcia!
\end{itemizeTriangle}

Sposób działania

Funkcje Phonebox
\begin{itemizeTriangle}
	\itemTriangle Bezprzewodowe ładowanie telefonów.
	\itemTriangle Wzmocnienie sygnału telefonu (dotyczy tylko niektórych krajów)
\end{itemizeTriangle}

Phonebox znajduje się w schowku w konsoli środkowej z przodu.

Do schowka można włożyć telefon o wymiarach maksymalnie 160 × 80 mm.

Wskaźnik ładowania

Stan naładowania jest wskazywany przez tekst na ekranie Infotainment.

Zalecenia dotyczące optymalnej funkcji
\begin{itemizeTriangle}
	\itemTriangle Telefon odkładać telefon, kierując ekran do góry.
	\itemTriangle Włóż telefon bez etui ochronnego.
	\itemTriangle Połóż telefon na środku na symbolu telefonu na wsporniku.
\end{itemizeTriangle}

Warunki działania

Warunki bezprzewodowego ładowania
\begin{itemizeTick}
	\itemTick Zapłon jest włączony.
	\itemTick Telefon obsługuje standard Qi.
	\itemTick Między nakładką a telefonem nie ma żadnych obiektów.
\end{itemizeTick}

Rozwiązywanie problemów

Ekran systemu informacyjnego wyświetlił komunikat, że telefonu komórkowego nie można naładować.

Symbol XX świeci się na pasku stanu wraz z XX.
\begin{itemizeTriangle}
	\itemTriangle Sprawdzić czy pomiędzy powierzchnią ładowania a telefonem nie znajduje się żaden przedmiot. W takim przypadku usuń telefon i przedmiot. Telefon ponownie położyć na środku powierzchni ładowania, na symbolu telefonu.
	\itemTriangle Sprawdzić, czy pozycja telefonu nie uległa zmianie podczas jazdy. Jeżeli wystąpi taka sytuacja, należy wyjąć telefon i ponownie położyć go na środku powierzchni ładowania, na symbolu telefonu.
\end{itemizeTriangle}

Uchwyt na napoje z tyłu

% FIXME: Rys.

\begin{itemizeArrow}
	\itemArrow Aby otworzyć, otwórz pokrywę.
\end{itemizeArrow}

Uchwyt na urządzenia multimedialne

% FIXME: Rys.

Uchwyt multimediów jest umieszczony w uchwycie na kubek.
A Schowek na telefon komórkowy
B Schowek na monety

Pojemnik na odpady

Wymiana worka

% FIXME: Rys.

Sufit

Wyjmowanie i wkładanie

Koc znajduje się w torbie, która może być przymocowana w jednym z poniższych miejsc:
\begin{itemizeTriangle}
	\itemTriangle Na prowadnicach zagłówków przednich
	\itemTriangle W kieszeniach z tyłu oparć foteli przednich
\end{itemizeTriangle}

Uchwyt na tablet


Ustawienia

Przechyl i obróć
% FIXME: Rys.

Dostosuj rozmiar
% FIXME: Rys.

\begin{itemizeArrow}
	\itemArrow Wyciągnij zaczep blokujący i przesuń górną część uchwytu do żądanej pozycji.
\end{itemizeArrow}

Ustaw minimalny rozmiar pustego uchwytu, aby uniknąć hałasu podczas jazdy.

Zdejmowanie i wkładanie

Zamocuj za zagłówkami
% FIXME: Rys.
\begin{itemizeArrow}
	\itemArrow Aby założyć, umieścić otwarty adapter na prowadnicach zagłówka z przodu i ostrożnie wcisnąć.
\end{itemizeArrow}
% FIXME: Rys.
\begin{itemizeArrow}
	\itemArrow Włożyć uchwyt do adaptera.
\end{itemizeArrow}

Zdejmowanie
% FIXME: Rys.
\begin{itemizeArrow}
	\itemArrow Naciśnij przycisk bezpieczeństwa i wyjmij uchwyt.
\end{itemizeArrow}
% FIXME: Rys.
\begin{itemizeArrow}
	\itemArrow Naciśnij adapter i wyjmij go.
\end{itemizeArrow}

Włóż w tylny podłokietnik
% FIXME: Rys.
\begin{itemizeArrow}
	\itemArrow Zapnij uchwyt w otworze.
\end{itemizeArrow}

Zdejmowanie
% FIXME: Rys.
\begin{itemizeArrow}
	\itemArrow Naciśnij przycisk bezpieczeństwa i wyjmij uchwyt.
\end{itemizeArrow}

Dane techniczne

Uchwyt służy do mocowania tabletu o wysokości min. 11,5 cm i maks. 19,5 cm.

Maksymalne dozwolone obciążenie uchwytu wynosi 0,75 kg.

Gniazdo 12 V

\begin{itemizeTriangle}
	\itemTriangle Z gniazda elektrycznego można zasilać wyłącznie dopuszczony osprzęt elektryczny o łącznej mocy do 120 W.
	\itemTriangle Wyłącz odbiorniki przed włączeniem lub wyłączeniem zapłonu i przed uruchomieniem silnika.
\end{itemizeTriangle}

Gniazdo 230 V

% FIXME: Rys.

\begin{itemizeTriangle}
	\itemTriangle Nie podłączaj opraw z lampą fluorescencyjną do gniazda.
	\itemTriangle Wyłącz odbiorniki przed włączeniem lub wyłączeniem zapłonu i przed uruchomieniem silnika.
\end{itemizeTriangle}

Wskaźnik stanu

\begin{itemizeTriangle}
	\itemTriangle Świeci na zielono - gniazdo jest włączone.
	\itemTriangle Miga na zielono - zasilanie jest włączone przez około 10 minut po zatrzymaniu silnika, o ile odbiornik był podłączony przed zatrzymaniem silnika.
\end{itemizeTriangle}

Gniazdo 230 V posiada zabezpieczenie przed dziećmi. Podczas wkładania wtyczki bezpiecznik jest odblokowany i gniazdo jest włączone.

Warunki działania
\begin{itemizeTick}
	\itemTick Silnik pracuje
\end{itemizeTick}

Rozwiązywanie problemów

Kontrolka miga na czerwono

Gniazdo to jest wyłączone np. z następujących powodów:
\begin{itemizeTriangle}
	\itemTriangle nadmiernego poboru prądu,
	\itemTriangle niskiego poziomu naładowania akumulatora pojazdu 12 V,
	\itemTriangle przegrzania gniazda.
\end{itemizeTriangle}

\begin{itemizeArrow}
	\itemArrow Jeśli powyższe przyczyny nie mają już zastosowania, a gniazdo nie włącza się automatycznie, należy odłączyć podłączony odbiornik od gniazda i ponownie podłączyć po krótkim czasie.
\end{itemizeArrow}

Schowek z dyszą wylotową powietrza

Maksymalne dozwolone obciążenie schowka wynosi 3 kg.
\begin{itemizeArrow}
	\itemArrow Aby otworzyć dyszę wylotową powietrza, obróć pokrętło w schowku do położenia XX.
\end{itemizeArrow}

Bagażnik dachowy i hak holowniczy

Bagażnik dachowy

Dane techniczne

Maksymalnie dopuszczalna masa ładunku łącznie z bagażnikiem wynosi 75 kg.

Poprzeczki dachowe można mocować do relingów dachowych.

Zdejmowany zaczep do holowania

Czego należy przestrzegać

OSTRZEŻENIE
\begin{itemizeTriangle}
	\itemTriangle Przed każdą jazdą z założonym zaczepem kulowym sprawdzić, czy jest on prawidłowo i stabilnie zamocowany w gnieździe.
	\itemTriangle Nie używaj zaczepu kulowego, jeśli jest uszkodzony lub niekompletny.
\end{itemizeTriangle}

OSTRZEŻENIE
Niebezpieczeństwo wypadku!
Zabrudzenie uniemożliwia prawidłowe zamocowanie zaczepu kulowego!
\begin{itemizeTriangle}
	\itemTriangle Utrzymywać gniazdo zaczepu holowniczego w czystości.
\end{itemizeTriangle}

Czyszczenie i pielęgnacja

Zdarzenia serwisowe

Okresy międzyobsługowe

Przestrzeganie okresów międzyobsługowych ma decydujące znaczenie dla okresu eksploatacji i zachowania wartości samochodu.

W dniu wymagalności usługi pojawi się symbol i odpowiedni komunikat na wyświetlaczu zestawu wskaźników.

Informacji na temat typu okresu międzyobsługowego, opcji jego zmiany oraz zakresów serwisowania udziela specjalistyczna stacja obsługi.

Wszystkie usługi serwisowe i wymiana lub uzupełnianie płynów eksploatacyjnych są płatne, także w okresie gwarancyjnym, chyba że postanowienia gwarancyjne ŠKODA AUTO lub inne wiążące uzgodnienia stanowią inaczej.

Książka serwisowa

Specjalistyczna stacja obsługi potwierdza wykonanie przeglądów w serwisowym systemie informacji o nazwie Cyfrowa książka serwisowa.

Dowód przeglądu może zostać wydrukowany.

Pokaż termin przeglądu

\begin{itemizeArrow}
	\itemArrow Menu w Infotainment XX Wybierz pozycję menu dotyczącą serwisu.
\end{itemizeArrow}

Zresetuj informacje

Zalecamy, aby nie resetować informacji dotyczących zdarzeń serwisowych samodzielnie. W przeciwnym razie może dojść do błędnego ustawienia wskaźnika okresów międzyobsługowych i na skutek tego także do ewentualnych awarii pojazdu.

W samochodach ze zmiennym okresem międzyobsługowym po zresetowaniu wskaźnika wymiany oleju w specjalistycznej stacji obsługi wyświetlane będą nowe wartości okresu międzyobsługowego określone według poprzednich warunków eksploatacji pojazdu. Wartości te będą na bieżąco dostosowywane zgodnie z aktualnymi warunkami eksploatacji pojazdu.

Prace serwisowe, korekty i zmiany techniczne

Podczas używania akcesoriów oraz wykonywania dostosowań, napraw lub zmian technicznych samochodu należy przestrzegać wskazówek i wytycznych ŠKODA AUTO.

Wnętrze samochodu

Czego należy przestrzegać

INFORMACJA
\begin{itemizeTriangle}
	\itemTriangle Używaj środków czyszczących przeznaczonych do czyszczenia i pielęgnacji poszczególnych materiałów.
	\itemTriangle Nie używaj agresywnych środków czyszczących ani rozpuszczalników chemicznych.
\end{itemizeTriangle}
Naturalna skóra / sztuczna skóra / Alcantara ® / Suedia / tkanina

INFORMACJA
\begin{itemizeTriangle}
	\itemTriangle Usuń zanieczyszczenia tak szybko, jak to możliwe.
	\itemTriangle Do obić z materiału Alcantara® i Suedia nie używać środków do czyszczenia skóry, pasty do podłóg, pasty do butów, wywabiaczy do plam itp.
	\itemTriangle Upewnij się, że naturalna skóra nie jest zwilżona podczas czyszczenia i że woda nie przedostaje się do szwów.
	\itemTriangle Nie myć okładziny dachu za pomocą szczotki.
\end{itemizeTriangle}

INFORMACJA
Ryzyko blaknięcia tkanin pokryciowych!
\begin{itemizeTriangle}
	\itemTriangle Nie należy pozostawiać pojazdu przez dłuższy czas w bezpośrednim słońcu. W razie potrzeby należy chronić obicia, przykrywając je.
	\itemTriangle Przy użytkowaniu samochodu na elementach ze skóry i materiału Alcantara® i Suedia mogą się pojawić niewielkie widoczne zmiany (np. fałdki, przebarwienia). Nie oznaczają one braków materiałowych.
	\itemTriangle Niektóre tkaniny odzieżowe, np. ciemny dżins, czasami nie mają wystarczającej trwałości kolorów. W rezultacie może wystąpić odbarwienie widoczne na pokrowcach. Przyczyną nie są wady obicia.
	\itemTriangle Ostre suwaki, nity, klamry i podobne elementy odzieży mogą uszkodzić tapicerkę w pojeździe. Takich uszkodzeń nie można uznać za uzasadniony powód reklamacji.
\end{itemizeTriangle}

Obicia siedzeń ogrzewanych elektrycznie
INFORMACJA
Niebezpieczeństwo uszkodzenia systemu grzewczego!
\begin{itemizeTriangle}
	\itemTriangle Nie czyść foteli wodą ani innymi płynami.
	\itemTriangle Nie suszyć foteli poprzez włączanie ogrzewania.
\end{itemizeTriangle}

Instrukcje dotyczące czyszczenia

Naturalna skóra / sztuczna skóra / Alcantara ® / Suedia / tkanina

\begin{itemizeArrow}
	\itemArrow Usuń kurz i brud z powierzchni za pomocą odkurzacza.
	\itemArrow Usuń świeże zanieczyszczenia wodą, lekko zwilżoną bawełnianą szmatką lub wełnianą szmatką, w razie potrzeby łagodnym roztworem mydła\footnote{Łagodny roztwór mydła to 2 łyżki stołowe białego naturalnego mydła na 1 litr letniej wody.} i wytrzyj suchą szmatką.
	\itemArrow Usuń uporczywe plamy za pomocą odpowiedniego środka czyszczącego.
	\itemArrow Do regularnej pielęgnacji skóry naturalnej używaj środków przewidzianych do tego celu. Po każdym czyszczeniu używaj kremu do pielęgnacji z lekką ochroną i efektem impregnacji.
	\itemArrow Podczas pielęgnacji materiału Alcantara®, Suedia i powierzchni tkanin usuwają uporczywe włosy za pomocą rękawicy czyszczącej. Pęczki na tkaninach usunąć szczotką.
\end{itemizeArrow}

Elementy z tworzyw sztucznych

\begin{itemizeArrow}
	\itemArrow W razie potrzeby usuń zanieczyszczenia za pomocą wody, lekko zwilżonej szmatki lub gąbki za pomocą odpowiedniego środka czyszczącego.
\end{itemizeArrow}

Szyby

\begin{itemizeArrow}
	\itemArrow Usunąć zanieczyszczenia czystą wodą i wysuszyć szmatką do tego przeznaczoną.
\end{itemizeArrow}

Ekran urządzenia Infotainment

\begin{itemizeArrow}
	\itemArrow Usuń zanieczyszczenia z ekranu za pomocą dostarczonych środków czyszczących.
\end{itemizeArrow}

INFORMACJA
Niebezpieczeństwo uszkodzenia ekranu!
\begin{itemizeTriangle}
	\itemTriangle Podczas usuwania brudu nie naciskaj na ekran.
\end{itemizeTriangle}

Obicia siedzeń ogrzewanych elektrycznie

\begin{itemizeArrow}
	\itemArrow Usuń zanieczyszczenia za pomocą odpowiedniego środka czyszczącego.
\end{itemizeArrow}

Pasy bezpieczeństwa

\begin{itemizeArrow}
	\itemArrow Usuń zanieczyszczenia miękką ściereczką i łagodnym roztworem mydła\footnote{Łagodny roztwór mydła to 2 łyżki stołowe białego naturalnego mydła na 1 litr letniej wody.}.
\end{itemizeArrow}

Na zewnątrz

Czego należy przestrzegać

OSTRZEŻENIE
Niebezpieczeństwo wypadku!
Po umyciu pojazdu na działanie układu hamulcowego może wpływać wilgoć, a zimą lód.
\begin{itemizeTriangle}
	\itemTriangle Suszyć i czyścić hamulce kilkakrotnie hamując.
\end{itemizeTriangle}

INFORMACJA
\begin{itemizeTriangle}
	\itemTriangle Ptasie odchody, pozostałości owadów, śmieci i pozostałości soli morskiej, rozlane paliwo, usunąć tak szybko, jak to możliwe.
	\itemTriangle Do usuwania zabrudzeń nie używaj szorstkich gąbek, skrobaków itp.
	\itemTriangle Używaj środków czyszczących przeznaczonych do czyszczenia i pielęgnacji poszczególnych materiałów.
	\itemTriangle Nie używaj agresywnych środków czyszczących ani rozpuszczalników chemicznych.
	\itemTriangle Nie poleruj pojazdu w zakurzonym otoczeniu.
\end{itemizeTriangle}

INFORMACJA
\begin{itemizeTriangle}
	\itemTriangle Uszkodzone miejsca należy możliwie jak najszybciej naprawiać.
	\itemTriangle Na częściach lakierowanych matowo nie stosować środków polerujących ani wosków stałych.
	\itemTriangle Nie poleruj folii.
	\itemTriangle Zalecamy traktowanie uszczelek drzwi i prowadnic okien odpowiednimi środkami z oryginalnych akcesoriów ŠKODA. Zapewniają one, że warstwa lakieru ochronnego uszczelek i prowadnic okiennych nie zostanie zaatakowana.
	\itemTriangle Nie używaj ściernych środków czyszczących do czyszczenia kamery cofania.
\end{itemizeTriangle}

Przed przejazdem przez myjnię samochodową
\begin{itemizeArrow}
	\itemArrow Przestrzegaj zwyczajowych zaleceń myjni, np. zamknij wszystkie okna, złóż lusterka itp.
	\itemArrow Przesuń ramiona wycieraczek szyby przedniej do pozycji XX.
	\itemArrow Jeśli w Twoim pojeździe znajdują się specjalne dodatki, przestrzegaj instrukcji operatora myjni.
\end{itemizeArrow}

INFORMACJA
Przed przejazdem przez myjnię samochodową należy spełnić następujące warunki, jeśli konieczne jest przetoczenie pojazdu.
\begin{itemizeTick}
	\itemTick Zapłon włączony
	\itemTick Dźwignia zmiany biegów w trybie \gearN
	\itemTick Elektryczny hamulec postojowy wyłączony
	\itemTick Auto Hold Funkcja nieaktywna
\end{itemizeTick}

INFORMACJA
W pojazdach z elektryczną pokrywą bagażnika może ona otwierać się automatycznie z powodu nacisku szczotek myjących.
\begin{itemizeTriangle}
	\itemTriangle Zamknąć pojazd, np. blokadą centralnego zamka.
\end{itemizeTriangle}

Po umyciu z konserwacją woskiem
\begin{itemizeArrow}
	\itemArrow Wytrzyj wycieraczki szyby przedniej suchą szmatką.
\end{itemizeArrow}

Mycie myjką ciśnieniową

INFORMACJA
\begin{itemizeTriangle}
	\itemTriangle Przestrzegać instrukcji obsługi myjki wysokociśnieniowej, w szczególności wskazówek dotyczących ciśnienia i odległości natrysku od powierzchni pojazdu.
	\itemTriangle Nie kieruj strumienia wody bezpośrednio na następujące części pojazdu.
	\itemTriangle Elementy układu wysokiego napięcia, np. gniazdo ładowania, kabel wysokiego napięcia itp.
	\itemTriangle Folie
	\itemTriangle Wkładka zamka
	\itemTriangle Szczeliny w pojeździe
	\itemTriangle Gniazdo przyczepy
	\itemTriangle Obrotowy drążek kulowy
	\itemTriangle Obiektywy i czujniki aparatu
	\itemTriangle Części z tworzyw sztucznych, chromowane i anodowane
\end{itemizeTriangle}

Usuwanie śniegu i lodu

INFORMACJA
\begin{itemizeTriangle}
	\itemTriangle Usuń śnieg i lód za pomocą plastikowego skrobaka lub odpowiedniego środka przeciwoblodzeniowego.
	\itemTriangle Wyczyść kamery za pomocą szczotki ręcznej.
	\itemTriangle Skrobak przesuwać tylko w jednym kierunku.
	\itemTriangle Nie używaj skrobaków ani innych ostrych przedmiotów do folii.
	\itemTriangle Nie usuwaj śniegu i lodu ciepłą lub gorącą wodą.
	\itemTriangle Nie usuwaj śniegu i lodu z powierzchni o dużym zabrudzeniu.
\end{itemizeTriangle}

Instrukcje dotyczące czyszczenia

Wskazówki na temat układu wysokiego napięcia

\begin{itemizeArrow}
	\itemArrow Zakończyć ładowanie i całkowicie zamknąć klapkę gniazda ładowania.
	\itemArrow Wyłączyć zapłon.
	\itemArrow Elementy układu wysokiego napięcia, np. pomarańczowe kable, nie mogą być uszkodzone.
\end{itemizeArrow}

Mycie ręczne

\begin{itemizeArrow}
	\itemArrow Samochód należy myć od góry do dołu, miękką gąbką lub rękawicą do mycia, używając dużej ilości wody, ewentualnie z dodatkiem odpowiednich środków czyszczących.
	\itemArrow W przypadku folii i reflektorów użyj łagodnego roztworu mydła zawierającego dwie łyżki białego neutralnego mydła na 1 litr letniej wody.
	\itemArrow Do mycia piór wycieraczek szyby przedniej użyj środka do mycia szyb.
	\itemArrow Kamery myć czystą wodą i wysuszyć odpowiednią czystą ściereczką.
\end{itemizeArrow}

INFORMACJA
\begin{itemizeTriangle}
	\itemTriangle Nie myj pojazdu przy intensywnym nasłonecznieniu.
	\itemTriangle Nie wywieraj nacisku na nadwozie podczas mycia.
	\itemTriangle Temperatura wody do mycia może wynosić maks. 60°C
\end{itemizeTriangle}

Po myciu ręcznym
\begin{itemizeArrow}
	\itemArrow Wypłucz pojazd i wytrzyj go odpowiednią czystą szmatką.
\end{itemizeArrow}

Lakier
\begin{itemizeArrow}
	\itemArrow Lakier należy konserwować co najmniej dwa razy w roku za pomocą twardego wosku.
	\itemArrow Użyj środka polerującego do matowych lakierów.
\end{itemizeArrow}

Folie

Folie starzeją się i kruszą, to zupełnie normalne, nie stanowi to wady.

Następujące czynniki mają negatywny wpływ na żywotność lub trwałość koloru folii:
\begin{itemizeTriangle}
	\itemTriangle Promieniowanie słoneczne
	\itemTriangle szkód spowodowanych przez wilgoć
	\itemTriangle Zanieczyszczenie powietrza
	\itemTriangle Uderzenia kamieni, np. przez odbicie od ładunku podczas transportu na bagażniku dachowym
\end{itemizeTriangle}

Konserwacja profili zamkniętych

Wnęki pojazdu podatne na korozję są fabrycznie trwale zabezpieczone woskiem konserwującym.

\begin{itemizeArrow}
	\itemArrow Usuń rozlany wosk za pomocą plastikowego skrobaka, wyczyść plamy za pomocą benzyny ekstrakcyjnej.
\end{itemizeArrow}

Ochrona podwozia

Podwozie samochodu jest trwale zabezpieczone przed wpływami chemicznymi i mechanicznymi.

\begin{itemizeArrow}
	\itemArrow Przed rozpoczęciem i po zimowym sezonie należy sprawdzić powłokę ochronną w specjalistycznej stacji obsługi.
\end{itemizeArrow}

Koła
\begin{itemizeArrow}
	\itemArrow Po umyciu należy zabezpieczyć koła odpowiednim środkiem.
\end{itemizeArrow}

OSTRZEŻENIE
Niebezpieczeństwo wypadku!
\begin{itemizeTriangle}
	\itemTriangle Silne zabrudzenie kół może powodować ich niewyważenie.
\end{itemizeTriangle}

Lewarek
\begin{itemizeArrow}
	\itemArrow Jeśli to konieczne, pokryć ruchome części odpowiednim smarem.
\end{itemizeArrow}

Hak holowniczy
\begin{itemizeArrow}
	\itemArrow Głowicę kulistą zaczepu holowniczego powlec w razie potrzeby odpowiednim smarem.
\end{itemizeArrow}

Skrobaczka do lodu

Na klapce wlewu paliwa
% FIXME: Rys.


Dane techniczne i przepisy

Specyfikacje dotyczące danych technicznych

Normę emisji spalin, informacje o zużyciu paliwa i inne informacje ważne dla pojazdu można znaleźć w dokumentacji pojazdu technicznego oraz w deklaracji zgodności (w tzw.dokumencie COC).

Wymienione informacje zostały ustalone zgodnie z przepisami i na warunkach określonych przez przepisy prawne lub techniczne.

Te i inne informacje na temat pojazdu i deklaracji zgodności można uzyskać na stronie partnera ŠKODA.

Informacje w dokumentach technicznych i deklaracji zgodności pojazdu mają pierwszeństwo przed informacjami opublikowanymi w tej instrukcji obsługi. Te specyfikacje i wartości dotyczą pojazdu w stanie i konfiguracji w momencie dostawy od producenta.

Dane samochodu

Numer identyfikacyjny samochodu (VIN)

Numer identyfikacyjny pojazdu znajduje się w następujących miejscach:
\begin{itemizeTriangle}
	\itemTriangle Po prawej stronie w komorze silnika na gnieździe kolumny McPhersona
	\itemTriangle Na tabliczce pod przednią szybą w lewym dolnym rogu
	\itemTriangle Na tabliczce znamionowej na dole środkowego słupka nadwozia pojazdu
\end{itemizeTriangle}

% FIXME: Rys.
Tabliczka znamionowa

A Nazwa producenta pojazdu
B Numer identyfikacyjny samochodu (VIN)

Wskazanie VIN

Wyświetlanie pliku VIN odbywa się w systemie Infotainment pod punktem menu XX Punkt menu dotyczący serwisu.

Numer silnika

Numer silnika jest wybity na kadłubie silnika.

Maksymalne dopuszczalne ciężary

Maksymalne dopuszczalne masy są podane na tabliczce znamionowej.

Tabliczka znamionowa znajduje się na dole środkowego słupka nadwozia samochodu.
% FIXME: Rys.
A Maksymalna dozwolona masa całkowita
B Maksymalny dozwolony ciężar zestawu (pojazd ciągnący i przyczepa)
C Maksymalny dozwolony nacisk na oś przednią
D Maksymalny dozwolony nacisk na oś tylną

Obciążenie

Na podstawie różnicy między maksymalną dozwoloną masą całkowitą pojazdu a masą własną można określić przybliżone maksymalne obciążenie.

Na obciążenie składają się następujące ciężary:
\begin{itemizeTriangle}
	\itemTriangle Ciężar pasażerów
	\itemTriangle Ciężar całego bagażu i innych obciążeń
	\itemTriangle Ciężar obciążenia dachu wraz z ciężarem bagażnika dachowego
	\itemTriangle Ciężar wyposażenia nieuwzględnionego w masie własnej
\end{itemizeTriangle}

Masa własna

% FIXME: Dać z deklaracji

Informacja o masie eksploatacyjnej

Specyfikacja odpowiada najniższej możliwej masie eksploatacyjnej bez dodatkowego wyposażenia zwiększającego masę. Wliczono w nią także 75 kg jako masę kierowcy, a także masę płynów eksploatacyjnych, zestawu narzędzi i zbiornika paliwa wypełnionego w co najmniej 90\%.

% https://digital-manual.skoda-auto.com/w/pl_PL/show/96c123dbaa98618d6ae9f078ea1cde9a_10_pl_PL?ct=96c123dbaa98618d6ae9f078ea1cde9a_10_pl_PL#titled72747503e204180

Typ silnika	Skrzynia biegów	Masa własna (kg)
2,0 l/140 kW TSI	Automatyczna skrzynia biegów 4x4	1590


Wymiary samochodu

% https://digital-manual.skoda-auto.com/w/pl_PL/show/1ca1c0ab02f3e283c7defc375daf295c_9_pl_PL?ct=1ca1c0ab02f3e283c7defc375daf295c_9_pl_PL#titled72747503e204935

% FIXME: Dać z deklaracji

Wymiary

Dane	Wartość (w mm)

Wysokość pojazdu	1487
Szerokość pojazdu z odchylonymi lusterkami	1829
Szerokość pojazdu ze złożonymi lusterkami	2003
Prześwit pojazdu	161
Długość pojazdu	4703


Specyfikacja silnika

% https://digital-manual.skoda-auto.com/w/pl_PL/show/a5798f436ea34fef078ddc872e4ef279_5_pl_PL?ct=a5798f436ea34fef078ddc872e4ef279_5_pl_PL#titled72747503e209070

Octavia Combi

Liczba cylindrów / pojemność skokowa (cm3)	4/1984
Skrzynia biegów	Automatyczna skrzynia biegów 4x4
Moc (kW/obroty na minutę)	140/4200-6000
Maksymalny moment obrotowy (Nm/obroty na minutę)	320/1500-4100
Maksymalna prędkość (km/h)	229
Z włożonym wiodącym biegiem	6
Przyspieszenie 0-100 km/h (s)	7,2




\section{Specyfikacje}

Płyn hamulcowy -- zgodny z normą VW 501 14 (ta norma jest zgodna z wymaganiami normy FMVSS 116 DOT4).

Jeśli nie ma oleju o prawidłowej specyfikacji, można do następnej wymiany oleju zastosować maks. 0,5 l oleju o następującej specyfikacji:
\begin{itemizeTriangle}
	\itemTriangle VW 504 00, VW 508 00, ACEA C3, ACEA C5
\end{itemizeTriangle}

Specyfikacja
Aby uzupełnić dodatek do płynu chłodzącego, użyć G12evo (TL 774 L).


\section{Ważne uwagi}

\begin{itemizeTriangle}
	\itemTriangle Nie klękaj na fotelach ani nie obciążaj ich punktowo w żaden inny sposób -- ryzyko pęknięcia grzałek.
\end{itemizeTriangle}


\end{document}
